\begin{dfn}[Dual Categories]\hypertarget{dual}{}
  
  Let $\CC \in \obj{\CAT}$. 
  Then the \emph{dual category of $\CC$}, denoted $\CC\op$,
  is defined by : 
  \begin{enumerate}
    \item $\obj{\CC\op} := \obj{\CC}$. 
    \item For all $U, V \in \obj{\CC\op}$,
    $\mor{U}{V}{\CC\op} := \mor{V}{U}{\CC}$.
  \end{enumerate}
\end{dfn}

\begin{dfn}[Contravariant Functors]\hypertarget{contravar}{}
  
  Let $\CC, \DD \in \obj{\CAT}$. 
  Then a \emph{contravariant functor from $\CC$ to $\DD$} is 
  just a functor $\map{\CC\op}{\DD}{\CAT}{}$. 
  Functors $\map{\CC}{\DD}{\CAT}{}$ are henceforth called 
  \emph{covariant functors from $\CC$ to $\DD$}. 
\end{dfn}

% \begin{dfn}[Category of Presheaves on a Category]\hypertarget{presheaf}
%   
%   Let $\CC \in \obj{\CAT}$. 
%   Then the \emph{category of presheaves on $\CC$} is defined as 
%   the category of contravariant functors from $\CC$ to $\SET$. 
%   It is denoted $\PSH{\CC}$. 
% \end{dfn}

\begin{dfn}[Morphism Functor]\hypertarget{mor_funk}{}
  
  Let $\CC$ be a category and $U \in \obj{\CC}$. 
  Then $h_U : \map{\CC\op}{\SET}{\CAT}{}$ is defined as : 
  \begin{enumerate}
    \item For all $V \in \obj{\CC\op}$, 
    $h_U(V) := \mor{V}{U}{\CC}$. 
    \item For all $V, W \in \obj{\CC\op}$ and $f : \map{V}{W}{\CC\op}{}$, 
    $h_U(f) : \map{h_U(V)}{h_U(W)}{}{}, g \mapsto g \circ f$. 
  \end{enumerate}
  Similarly, $h^U : \map{\CC}{\SET}{\CAT}{}$ is defined as : 
  \begin{enumerate}
    \item For all $V \in \obj{\CC}$, 
    $h^U(V) := \mor{U}{V}{\CC}$. 
    \item For all $V, W \in \obj{\CC}$ and $f : \map{V}{W}{\CC}{}$, 
    $h^U(f) : \map{h^U(V)}{h^U(W)}{}{}, 
    g \mapsto f \circ g$. 
  \end{enumerate}
\end{dfn}

\begin{prop}[Morphism Functor is Functorial]\hypertarget{mor_funk_funk}{}
  
  Let $\CC$ be a category. 
  Then $h_{\star} : \map{\CC}{\SET^{\CC\op}}{\CAT}{}$. 
  Similarly, $h^\star : \map{\CC\op}{\SET^\CC}{\CAT}{}$. 
\end{prop}

\begin{rmk}[Functor of Points]\hypertarget{funk_pts}{}
  
  Because of its relevance in algebraic geometry, 
  $h_U$ is called the \emph{functor of points of $U$}. 
\end{rmk}

\begin{prop}[Yoneda's Lemma]\hypertarget{yoneda}{}
  
  Let $\CC$ be a category. 
  Then $h_{\star} : \map{\CC}{\SET^{\CC\op}}{\CAT}{}$ is fully faithful.
  Since \hyperlink{full_faith_inj}{fully faithful functors are injective}, 
  $h_\star$ is called the \emph{Yoneda embedding}. 

  More generally, for any $U \in \obj{\CC}$ and $F \in \obj{\SET^{\CC\op}}$,
  $\mor{h_U}{F}{\SET^{\CC\op}}$ bijects with $F(U)$ via 
  $s \mapsto s_U(\id{U})$ and this bijection is natural in 
  both $U$ and $F$.

  Dually, $h^\star : \map{\CC\op}{\SET^{\CC}}{\CAT}{}$ is fully faithful
  and more generally, 
  for any $U \in \obj{\CC\op}$ and $F \in \obj{\SET^\CC}$,
  $\mor{h^U}{F}{\SET^\CC}$ naturally bijects with $F(U)$ 
  via $s \mapsto s_U(\id{U})$.

\end{prop}
%\begin{proof}
%
%  We first prove the general statement.
%  Let $U \in \obj{\CC}$ and $F \in \obj{\SET^{\CC\op}}$.
%  To show injectivity, let $s, t \in \mor{h_U}{F}{\SET^{\CC\op}}$ and
%  assume $s_U(\id{U}) = t_U(\id{U})$. 
%  To show $s = t$, let $W \in \obj{\CC}$, $f \in h_U(W)$ and
%  consider the following commutative diagram, 
%  \begin{figure}[H]
%    \centering
%    \begin{tikzcd}
%      h_U(U) \arrow[rrr,"s_U"] \arrow[ddd,"h_U(f)"{swap}] 
%      & 
%      & 
%      & F(U) \arrow[ddd,"F(f)"] 
%      \\
%      & \id{U} \ar[r,mapsto] \ar[d,mapsto]& s_U(\id{U}) \ar[d,mapsto]& \\
%      & f \ar[r,mapsto] & s_W(f) = F(f)(s_U(\id{U})) & \\
%      h_U(W) \arrow[rrr,"s_W"{swap}] 
%      & 
%      & 
%      & F(W) 
%    \end{tikzcd}
%  \end{figure}
%  By considering an analogous for $t$,
%  we get $s_W(f) = t_W(f)$.
%  So $s_W = t_W$, and hence $s = t$. 
%  To show surjectivity, let $x \in F(U)$. 
%  Define $s \in \mor{h_U}{F}{\SET^{\CC\op}}$ by 
%  for all $V \in \obj{\CC}$, \[
%    s_V : f \in h_U(V) \mapsto F(f)(x) \in F(V)
%  \]
%  Then by the above diagram, $s_U(\id{U}) = x$. 
%  This proves the desired bijection.
%
%  For naturality in the first component,
%  let $f : \map{U}{V}{\CC}{}$.
%  Then we have the following commutative diagram. 
%  \begin{figure}[H]
%    \centering
%    \begin{tikzcd}
%      \mor{h_U}{F}{\SET^{\CC\op}} \ar[rrr]
%      & & & F(U) \\
%      & s\circ h^f \ar[r,mapsto]
%      & (s\circ h^f)_U (\id{U}) 
%        = s_V(f)
%        = F(f) \circ s_V (\id{V}) & \\
%      & s \ar[u,mapsto] \ar[r,mapsto] & s_V(\id{V}) \ar[u,mapsto] & \\
%      \mor{h_V}{F}{\SET^{\CC\op} } \ar[uuu,"(\star\circ h^f)"] \ar[rrr,""]
%      & & & F(V) \ar[uuu,"F(f)"{swap}]
%    \end{tikzcd}
%  \end{figure}
%  For naturality in the second component, 
%  let $\al : \map{F}{G}{\SET^{\CC\op}}{}$.
%  Then we have the following commutative diagram.
%  \begin{figure}[H]
%    \centering
%    \begin{tikzcd}
%      \mor{h_U}{F}{\SET^{\CC\op}} \ar[rrr]
%        \ar[ddd,"(\al\circ \star)"{swap}]
%      & & & F(U) \ar[ddd,"\al_U"] \\
%      & s \ar[r,mapsto] \ar[d,mapsto]
%      & s_U (\id{U}) \ar[d,mapsto] & \\
%      & \al\circ s \ar[r,mapsto] 
%      & (\al\circ s)_U (\id{U}) = \al_U \circ s_U(\id{U})
%      & \\
%      \mor{h_U}{G}{\SET^{\CC\op} } \ar[rrr,""]
%      & & & G(U)
%    \end{tikzcd}
%  \end{figure}
%  We thus have the desired result. 
%
%  To show $h_\star$ is fully faithful, 
%  let $U, V \in \obj{\CC}$ and apply the above to $F := h_V$.
%\end{proof}
\begin{proof}
  We first prove the general statement. 
  Let $U \in \obj{\CC}, F \in \obj{\SET^{\CC\op}}$.
  Given an element $s \in F(U)$,
  we are tasked with constructing a natural transformation $h_U \to F$.
  For $V \in \CC$ we want to map elements 
  $f \in h_U(V)$ to some element of $F(V)$.
  Well, $f$ is a morphism from $V$ to $U$,
  so $F(f)$ is a morphism from $F(U)$ to $F(V)$,
  and we are given an element $s \in F(U)$.
  So define $\al^s_V : h_U(V) \to F(V) := f \mapsto F(f)(s)$.
  For the collection of $\al^s_V$ to form a natural transformation,
  we need naturality. 
  So given $f \in \CC(V,W)$,
  we need the following diagram to commute : 
  \begin{cd}
    h_U(W) \ar[d,"h_U(f)"] \ar[r,"\al^s_W"] & F(W) \ar[d,"F(f)"]\\
    h_U(V) \ar[r,"\al^s_V"] & F(V)
  \end{cd}
  For $g \in h_U(W)$, then we have as desired \[
    \al^s_V \circ h_U(f) (g) = \al^s_V(g \circ f)
    = F(g \circ f) (x) = F(f) \circ F(g) (x)
    = F(f) \circ \al^s_W (g)
  \]
  So $\al^s : h_U \to F$ is a natural transformation.
  
  Note that we can recover $s$ from $\al^s$ by $\al^s_U(\id{U}) = s$.
  This motivates us to define the inverse map by 
  $\al \in \SET^{\CC\op}(h_U,F) \mapsto \al_U(\id{U})$.
  To show these two maps are indeed inverses, 
  first consider the following diagram where 
  $\al : h_U \to F$ is a natural tranformation,
  $W \in \obj{\CC}$ and $f \in h_U(W)$ : 
  \begin{figure}[H]
    \centering
    \begin{tikzcd}
      h_U(U) \arrow[rrr,"a\l_U"] \arrow[ddd,"h_U(f)"{swap}] 
      & 
      & 
      & F(U) \arrow[ddd,"F(f)"] 
      \\
      & \id{U} \ar[r,mapsto] \ar[d,mapsto]& \al_U(\id{U}) \ar[d,mapsto]& \\
      & f \ar[r,mapsto] & \al_W(f) = F(f)(\al_U(\id{U})) & \\
      h_U(W) \arrow[rrr,"\al_W"{swap}] 
      & 
      & 
      & F(W) 
    \end{tikzcd}
  \end{figure}
  The above diagram commutes by naturality of $\al$.
  What it shows is that $\al_W$ is completely determined by $\al_U(\id{U})$,
  and hence $\al$ is completely determined by $\al_U(\id{U})$.
  This proves one side of the inverse situation. 
  The other side is clear.
  Thus we have a bijection between $\SET^{\CC\op}(h_U,F) \iso F(U)$.

  At this point, we can already get $h_\star$ fully faithful by
  applying the above bijection to $F = h_\star$ itself and noting
  the bijection turns $f \in h_V(U)$ into $h_f$. 

  For naturality in the first component,
  let $f : \map{U}{V}{\CC}{}$.
  Then we have the following commutative diagram. 
  \begin{figure}[H]
    \centering
    \begin{tikzcd}
      \mor{h_U}{F}{\SET^{\CC\op}} \ar[rrr]
      & & & F(U) \\
      & \al\circ h^f \ar[r,mapsto]
      & ( \al\circ h^f)_U (\id{U}) 
        = \al_V(f)
        = F(f) \circ \al_V (\id{V}) & \\
      & \al \ar[u,mapsto] \ar[r,mapsto] & \al_V(\id{V}) \ar[u,mapsto] & \\
      \mor{h_V}{F}{\SET^{\CC\op} } \ar[uuu,"(\star\circ h^f)"] \ar[rrr,""]
      & & & F(V) \ar[uuu,"F(f)"{swap}]
    \end{tikzcd}
  \end{figure}
  For naturality in the second component, 
  let $\phi : \map{F}{G}{\SET^{\CC\op}}{}$.
  Then we have the following commutative diagram.
  \begin{figure}[H]
    \centering
    \begin{tikzcd}
      \mor{h_U}{F}{\SET^{\CC\op}} \ar[rrr]
        \ar[ddd,"(\phi\circ \star)"{swap}]
      & & & F(U) \ar[ddd,"\phi_U"] \\
      & \al \ar[r,mapsto] \ar[d,mapsto]
      & \al_U (\id{U}) \ar[d,mapsto] & \\
      & \phi\circ \al \ar[r,mapsto] 
      & (\phi\circ \al)_U (\id{U}) = \phi_U \circ \al_U(\id{U})
      & \\
      \mor{h_U}{G}{\SET^{\CC\op} } \ar[rrr,""]
      & & & G(U)
    \end{tikzcd}
  \end{figure}
  We thus have the desired result. 
\end{proof}

\begin{dfn}[Representable Functors]
  \hypertarget{rep}{}
  
  Let $G : \map{\CC}{\SET}{\CAT}{}$ be a covariant functor. 
  Then a \emph{representation of $G$} is a $(U,u) \in h^\star\darrow G$
  where $u : \map{h^\star}{G}{\SET^\CC}{\sim}$.

  Dually, let $F : \map{\CC\op}{\SET}{}{}$ be a contravariant functor. 
  Then a \emph{representation of $F$} is a $(U,u) \in h_\star\darrow F$
  where $u : \map{h_\star}{F}{\SET^{\CC\op}}{\sim}$.

  A functor (covariant or contravariant) that has a representation is called 
  \emph{representable}.
\end{dfn}

\begin{rmk}
  If a functor has a representation,
  Yoneda's lemma implies it is canonical.
  This is the \hyperlink{canonical_rep}{next result}. 

  Before this, we first relate universal morphisms to representable functors.
  This is important as it leads to the notion of \emph{adjunction}. 
\end{rmk}

\begin{prop}[Universal iff Represents]
  \hypertarget{uni_iff_rep}{}
  
  Let $R : \map{\CC}{\DD}{\CAT}{}$, $X \in \obj{\DD}$,
  $(L(X),\eta_X) \in \obj{X\darrow R}$. 
  Then the following are equivalent : 
  \begin{enumerate}
    \item $(L(X),\eta_X)$ is a universal morphism from $X$ to $R$.
    \item $L(X)$ represents the covariant functor $\mor{X}{R(\star)}{\DD}$
    and $\id{L(X)}$ corresponds to $\eta_X$. 
  \end{enumerate}

  Dually, let $L : \map{\DD}{\CC}{\CAT}{}$, $U \in \obj{\CC}$,
  $(R(U),\ep_U) \in \obj{L\darrow U}$. 
  Then the following are equivalent : 
  \begin{enumerate}
    \item $(R(U),\ep_U)$ is a universal morphism from $L$ to $U$.
    \item $R(U)$ represents the contravariant functor $\mor{L(\star)}{U}{\CC}$
    and $\id{R(U)}$ corresponds to $\ep_U$.
  \end{enumerate}
\end{prop}
\begin{proof}
  (Universal implies Represents) 
  Let $(L(X),\eta_X)$ be a universal morphism from $X$ to $R$. 
  Define the following natural transformation,
  \begin{align*}
    &\map{h^{L(X)}}{\mor{X}{R(\star)}{\DD}}{\SET^\CC}{} := \\
    &W \in \obj{\CC} \mapsto \sqbrkt{
      f \in h^{L(X)}(W) \mapsto R(f) \circ \eta_X \in \mor{X}{R(W)}{\DD}
    }
  \end{align*}
  Then for every $W \in \obj{\CC}$, 
  this is an isomorphism between $h^{L(X)}(W)$ and $\mor{X}{R(W)}{\DD}$,
  and hence a natural isomorphism. 
  Indeed, $\id{L(X)}$ corresponds to $\eta_X$ under this natural isomorphism.

  (Represents implies Universal)
  Let $\al : \map{h^{L(X)}}{\mor{X}{R(\star)}{\DD}}{\SET^\CC}{}$
  be a natural isomorphism where at $L(X)$,
  $\al_{L(X)}(\id{L(X)}) = \eta_X$.
  Let $(V,v) \in \obj{X\darrow R}$.
  For any $f : \map{L(X)}{V}{\CC}{}$,
  consider the following commutative diagram. 
  \begin{figure}[H]
    \centering
    \begin{tikzcd}
      h^{L(X)}(L(X)) \arrow[rrr,"\al_{L(X)}"] \arrow[ddd,"h^{L(X)}(f)"{swap}] 
      & 
      & 
      & \mor{X}{RL(X)}{\DD} \arrow[ddd,"\mor{X}{R(f)}{\DD}"] \\
      & \id{L(X)} \ar[r,mapsto] \ar[d,mapsto]
      & \al_{L(X)}(\id{L(X)}) = \eta_X \ar[d,mapsto] 
      & \\
      & f \ar[r,mapsto] 
      & \al_V(f) = R(f) \circ \eta_X 
      & \\
      h^{L(X)}(V) \arrow[rrr,"\al_V"{swap}] 
      & 
      & 
      & \mor{X}{R(V)}{\DD}
    \end{tikzcd}
  \end{figure}
  Thus $f : \map{(L(X),\eta_X)}{(V,v)}{X\darrow R}{}$
  if and only if $\al_V(f) = v$.
  Then $\al_V\inv(v)$ is 
  the unique morphism $\map{(L(X),\eta_X)}{(V,v)}{X\darrow R}{}$.
  Since there exists a unique $\map{(L(X),\eta_X)}{(V,v)}{X\darrow R}{}$,
  $(L(X),\eta_X)$ is universal.

  The dual equivalence has an analogous proof.
\end{proof}

\begin{prop}[Canonical Representation]
  \hypertarget{canonical_rep}{}
  
  Let $G : \map{\CC}{\DD}{\CAT}{}$ and $(U,u) \in h^\star\darrow G$.
  Then the following are equivalent : 
  \begin{enumerate}
    \item $(U,u)$ is a representation of $G$.
    \item $(U,u)$ is a universal morphism from $h^\star$ to $G$.
  \end{enumerate}
  In particular, 
  representations of $G$ are canonically isomorphic. 

  Dually, let $F : \map{\CC\op}{\DD}{\CAT}{}$ and $(V,v) \in h_\star\darrow F$.
  Then the following are equivalent : 
  \begin{enumerate}
    \item $(V,v)$ is a representation of $F$.
    \item $(V,v)$ is a universal morphism from $h_\star$ to $F$.
  \end{enumerate}
  In particular, 
  representations of $F$ are canonically isomorphic. 
\end{prop}
\begin{proof}
  (Representation implies Universal)
  Let $(W,w) \in \obj{h^\star\darrow G}$.
  Then $u\inv \circ w : \map{h^W}{h^U}{\SET^\CC}{}$.
  By \hyperlink{yoneda}{Yoneda's lemma},
  there exists a unique $u(W,w) : \map{U}{W}{\CC}{}$ such that 
  $u\inv \circ w = h^{u(W,w)}$.
  Hence $u(W,w)$ is the unique morphism 
  $\map{(W,w)}{U,u}{h^\star\darrow G}{}$.

  (Universal implies Representation)
  By \hyperlink{uni_iff_rep}{universal iff represents}
  and \hyperlink{yoneda}{Yoneda's lemma},
  we have the following diagram. 
  \begin{figure}[H]
    \centering
    \begin{tikzcd}[sep = huge]
      \mor{h^\star}{G}{\SET^\CC} 
        \ar[r,"V \in \obj{\CC} \mapsto \sqbrkt{
          s \in \mor{h^V}{G}{\SET^\CC} \mapsto s_V(\id{V})
        }"{yshift = 3mm},"\sim"{swap}]
      & G \\
      h^U \ar[u,
        "V \in \obj{\CC} \mapsto \sqbrkt{
          f \in h^U(V) \mapsto u \circ h^f
        }",
        "\sim"{swap}
        ]
        \ar[ur,"u"]
      &
    \end{tikzcd}
  \end{figure}
  The claim is that the above commutes, and hence $u$ is an isomorphism.
  Let $V \in \obj{\CC}$ and $f \in h^U(V)$.
  Then \begin{align*}
    (h^f \circ u)_V(\id{V})
    = u_V \circ \brkt{h^f}_V \brkt{\id{V}}
    = u_V (f)
  \end{align*}
  So the above diagram commutes. 

  For the dual, the argument is similar. 
\end{proof}