\begin{dfn}[Finite Categories]\link{fin}
  
  Let $\CC$ be a category. 
  Then it is called \emph{finite} when 
  $\obj{\CC}$ is finite and for any $U, V \in \obj{\CC}$,
  $\CC(U,V)$ is finite. 
\end{dfn}

\begin{dfn}[(Finite) (Co)Completeness]\link{complete}
  
  Let $\CC$ be a category. 
  Then it is called \emph{(finitely) complete} when 
  for all (finite) categories $\II$ and diagrams $X : \map{\II}{\CC}{\CAT}{}$,
  there exists the limit of $X$ in $\CC$.

  Dually, it is called \emph{(finitely) cocomplete} when 
  for all (finite) categories $\II$ and 
  diagrams $X : \map{\II}{\CC}{\CAT}{}$,
  there exists the colimit of $X$ in $\CC$.
\end{dfn}

\begin{rmk}
  If a category $\CC$ has nice constructions,
  then for any category $\II$, $\CC^\II$ will often inherit these constructions.
  This is the following result. 
\end{rmk}

\begin{prop}[Characterisation of (Co)Limits of Functors]
  \link{char_lim_funk}
  
  Let $\CC, \DD$ be categories.
  Let $X : \II \to \DD^\CC$ be an $\II$-shaped diagram of functors from 
  $\CC$ to $\DD$.

  Let $(F,\phi) \in \obj{\De\darrow X}$.
  For each $U \in \obj{\CC}$,
  we obtain a functor $X_U : \II \to \DD, i \in \obj{\II} \mapsto X(i,U)$
  by restricting the second argument to $U$
  (mapping morphisms is obvious). 
  In a similar way, we obtain $(F(U),\phi_U) \in \obj{\De\darrow X_U}$.
  Then the following are equivalent : 
  \begin{itemize}
    \item For all $U \in \obj{\CC}$, 
    $(F(U),\phi_U)$ is a limit of $X_U$.
    \item $(F,\phi)$ is a limit of $X$ and for every $U\in\obj{\CC}$,
    the limit of $X_U$ exists in $\DD$.
  \end{itemize}
  Paraphrasing : if $\DD$ has limits of $\II$-shaped diagrams,
  then so does $\DD^\CC$.

  Dually, let $(F,\phi) \in \obj{X\darrow\De}$.
  For each $U \in \obj{\CC}$,
  we obtain a functor $X^U : \II \to \DD, i \in \obj{\II} \mapsto X(i,U)$
  (mapping morphisms is obvious). 
  In a similar way, we obtain $(F(U),\phi^U) \in \obj{X^U\darrow\De}$.
  Then the following are equivalent : 
  \begin{itemize}
    \item For all $U \in \obj{\CC}$, 
    $(F(U),\phi^U)$ is a colimit of $X^U$.
    \item $(F,\phi)$ is a colimit of $X$ and for every $U \in \obj{\CC}$,
    the colimit of $X^U$ exists in $\DD$.
  \end{itemize}
  Paraphrasing : if $\DD$ has colimits of $\II$-shaped diagrams,
  then so does $\DD^\CC$.
\end{prop}
\begin{proof}
  
  $(\implies)$
  Suppose for all $U \in \obj{\CC}$, 
  $(F(U),\phi_U)$ is a limit of $X_U$.
  We show that $(F,\phi)$ has the desired UP.
  Let $(G,\ga) \in \obj{\De\darrow X}$.
  For each $U \in \obj{\CC}$,
  define $(G(U),\ga_U) \in \obj{\De\darrow X_U}$
  by restricting the second argument to $U$.
  Then by the UP of limits, 
  we have a unique morphism $\LIM \ga_U \in 
  \De\darrow X_U((G(U),\ga_U),(F(U),\phi_U))$.
  \begin{cd}
    \De(G(U)) \ar[r,"\De(\LIM \ga_U)", dashed] \ar[rd,"\ga_U"{swap}] & 
    \De(F(U)) \ar[d,"\phi_U"] &
    \De(G) \ar[r,"\De(\LIM \ga)", dashed] \ar[rd,"\ga"{swap}] & 
    \De(F) \ar[d,"\phi"] \\
    & 
    X_U &
    & 
    X
  \end{cd}
  Then again by the UP of limits,
  the left triangle is functorial in $U$,
  so the collection of $(\LIM \ga_U)_{U \in \obj{\CC}}$ forms a 
  $\LIM \ga \De\darrow X((G,\ga),(F,\phi))$.
  The uniqueness of $\LIM \ga$ comes from 
  the componentwise uniqueness of $(\LIM \ga_U)_{U \in \obj{\CC}}$.

  $(\limplies)$
  Let $(F,\phi)$ be a limit of $X$ and 
  suppose for every $U \in \obj{\CC}$,
  the limit of $X_U$ exists. 
  Then we can use the forward implication to explicitly construct a limit 
  $(\bar{F},\bar{\phi})$ of $X$ by defining
  the components $(\bar{F}(U),\bar{\phi}_U)$ to be limits of $X_U$.
  Functoriality of $\bar{F}$ comes from the UP of limits in each component.
  Since limits are canonical,
  we have a unique isomorphism between $(F,\phi)$ and $(\bar{F},\bar{\phi})$.
  Restricting to each component yields an isomorphism of cones 
  $(F(U),\phi_U) \iso (\bar{F}(U),\bar{\phi}_U)$ for every $U \in \obj{\CC}$.
  Since $(\bar{F}(U),\bar{\phi}_U)$ is by construction a limit of $X_U$,
  and being isomorphic to the limit \linkto{iso_uni_implies_uni}{implies}
  being a limit.
  We conclude that $(F(U),\phi_U)$ is a limit of $X_U$ for every $U$.

  The dual argument is similar. 
\end{proof}

\begin{rmk}
  The following result gives an easier check for (co)completeness.
\end{rmk}

\begin{prop}[Characterisation of Completeness, Cocompleteness]
  \link{char_complete}
  
  Let $\CC$ be a category. 
  Then the following are equivalent : 
  \begin{itemize}
    \item $\CC$ is (finitely) complete.
    \item $\CC$ has (finite) products and (finite) equalizers.
  \end{itemize}
  Dually, the following are equivalent : 
  \begin{itemize}
    \item $\CC$ is (finitely) cocomplete.
    \item $\CC$ has (finite) coproducts and (finite) coequalizers. 
  \end{itemize}
\end{prop}
\begin{proof}
  For fun, we prove the dual argument instead for once.
  It suffices to prove the reverse implication. 
  Let $X : \II \to \CC$ be a diagram. 
  Define the \emph{arrow category of $\II$} as follows : 
  take the free category on the partially ordered set $\2 := \set{0,1}$ and 
  $\arr(\II) := \II^{\2}$.
  Define \emph{source} and \emph{target} functions 
  $s, t : \obj{\arr(\II)} \to \obj{\II}, (\phi : i \to j) \mapsto i, j$
  respectively. 
  Let $(\coprod_{\phi \in \obj{\arr\II}} X_{s(\phi)},\io)$
  be the coproduct of $X_{s(\phi)}$ across all arrows $\phi$ in $\II$ and 
  $(\coprod_{i \in \obj{\II}} X_i,\lift{}{})$ the coproduct of 
  $X(i)$ across all $i$ in $\II$.
  Now consider the following diagram : 
  \begin{cd}
    \coprod_{\phi \in \obj{\arr\II}} X_{s(\phi)} 
      \ar[r,"\al",yshift = 1mm] \ar[r,"\be"{swap},yshift = -1mm]&
    \coprod_{i \in \obj{\II}} X_i
  \end{cd}
  where $\al$ is defined using the UP of the coproduct via 
  for all $\phi \in \obj{\arr\II}$, $\al \circ \io_\phi := \lift{s(\phi)}{}$;
  $\be$ is also defined using the UP of the coproduct, but via 
  for all $\phi \in \obj{\arr\II}$, 
  $\be\circ \io_\phi := \lift{t(a)}{} \circ X(\phi)$. 
  It is then straightforward to check that 
  the coequalizer of $\al,\be$ with obvious morphisms from $X$ into it
  is a colimit of $X$. 
  Note that the above proof still works for finite categories $\II$. 

  The non-dual argument is similar. 
\end{proof}

\begin{rmk}
  We now proceed to show the (co)completeness of some standard categories.
  As this is unimportant to the theory, 
  the reader may skip to the next section. 
  
  For fun, we will not use the above characterisation,
  though the constructions are of similar nature. 
\end{rmk}

\begin{prop}[$\SET$ (Co)Complete]\link{set_complete}
  
  Let $I$ be a category and $X$ an $I$-shaped diagram in $\SET$.

  Consider the set $\SET^\II(\De(1),X)$ where $1$ is a singleton set.
  There is an obvious $\pi_X : \De(\SET^\II(\De(1),X)) \to X$.
  Then $(\SET^\II(\De(1),X),\pi_X)$ is a limit of $X$ in $\SET$.

  Dually, we construct the colimit of $X$ as follows. 
  For every $(i,x_i) \in \bigsqcup_{i \in I} X_i$,
  define $[(i,x_i)]$ to be the subset of $(j,x_j) \in \bigsqcup_{i \in I} X_i$
  such that there exists 
  a ``zig zag of elements and morphisms in $X$ joining them''
  (unenlightening to write down fully) : 
  \begin{cd}[cramped]
    (i,x_i) \ar[rd] & & 
    (k_1, x_1) \ar[ld] \ar[rd] & & 
    (k_n, x_n) \ar[ld] \ar[rd] & & 
    (j,x_j) \ar[ld] \\
    & (l_0,y_0) & 
    & \cdots & 
    & (l_n,y_n) & 
  \end{cd}
  Then $Q := \set{[(i,x_i)] \st (i,x_i) \in \bigsqcup_{i \in I} X(i)}$ forms
  a partition of $\bigsqcup_{i \in I} X(i)$,
  corresponding to the minimal equivalence relation generated by 
  $(i,x_i) \sim (j,x_j) := \exists \phi \in I(i,j), X(\phi)(x_i) = x_j$.
  There is an obvious morphism $\io_X : X \to \De(Q)$.
  Then $(Q,\io_X)$ is a colimit of $X$.

\end{prop}
\begin{proof}
  Straightforward.
\end{proof}

\begin{prop}[$\TOP$ Complete]\link{top_complete}
  
  Let $I$ be a category and $X$ an $I$-shaped diagram in $\TOP$.

  Let $(\LIM_\SET X,\pi_X)$ be the limit of $X_\SET$,
  obtained by forgeting from $\TOP$ to $\SET$.
  Let $\LIM X$ be endowed with 
  the topology generated by preimages of opens in components. 
  Then for all $(Y,y) \in \obj{\De\darrow X_\SET}$ where $Y \in \obj{\TOP}$
  and $\bar{y} \in \De\darrow X((Y,y),(\LIM_\SET X,\pi_X))$,
  $\bar{y} \in \TOP(Y,\LIM X)$ if and only if 
  $y \in \TOP^\II(\De(Y),X)$.
  Hence, $\pi_X \in \TOP^\II(\De(\LIM X), X)$ and 
  $(\LIM X, \pi_X)$ is a limit of $X$ in $\TOP$.

  Dually, similar trick to turn the colimit in $\SET$ to colimit in $\TOP$ : 
  take the topology generated by 
  sets whose preimages are open in every component of the diagram $X$.
\end{prop}
\begin{proof}
  Straightforward.
\end{proof}

\begin{prop}[$\GRP$ Complete]\link{grp_complete}
  
  Let $I$ be a category and $X$ an $I$-shaped diagram in $\GRP$.

  The construction $\SET^\II(\De(1),X)$ from 
  the limit of $X$ as a diagram in $\SET$ has a group structure by
  multiplying ``objectwise'' in the diagram $X$.
  This makes $\pi_X \in \GRP^{\II}(\De(\SET^\II(\De(1),X)),X)$,
  and hence $\SET^\II(\De(1),X)$ a limit of $X$ in $\GRP$.

  Dually, we imitate the colimit construction in $\SET$ and consider 
  the free product quotiented by the subgroup generated by 
  ``eventually being equal in $X$'' : 
  \[
    \brkt{\bigsqcup_{i \in\II} X_i} /
    \<(i,x_i)(j,x_j)\inv \st \exists\,\phi\in\II(i,j), X(\phi)(x_i) = x_j\>
  \]
  Then the above is a colimit of $X$.

\end{prop}
\begin{proof}
  Straightforward. 
\end{proof}

\begin{rmk}
  We will show (co)completeness of modules over a ring first,
  then show (co)completeness of rings,
  since for the case of colimits, 
  we will make use of the fact that $\RING$ forgets into $\AB = \Z\MOD$. 
\end{rmk}

\begin{prop}[$R\MOD$ Complete]\link{mod_complete}
  
  Let $R \in \RING$, $\II$ a category, 
  $X$ an $\II$-diagram in $R\MOD$.

  The construction $\SET^\II(\De(1),X)$ from 
  the limit of $X$ as a diagram in $\SET$ has an $R\MOD$ structure by
  adding and scalar multiplying ``objectwise'' in the diagram $X$.
  This gives the morphism $\pi_X \in R\MOD^{\II}(\De(\SET^\II(\De(1),X)),X)$,
  and hence $\SET^\II(\De(1),X)$ a limit of $X$ in $R\MOD$.

  Dually, we imitate the colimit construction in $\SET$ and consider 
  the direct sum quotiented by the left $R$-submodule generated by 
  ``eventually being equal in $X$'' : 
  \[
    \brkt{\bigoplus_{i\in\II} X_i} / 
    R\set{(i,x_i) - (j,x_j) \st \exists\, \phi\in\II(i,j), X(\phi)(x_i) = x_j}
  \]
  Then the above is a colimit for $X$ in $R\MOD$.
\end{prop}
\begin{proof}
  Straightforward.
\end{proof}

\begin{prop}[$\RING$ Complete]\link{ring_complete}
  
  Let $\II$ be a category and $X$ an $\II$-shaped diagram in $\RING$.

  The construction $\SET^\II(\De(1),X)$ from 
  the limit of $X$ as a diagram in $\SET$ has a ring structure by
  adding and multiplying ``objectwise'' in the diagram $X$.
  Then $\pi_X \in \RING^{\II}(\De(\SET^\II(\De(1),X)),X)$,
  and hence $\SET^\II(\De(1),X)$ a limit of $X$ in $\RING$.

  Dually, things get nasty. 
\end{prop}
\begin{proof}
  (Nasty things go here)
  (TODO)
\end{proof}

\begin{rmk}
  One may wonder : 
  (1) Why are all the limits for these categories the same as in $\SET$? 
  (2) Why do the colimits differ to $\SET$ for everything except $\TOP$?

  The \linkto{adjoint_cts}{answer} is the related to adjunctions,
  and is the focus of the next section. 
  In a nutshell : 
  all of the following categories are subcategories of $\SET$
  which admit free functors. 
  This implies that the forgetful functor preserves limits.
  This answers the first question. 
  This is also why colimits differ to the ones in $\SET$
  \emph{except} for the case of $\TOP$,
  since the forgetful functor also admits a \emph{right adjoint},
  namely taking the indiscrete topology. 
  This then implies the forgetful functor also preserves colimits.
  This answers the second question.
\end{rmk}