\begin{dfn}[Categories]\link{cat}
  
  A \emph{category $\CC$} is defined by the following data : 
  \begin{enumerate}
    \item A set of \emph{objects}, $\obj{\CC}$. 
    \item For every $U, V \in \obj{\CC}$, 
    a set of \emph{$\CC$-morphisms} from $U$ to $V$,
    denoted $\mor{U}{V}{\CC}$.
    We denote $f : \map{U}{V}{\CC}{}$ for $f \in \mor{U}{V}{\CC}$. 
    \item For every $U, V, W \in \obj{\CC}$, 
    $f : \map{U}{V}{\CC}{}$ and $g : \map{V}{W}{\CC}{}$, 
    a $\CC$-morphism called the \emph{composition of $f$ with $g$},
    denoted $g \circ f : \map{U}{W}{\CC}{}$.
    \item Associativity of $\circ$.
    \item For every $U \in \obj{\CC}$, 
    an \emph{identity morphism} $\id{U} : \map{U}{U}{\CC}{}$.
    \item For all $U, V, W \in \obj{\CC}$, 
    $f : \map{U}{V}{\CC}{}$ and $g : \map{W}{U}{\CC}{}$, 
    we have $f \circ \id{U} = f$ and $\id{U} \circ g = g$. 
  \end{enumerate}
\end{dfn}

\begin{rmk}
Morphisms in a category do \emph{not} have to be functions.
See the example of preordered sets as categories at
\linkto{eg:cat_ord}{end of this section}.
\end{rmk}

\begin{dfn}[Isomorphisms]\link{iso}
  
  Let $\CC$ be a category, $U, V \in \obj{\CC}$, $f : \map{U}{V}{\CC}{}$.
  Then $f$ is called an \emph{isomorphism} when 
  there exists $g : \map{V}{U}{\CC}{}$ such that 
  $g \circ f = \id{U}$ and $f \circ g = \id{V}$.
  In this case, we denote $f : \map{U}{V}{\CC}{\sim}$.
  When there exists an isomorphism from $U$ to $V$, 
  we say they are \emph{isomorphic} and write $U \iso V$. 
\end{dfn}

\begin{dfn}[Subcategories]\link{subcat}
  
  Let $\CC,\DD$ be categories. 
  Then $\DD$ is called a \emph{subcategory of $\CC$} when 
  $\obj{\DD} \subs \obj{\CC}$ and 
  for all $U,V \in \obj{\DD}$, $\mor{U}{V}{\DD} \subs \mor{U}{V}{\CC}$.
\end{dfn}

\begin{eg}[Standard Categories]\link{eg:cat}~
  \begin{enumerate}
    \item $\SET$ denotes the \emph{category of sets}, where 
    $\obj{\SET}$ contains sets and for $U, V \in \obj{\SET}$, 
    $\SET(U,V)$ is the set of maps from $U$ to $V$. 
    \item $\TOP$ denotes the \emph{category of topological spaces}, where
    $\obj{\TOP}$ contains topological spaces and for $U, V \in \obj{\TOP}$, 
    $\TOP(U,V)$ is the set of continuous maps from $U$ to $V$. 
    $\TOP$ is a subcategory of $\SET$. 
    \item The \emph{category of groups} $\GRP$ has 
    $\obj{\GRP}$ containing groups and 
    $\GRP(U,V)$ containing group homomorphisms from $U$ to $V$. 
    $\GRP$ is a subcategory of $\SET$.  
    \item The \emph{category of abelian groups} $\AB$ has 
    $\obj{\AB}$ containing abelian groups and 
    $\AB(U,V)$ containing group homomorphisms from $U$ to $V$. 
    $\AB$ is a subcategory of $\GRP$. 
    \item The \emph{category of rings} $\RING$ has 
    $\obj{\RING}$ containing rings and 
    $\RING(U,V)$ containing ring homomorphisms from $U$ to $V$.
    $\RING$ is a subcategory of $\SET$. 
    \item The \emph{category of commutative rings} $\CRING$ has 
    $\obj{\CRING}$ containing commutative rings and 
    $\CRING(U,V)$ containing ring homomorphisms from $U$ to $V$.
    $\CRING$ is a subcategory of $\RING$. 
    \item Let $R$ be a ring. 
    Then the \emph{category of left $R$-modules} $R\MOD$ has 
    $\obj{R\MOD}$ containing left $R$-modules and 
    $R\MOD(U,V)$ contains $R$-linear maps from $U$ to $V$.
    This is a subcategory of $\AB$. 
    \item Let $R$ be a commutative ring. 
    Then the \emph{category of $R$-algebras} $R\ALG$ has 
    $\obj{R\ALG}$ containing pairs $(S,\si)$ where 
    $\si : \map{R}{S}{\RING}{}$.
    $R\ALG((U,u),(V,v))$ contains $f : \map{U}{V}{\RING}{}$ such that 
    $f \circ u = v$. 
  \end{enumerate}
\end{eg}

\begin{eg}[Preordered Sets as Categories]\link{eg:cat_ord}
  
  Let $I$ be a set, $\leq$ a relation on $I$. 
  Then $(I,\leq)$ is called a \emph{preordered set} when 
  $\leq$ satisfies all of the following : 
  \begin{enumerate}
    \item (Reflexivity) For all $i \in I$, $i \leq i$. 
    \item (Transitivity) For all $i, j, k \in I$, 
    $i \leq j$ and $j \leq k$ implies $i \leq k$. 
  \end{enumerate}
  If $(I,\leq)$ is a preordered set where $\leq$ is clear, 
  we abbreviate to $I$.

  Let $I$ be a preordered set. 
  Then we can turn $I$ into a category as follows : 
  \begin{enumerate}
    \item $\obj{I}$ is $I$. 
    \item For $i, j \in \obj{I}$, $I(i,j)$ is 
    singleton when $i \leq j$ and empty otherwise.  
  \end{enumerate}

  Things get meta. 
  We can form the \emph{category of preordered sets} $\ORD$ where 
  $\obj{\ORD}$ contains preoredered sets and 
  $\ORD(I,J)$ contains $f : \map{I}{J}{\SET}{}$ such that 
  for all $i, j \in I$, $i \leq j$ implies $f(i) \leq f(j)$.
\end{eg}

\begin{eg}[Category of Partially Ordered Sets]\link{eg:cat_poset}
  
  Let $I \in \obj{\ORD}$. 
  Then $I$ is called a \emph{partially ordered set} when 
  $\leq$ is \emph{antisymmetric}, i.e.
  for all $i, j \in I$, $i \leq j$ and $j \leq i$ implies $i = j$. 
  We thus have the \emph{category of partially ordered sets} $\POSET$ where 
  $\obj{\POSET}$ contains partially ordered sets and 
  $\POSET(I,J) = \ORD(I,J)$. 
  We see that $\POSET$ is a subcategory of $\ORD$.
\end{eg}

\begin{eg}[Partially Ordered Sets]\link{eg:poset}~
  \begin{enumerate}
    \item Let $X$ be a set. 
    Then its powerset $(2^X,\subs) \in \obj{\POSET}$.
    \item Let $X$ be a topological space. 
    Then the set of its opens $(\tau_X,\subs) \in \obj{\POSET}$.
    \item Let $G$ be a group. 
    Then the set of its subgroups $(\SUB\GRP(G),\subs)\in\obj{\POSET}$.
    \item Let $R$ be a ring and $M$ be a left $R$-module.
    Then the set of left $R$-submodules of $M$, $R\SUB\MOD(M)$,
    is in $\obj{\POSET}$.
    \item Let $R$ be a commutative ring and $(S,\si)$ an $R$-algebra. 
    Then the set of all $R$-subalgebras of $S$, $R\SUB\ALG(S)$, 
    is in $\obj{\POSET}$. 
    \item Let $X$ be a set and $\fil X$ the set of all filters on $X$.
    Then $(\fil X,\subs) \in \obj{\POSET}$. 
    \item Consider the relation on $\N$ that is $a \dvd b$. 
    This is a partial order on $\N$. 
  \end{enumerate}
\end{eg}

\begin{eg}[A Group as a Category]\link{eg:cat_group}
  
  The data of a group $G$ is equivalent to 
  a category $G$ where there is only one object $\bullet$ and 
  all morphisms are isomorphisms. 

  A direct generalization is a \emph{groupoid} : 
  a category where every morphism is an isomorphism. 
\end{eg}

\begin{eg}[Vector Spaces together with a Basis]\link{eg:vec_basis}
  Let $K$ be a field. 
  Define a category $\CC$ as follows : 
  \begin{itemize}
    \item objects are pairs $(V,B)$ where $V$ is a $K$-vector space and 
    $B$ is a basis of $V$. 
    \item For $(V,B_V), (W,B_W)$ objects in $\CC$, 
    define $\CC((V,B_V),(W,B_W))$ as the set of $K$-linear maps 
    from $V$ to $W$ such that maps $B_V$ into $B_W$.
    \item For every $K$-vector space with a basis $(V,B_V)$, 
    $\id{(V,B_V)}$ is defined to be the identity map of $V$. 
    \item Composition of the underlying $K$-linear maps of morphisms yields
    another morphism in this category. 
  \end{itemize}
  This category has \linkto{eg:functors}{nice connections} to change of basis. 
\end{eg}