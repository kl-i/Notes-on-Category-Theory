\documentclass{article}

\usepackage[left=1in,right=1in]{geometry}
\usepackage{subfiles}
\usepackage{amsmath, amsthm, amssymb, verbatim} % thms
\usepackage{hyperref,nameref,cleveref,thmtools,enumitem} % Easy to use thms
\usepackage[dvipsnames]{xcolor} % Fancy Colours
\usepackage{mathrsfs} % Fancy font
\usepackage{tikz, tikz-cd, float} % Commutative Diagrams
\usepackage{mdframed} % Customizable Boxes
\usepackage{perpage}
\usepackage{parskip} % So that paragraphs look nice
\usepackage{ifthen,xargs} % For defining better commands

% Shortcuts

% % Misc
\newcommand{\brkt}[1]{\left(#1\right)}
\newcommand{\sqbrkt}[1]{\left[#1\right]}
\newcommand{\dash}{\text{-}}

% % Logic
\renewcommand{\implies}{\Rightarrow}
\renewcommand{\iff}{\Leftrightarrow}
\newcommand{\limplies}{\Leftarrow}

% % Sets
\DeclareMathOperator{\supp}{supp}
\newcommand{\set}[1]{\left\{#1\right\}}
\newcommand{\st}{\,|\,}
\newcommand{\minus}{\setminus}
\newcommand{\subs}{\subseteq}
\newcommand{\ssubs}{\subsetneq}
\DeclareMathOperator{\im}{Im}
\newcommand{\nothing}{\varnothing}

% % Greek 
\newcommand{\al}{\alpha}
\newcommand{\be}{\beta}
\newcommand{\ga}{\gamma}
\newcommand{\de}{\delta}
\newcommand{\ep}{\varepsilon}
\newcommand{\io}{\iota}
\newcommand{\ka}{\kappa}
\newcommand{\la}{\lambda}
\newcommand{\om}{\omega}

% % Mathbb
\newcommand{\N}{\mathbb{N}}
\newcommand{\Z}{\mathbb{Z}}
\newcommand{\Q}{\mathbb{Q}}
\newcommand{\R}{\mathbb{R}}
\newcommand{\C}{\mathbb{C}}
\newcommand{\F}{\mathbb{F}}
\newcommand{\bP}{\mathbb{P}}

% % Mathcal
\newcommand{\CC}{\mathcal{C}}
\newcommand{\DD}{\mathcal{D}}
\newcommand{\EE}{\mathcal{E}}

% % Mathfrak
\newcommand{\f}[1]{\mathfrak{#1}}

% % Mathrsfs
\newcommand{\s}[1]{\mathscr{#1}}

% % Category Theory
\newcommand{\obj}[1]{\mathrm{Obj}\left(#1\right)}
\newcommand{\Hom}[3]{\mathrm{Hom}_{#3}(#1, #2)\,}
\newcommand{\mor}[3]{\mathrm{Mor}_{#3}(#1, #2)\,}
\newcommand{\End}[2]{\mathrm{End}_{#2}#1\,}
\newcommand{\aut}[2]{\mathrm{Aut}_{#2}#1\,}
\newcommand{\CAT}{\mathbf{Cat}}
\newcommand{\SET}{\mathbf{Set}}
\newcommand{\TOP}{\mathbf{Top}}
\newcommand{\GRP}{\mathbf{Grp}}
\newcommand{\RING}{\mathbf{Ring}}
\newcommand{\MOD}[1][R]{#1\text{-}\mathbf{Mod}}
\newcommand{\VEC}[1][K]{#1\text{-}\mathbf{Vec}}
\newcommand{\ALG}[1][R]{#1\text{-}\mathbf{Alg}}
\newcommand{\PSH}[1]{\mathbf{PSh}\brkt{#1}}
\newcommand{\map}[4]{#1 \yrightarrow[#4][4pt]{#3}[-1pt] #2}
\newcommand{\op}{^{op}}
\newcommand{\darrow}{\downarrow}

% % Algebra
\newcommand{\iso}{\cong}
\newcommand{\nsub}{\trianglelefteq}
\newcommand{\id}[1]{\mathrm{id}_{#1}}
\newcommand{\inv}{^{-1}}

% % Analysis
\newcommand{\abs}[1]{\left\vert #1 \right\vert}
\newcommand{\norm}[1]{\left\Vert #1 \right\Vert}
\renewcommand{\bar}[1]{\overline{#1}}
\newcommand{\<}{\langle}
\renewcommand{\>}{\rangle}
\renewcommand{\hat}[1]{\widehat{#1}}
\renewcommand{\check}[1]{\widecheck{#1}}

% % Galois
\newcommand{\Gal}[2]{\mathrm{Gal}_{#1}(#2)}
\DeclareMathOperator{\Orb}{Orb}
\DeclareMathOperator{\Stab}{Stab}
\newcommand{\emb}[3]{\mathrm{Emb}_{#1}(#2, #3)}
\newcommand{\Char}[1]{\mathrm{Char}#1}

% Arrows with text above and below with adjustable displacement
% (Stolen from Stackexchange)
\newcommandx{\yaHelper}[2][1=\empty]{
\ifthenelse{\equal{#1}{\empty}}
  % no offset
  { \ensuremath{ \scriptstyle{ #2 } } } 
  % with offset
  { \raisebox{ #1 }[0pt][0pt]{ \ensuremath{ \scriptstyle{ #2 } } } }  
}

\newcommandx{\yrightarrow}[4][1=\empty, 2=\empty, 4=\empty, usedefault=@]{
  \ifthenelse{\equal{#2}{\empty}}
  % there's no text below
  { \xrightarrow{ \protect{ \yaHelper[ #4 ]{ #3 } } } } 
  % there's text below
  {
    \xrightarrow[ \protect{ \yaHelper[ #2 ]{ #1 } } ]
    { \protect{ \yaHelper[ #4 ]{ #3 } } } 
  } 
}

%% code from mathabx.sty and mathabx.dcl to get some symbols from mathabx
\DeclareFontFamily{U}{mathx}{\hyphenchar\font45}
\DeclareFontShape{U}{mathx}{m}{n}{
      <5> <6> <7> <8> <9> <10>
      <10.95> <12> <14.4> <17.28> <20.74> <24.88>
      mathx10
      }{}
\DeclareSymbolFont{mathx}{U}{mathx}{m}{n}
\DeclareFontSubstitution{U}{mathx}{m}{n}
\DeclareMathAccent{\widecheck}{0}{mathx}{"71}

\MakePerPage{footnote}

% xcolor
\definecolor{lightgrey}{gray}{0.90}
\definecolor{slightgrey}{gray}{0.95}

% hyperref
\hypersetup{
      colorlinks = true,
      linkcolor = {blue},
      citecolor = {blue}
}

% Boxes
\mdfdefinestyle{Definitions}{
    leftmargin=0cm,
    rightmargin=0cm,
    linecolor=gray!70,
    topline=false,
    bottomline=false,
    rightline=false,
    backgroundcolor=gray!4,
    footnoteinside=true}

% thmtool 

% % custom theoremstyles
\declaretheoremstyle[
spaceabove=10pt, spacebelow=10pt,
notebraces = {- }{},
headpunct = {.\vspace{1mm}\newline}
]{mydfn}

\declaretheoremstyle[
spaceabove=10pt, spacebelow=10pt,
notebraces = {- }{},
bodyfont = \itshape, % Italics body font
headpunct = {.\vspace{1mm}\newline}
]{mythm}

% % Theorems
\declaretheorem[
  name = Theorem,
  style = mythm, 
  refname = {theorem,theorems},
  Refname = {Theorem,Theorems},
  numbered = no,
  % shaded = {rulecolor = lightgrey, rulewidth = 1.5mm, 
  %   bgcolor = white, textwidth = 46em}
]{thm}
\declaretheorem[
  name = Lemma,
  style = mythm, 
  refname = {lemma,lemmas},
  numbered = no]{lem}
\declaretheorem[
  name = Proposition,
  style = mythm, 
  refname = {proposition,propositions},
  numbered = no]{prop}
\declaretheorem[
  name = Corollary,
  style = mythm, 
  refname = {corollary,corollaries},
  numbered = no]{cor}
\declaretheorem[
  name = Remark, 
  style = remark, 
  numbered = no
]{rmk}
\declaretheorem[
  style = mydfn, 
  name = Definition, 
  numbered = no, 
  shaded = {bgcolor = slightgrey, margin = 2mm, textwidth = 46em}
]{dfn}
\declaretheorem[
  name = Example, 
  style = remark, 
  numbered = no
]{eg}

% tikzcd
% % Substituting symbols for arrows in tikz comm-diagrams.
\tikzset{
  symbol/.style={
    draw=none,
    every to/.append style={
      edge node={node [sloped, allow upside down, auto=false]{$#1$}}}
  }
}

\renewcommand{\listtheoremname}{List of Definitions and Theorems}

\begin{document}
\title{Category Theory Intuitively}
\author{Ken Lee}
\date{Date}
\maketitle

\tableofcontents

Notations : 
\begin{enumerate}
  \item For a collection of sets $U_i$ indexed by a set $I$, 
  $\Pi i \in I, U_i$ denotes $\prod_{i \in I} U_i$.
\end{enumerate}

\section{Categories}
\begin{dfn}[Categories]\link{cat}
  
  A \emph{category $\CC$} is defined by the following data : 
  \begin{enumerate}
    \item A set of \emph{objects}, $\obj{\CC}$. 
    \item For every $U, V \in \obj{\CC}$, 
    a set of \emph{$\CC$-morphisms} from $U$ to $V$,
    denoted $\mor{U}{V}{\CC}$.
    We denote $f : \map{U}{V}{\CC}{}$ for $f \in \mor{U}{V}{\CC}$. 
    \item For every $U, V, W \in \obj{\CC}$, 
    $f : \map{U}{V}{\CC}{}$ and $g : \map{V}{W}{\CC}{}$, 
    a $\CC$-morphism called the \emph{composition of $f$ with $g$},
    denoted $g \circ f : \map{U}{W}{\CC}{}$.
    \item Associativity of $\circ$.
    \item For every $U \in \obj{\CC}$, 
    an \emph{identity morphism} $\id{U} : \map{U}{U}{\CC}{}$.
    \item For all $U, V, W \in \obj{\CC}$, 
    $f : \map{U}{V}{\CC}{}$ and $g : \map{W}{U}{\CC}{}$, 
    we have $f \circ \id{U} = f$ and $\id{U} \circ g = g$. 
  \end{enumerate}
\end{dfn}

\begin{rmk}
Morphisms in a category do \emph{not} have to be functions.
See the example of preordered sets as categories at
\linkto{eg:cat_ord}{end of this section}.
\end{rmk}

\begin{dfn}[Isomorphisms]\link{iso}
  
  Let $\CC$ be a category, $U, V \in \obj{\CC}$, $f : \map{U}{V}{\CC}{}$.
  Then $f$ is called an \emph{isomorphism} when 
  there exists $g : \map{V}{U}{\CC}{}$ such that 
  $g \circ f = \id{U}$ and $f \circ g = \id{V}$.
  In this case, we denote $f : \map{U}{V}{\CC}{\sim}$.
  When there exists an isomorphism from $U$ to $V$, 
  we say they are \emph{isomorphic} and write $U \iso V$. 
\end{dfn}

\begin{dfn}[Subcategories]\link{subcat}
  
  Let $\CC,\DD$ be categories. 
  Then $\DD$ is called a \emph{subcategory of $\CC$} when 
  $\obj{\DD} \subs \obj{\CC}$ and 
  for all $U,V \in \obj{\DD}$, $\mor{U}{V}{\DD} \subs \mor{U}{V}{\CC}$.
\end{dfn}

\begin{eg}[Standard Categories]\link{eg:cat}~
  \begin{enumerate}
    \item $\SET$ denotes the \emph{category of sets}, where 
    $\obj{\SET}$ contains sets and for $U, V \in \obj{\SET}$, 
    $\SET(U,V)$ is the set of maps from $U$ to $V$. 
    \item $\TOP$ denotes the \emph{category of topological spaces}, where
    $\obj{\TOP}$ contains topological spaces and for $U, V \in \obj{\TOP}$, 
    $\TOP(U,V)$ is the set of continuous maps from $U$ to $V$. 
    $\TOP$ is a subcategory of $\SET$. 
    \item The \emph{category of groups} $\GRP$ has 
    $\obj{\GRP}$ containing groups and 
    $\GRP(U,V)$ containing group homomorphisms from $U$ to $V$. 
    $\GRP$ is a subcategory of $\SET$.  
    \item The \emph{category of abelian groups} $\AB$ has 
    $\obj{\AB}$ containing abelian groups and 
    $\AB(U,V)$ containing group homomorphisms from $U$ to $V$. 
    $\AB$ is a subcategory of $\GRP$. 
    \item The \emph{category of rings} $\RING$ has 
    $\obj{\RING}$ containing rings and 
    $\RING(U,V)$ containing ring homomorphisms from $U$ to $V$.
    $\RING$ is a subcategory of $\SET$. 
    \item The \emph{category of commutative rings} $\CRING$ has 
    $\obj{\CRING}$ containing commutative rings and 
    $\CRING(U,V)$ containing ring homomorphisms from $U$ to $V$.
    $\CRING$ is a subcategory of $\RING$. 
    \item Let $R$ be a ring. 
    Then the \emph{category of left $R$-modules} $R\MOD$ has 
    $\obj{R\MOD}$ containing left $R$-modules and 
    $R\MOD(U,V)$ contains $R$-linear maps from $U$ to $V$.
    This is a subcategory of $\AB$. 
    \item Let $R$ be a commutative ring. 
    Then the \emph{category of $R$-algebras} $R\ALG$ has 
    $\obj{R\ALG}$ containing pairs $(S,\si)$ where 
    $\si : \map{R}{S}{\RING}{}$.
    $R\ALG((U,u),(V,v))$ contains $f : \map{U}{V}{\RING}{}$ such that 
    $f \circ u = v$. 
  \end{enumerate}
\end{eg}

\begin{eg}[Preordered Sets as Categories]\link{eg:cat_ord}
  
  Let $I$ be a set, $\leq$ a relation on $I$. 
  Then $(I,\leq)$ is called a \emph{preordered set} when 
  $\leq$ satisfies all of the following : 
  \begin{enumerate}
    \item (Reflexivity) For all $i \in I$, $i \leq i$. 
    \item (Transitivity) For all $i, j, k \in I$, 
    $i \leq j$ and $j \leq k$ implies $i \leq k$. 
  \end{enumerate}
  If $(I,\leq)$ is a preordered set where $\leq$ is clear, 
  we abbreviate to $I$.

  Let $I$ be a preordered set. 
  Then we can turn $I$ into a category as follows : 
  \begin{enumerate}
    \item $\obj{I}$ is $I$. 
    \item For $i, j \in \obj{I}$, $I(i,j)$ is 
    singleton when $i \leq j$ and empty otherwise.  
  \end{enumerate}

  Things get meta. 
  We can form the \emph{category of preordered sets} $\ORD$ where 
  $\obj{\ORD}$ contains preoredered sets and 
  $\ORD(I,J)$ contains $f : \map{I}{J}{\SET}{}$ such that 
  for all $i, j \in I$, $i \leq j$ implies $f(i) \leq f(j)$.
\end{eg}

\begin{eg}[Category of Partially Ordered Sets]\link{eg:cat_poset}
  
  Let $I \in \obj{\ORD}$. 
  Then $I$ is called a \emph{partially ordered set} when 
  $\leq$ is \emph{antisymmetric}, i.e.
  for all $i, j \in I$, $i \leq j$ and $j \leq i$ implies $i = j$. 
  We thus have the \emph{category of partially ordered sets} $\POSET$ where 
  $\obj{\POSET}$ contains partially ordered sets and 
  $\POSET(I,J) = \ORD(I,J)$. 
  We see that $\POSET$ is a subcategory of $\ORD$.
\end{eg}

\begin{eg}[Partially Ordered Sets]\link{eg:poset}~
  \begin{enumerate}
    \item Let $X$ be a set. 
    Then its powerset $(2^X,\subs) \in \obj{\POSET}$.
    \item Let $X$ be a topological space. 
    Then the set of its opens $(\tau_X,\subs) \in \obj{\POSET}$.
    \item Let $G$ be a group. 
    Then the set of its subgroups $(\SUB\GRP(G),\subs)\in\obj{\POSET}$.
    \item Let $R$ be a ring and $M$ be a left $R$-module.
    Then the set of left $R$-submodules of $M$, $R\SUB\MOD(M)$,
    is in $\obj{\POSET}$.
    \item Let $R$ be a commutative ring and $(S,\si)$ an $R$-algebra. 
    Then the set of all $R$-subalgebras of $S$, $R\SUB\ALG(S)$, 
    is in $\obj{\POSET}$. 
    \item Let $X$ be a set and $\fil X$ the set of all filters on $X$.
    Then $(\fil X,\subs) \in \obj{\POSET}$. 
    \item Consider the relation on $\N$ that is $a \dvd b$. 
    This is a partial order on $\N$. 
  \end{enumerate}
\end{eg}

\begin{eg}[A Group as a Category]\link{eg:cat_group}
  
  The data of a group $G$ is equivalent to 
  a category $G$ where there is only one object $\bullet$ and 
  all morphisms are isomorphisms. 

  A direct generalization is a \emph{groupoid} : 
  a category where every morphism is an isomorphism. 
\end{eg}

\begin{eg}[Vector Spaces together with a Basis]\link{eg:vec_basis}
  Let $K$ be a field. 
  Define a category $\CC$ as follows : 
  \begin{itemize}
    \item objects are pairs $(V,B)$ where $V$ is a $K$-vector space and 
    $B$ is a basis of $V$. 
    \item For $(V,B_V), (W,B_W)$ objects in $\CC$, 
    define $\CC((V,B_V),(W,B_W))$ as the set of $K$-linear maps 
    from $V$ to $W$ such that maps $B_V$ into $B_W$.
    \item For every $K$-vector space with a basis $(V,B_V)$, 
    $\id{(V,B_V)}$ is defined to be the identity map of $V$. 
    \item Composition of the underlying $K$-linear maps of morphisms yields
    another morphism in this category. 
  \end{itemize}
  This category has \linkto{eg:functors}{nice connections} to change of basis. 
\end{eg}
\section{Functors}
\begin{dfn}[Functors]\link{functors}
  
  Let $\CC, \DD$ be categories. 
  Then a \emph{functor $F$ from $\CC$ to $\DD$} is defined by 
  the following data :
  \begin{enumerate}
    \item A map of objects $\obj{\CC} \to \obj{\DD}$,
    which we will denote by the same name $F$. 
    \item A map of morphisms for all $U, V \in \obj{\CC}$, 
    $\mor{U}{V}{\CC} \to \mor{F(U)}{F(V)}{\DD}$,
    which we will also denote by the same name $F$. 
    \item (Compositions are Preserved)
    For all $f : \map{U}{V}{\CC}{}$ and $g : \map{V}{W}{\CC}{}$, 
    $F(g \circ f) = F(g) \circ F(f)$. 
    \item (Identity Morphisms are Preserved)
    For all $U \in \obj{\CC}$, $F(\id{U}) = \id{F(U)}$.
  \end{enumerate}
\end{dfn}

\begin{dfn}[Category of Categories]\hypertarget{bigcat}{}
  
  We define the \emph{category of categories} $\CAT$, 
  \begin{enumerate}
    \item $\obj{\CAT}$ consists of categories. 
    \item For $\CC, \DD \in \obj{\CAT}$, 
    $\mor{\CC}{\DD}{\CAT}$ consists of functors from $\CC$ to $\DD$.  
    \item For $\CC \in \obj{\CAT}$, $\id{\CC}$ is the obvious thing.
  \end{enumerate}
\end{dfn}

\begin{dfn}[Faithful, Full, Fully Faithful]\link{fully_faithful}
  
  Let $F : \map{\CC}{\DD}{\CAT}{}$.
  Then $F$ is called 
  \begin{enumerate}
    \item \emph{faithful} when for all $U, V \in \obj{\CC}$,
    $F : \mor{U}{V}{\CC} \to \mor{F(U)}{F(V)}{\DD}$ is injective. 
    \item \emph{full} when for all $U, V \in \obj{\CC}$,
    $F : \mor{U}{V}{\CC} \to \mor{F(U)}{F(V)}{\DD}$ is surjective.
    \item \emph{fully faithful} when for all $U, V \in \obj{\CC}$,
    $F : \mor{U}{V}{\CC} \to \mor{F(U)}{F(V)}{\DD}$ is bijective.
  \end{enumerate}
\end{dfn}

\begin{prop}[Fully Faithful Functors are Injective]
  \link{full_faith_inj}
  
  Let $F : \map{\CC}{\DD}{\CAT}{}$ be fully faithful,
  $U, V \in \obj{\CC}$ such that $F(U) \iso F(V)$.
  Then $U \iso V$. 
\end{prop}
\begin{proof}
  Let $f_1 \in \DD(F(U),F(V))$ and $f_2 \in \DD(F(V),F(U))$ such that 
  $\id{F(U)} = f_2 \circ f_1$ and $\id{F(V)} = f_1 \circ f_2$. 
  Then $f_1, f_2$ corresponds respectively to $g_1, g_2 \in \CC(U,V), \CC(V,U)$
  through $F$. 
  We thus have 
  \[ 
    F(g_2 \circ g_1) = F(g_2) \circ F(g_1) = f_2 \circ f_1 = \id{F(U)}
    = F(\id{U})
  \]
  which by $F$ fully faithful gives $g_2 \circ g_1 = \id{U}$. 
  Similarly, $g_1 \circ g_2 = \id{V}$.
\end{proof}

\begin{dfn}[Natural Transformations]\link{natural}
  
  Let $F, G : \map{\CC}{\DD}{\CAT}{}$. 
  Then a \emph{natural transformation $\eta$ from $F$ to $G$} is defined by 
  the following data : 
  \begin{enumerate}
    \item For all $U \in \obj{\CC}$, $\eta_U : \map{F(U)}{G(U)}{\DD}{}$. 
    \item (Naturality) 
    For all $U, V \in \obj{\CC}$ and $f : \map{U}{V}{\CC}{}$, 
    we have the following commutative diagram. \begin{figure}[H]
      \centering
      \begin{tikzcd}
        F(U) \arrow[r,"\eta_U"] \arrow[d,"F(f)"{swap}] & G(U) \arrow[d,"G(f)"] \\
        F(V) \arrow[r,"\eta_V"{swap}] & G(V) 
      \end{tikzcd}
    \end{figure}
  \end{enumerate}
\end{dfn}

\begin{dfn}[Category of Functors]\link{cat_functor}
  
  Let $\CC, \DD \in \obj{\CAT}$. 
  Then the \emph{category of functors from $\CC$ to $\DD$},
  denoted $\DD^\CC$, is defined by 
  \begin{enumerate}
    \item $\obj{\DD^\CC} := \mor{\CC}{\DD}{\CAT}$. 
    \item For all $F, G \in \obj{\DD^\CC}$, 
    $\mor{F}{G}{\DD^\CC} := $
    the set of natural transformations from $F$ to $G$. 
    \item For all $F \in \obj{\DD^\CC}$, $\id{F}$ is the obvious thing.
    \item The obvious way to define composition of natural tranformations is ``component-wise''.
  \end{enumerate}
\end{dfn}

\begin{dfn}[Equivalence of Categories]\link{equiv}
  
  Let $\CC, \DD$ be categories, $F \in \CAT(\CC,\DD)$.
  Then $F$ is called an \emph{equivalence of categories} when 
  there exists $G \in \CAT(\DD,\CC)$ such that  
  $G \circ F \iso \id{\CC}$ and $F\circ G \iso \id{\DD}$. 
\end{dfn}

\begin{dfn}[Essentially Surjective]\link{surj}
  
  Let $\CC,\DD$ be categories and $F : \map{\CC}{\DD}{\CAT}{}$. 
  Then $F$ is called \emph{essentially surjective} when 
  for all $X \in \obj{\DD}$, there exists $U \in \obj{\CC}$ such that 
  $F(U) \iso X$. 
\end{dfn}

\begin{prop}[Characterisation of Equivalence of Categories]\link{char_equiv}
  
  Let $F : \map{\CC}{\DD}{\CAT}{}$. 
  Then $F$ is an equivalence of categories if and only if 
  $F$ is fully faithful and essentially surjective. 
\end{prop}
\section{Universal Morphisms}
\begin{dfn}[Comma Category]\hypertarget{comma}{}
  
  Let $G : \map{\CC}{\DD}{\CAT}{}$ and $X \in \obj{\DD}$.
  Then the \emph{comma category $X\darrow G$} is defined as follows. 
  \begin{enumerate}
    \item $\obj{X\darrow G}$ consists of pairs $(U,u)$
    where $U \in \obj{\CC}$ and $u : \map{X}{G(U)}{\DD}{}$.
    \item For $(U,u), (V,v) \in \obj{X\darrow G}$, 
    $\mor{(U,u)}{(V,v)}{X\darrow G}$ consists of $f : \map{U}{V}{\CC}{}$
    such that \begin{figure}[H]
      \centering
      \begin{tikzcd}
        X \ar[r,"u"] \ar[rd,"v"{swap}] & G(U) \ar[d,"G(f)"] \\
        & G(V)
      \end{tikzcd}
    \end{figure}
  \end{enumerate}

  Dually, let $F : \map{\DD}{\CC}{\CAT}{}$ and $U \in \obj{\CC}$.
  Then the \emph{comma category $F\darrow U$} is defined as follows. 
  \begin{enumerate}
    \item $\obj{F\darrow U}$ consists of pairs $(X,x)$
    where $X \in \obj{\DD}$ and $x : \map{F(X)}{U}{\CC}{}$.
    \item For $(X,x), (Y,y) \in \obj{F\darrow U}$, 
    $\mor{(X,x)}{(Y,y)}{X\darrow F}$ consists of $g : \map{X}{Y}{\DD}{}$
    such that \begin{figure}[H]
      \centering
      \begin{tikzcd}
        F(X) \ar[rd,"x"] \ar[d,"F(g)"{swap}]& \\
        F(Y) \ar[r,"y"{swap}] & U
      \end{tikzcd}
    \end{figure}
  \end{enumerate}
\end{dfn}

\begin{rmk}
  Here is a special case of the comma category worth noting. 
\end{rmk}

\begin{dfn}[Over Category]\link{over_cat}
  
  Let $\CC$ be a category and $U \in \obj{\CC}$.
  Then the \emph{over category $\CC\darrow U$} is defined as 
  $\id{\CC}\darrow U$. 

  Dually, the \emph{under category $U \darrow \CC$} is defined as 
  $U\darrow \id{\CC}$.
\end{dfn}

\begin{eg}[Over and Under Categories]\link{eg:over_cat}
  \begin{enumerate}
    \item Let $R\in\obj{\RING}$. Then $\ALG[R] = R\darrow\RING$.
    \item Let $X\in\obj{\TOP}$. Then we have the category of covering spaces 
    of $X$ with is the subcategory of $\TOP\darrow X$ where 
    objects are $(\tilde{X},p)$ with $p$ a covering map.
  \end{enumerate}
\end{eg}

\begin{dfn}[Universal Morphism]\hypertarget{universal}{}
  
  Let $G : \map{\CC}{\DD}{\CAT}{}$ and $X \in \obj{\DD}$.
  Then a \emph{universal morphism from $X$ to $G$} is the following data. 
  \begin{enumerate}
    \item An object $(F(X),\eta_X)$ of the comma category $X\darrow G$.
    \item (Universal Property) For all $(V,v) \in \obj{X\darrow G}$,
    there exists a unique morphism $\map{(F(X),\eta_X)}{(V,v)}{X\darrow G}{}$.
  \end{enumerate}

  Dually, let $F : \map{\DD}{\CC}{\CAT}{}$ and $U \in \obj{\CC}$.
  Then a \emph{universal morphism from $F$ to $U$} is the following data. 
  \begin{enumerate}
    \item An object $(G(U),\ep_U)$ of the comma category $F\darrow U$.
    \item (Universal Property)For all $(Y,y) \in \obj{F\darrow U}$,
    there exists a unique morphism $\map{(Y,y)}{(G(U),\ep_U)}{F\darrow U}{}$.
  \end{enumerate}
\end{dfn}

\begin{prop}[Unique up to Unique Isomorphism]\hypertarget{uniprop}{}
  
  Let $G : \map{\CC}{\DD}{\CAT}{}$, $X \in \obj{\DD}$, 
  $(U,u), (V,v) \in X\darrow G$ both universal morphisms from $X$ to $G$. 
  Then there exists a unique $f : \map{(U,u)}{(V,v)}{X\darrow G}{}$
  and $g : \map{(V,v)}{(U,u)}{X\darrow G}{}$ such that 
  $g \circ f = \id{(U,u)}$ and $f \circ g = \id{(V,v)}$.
  Thus, if a universal morphism exists, 
  we say it is \emph{unique up to unique isomorphism}.

  Dually, let $F : \map{\DD}{\CC}{\CAT}{}$, $U \in \obj{\CC}$,
  $(X,x), (Y,y) \in F\darrow U$ both universal morphisms from $F$ to $U$.
  Then there exists a unique $f : \map{(X,x)}{(Y,y)}{F\darrow U}{}$
  and $g : \map{(Y,y)}{(X,x)}{F\darrow u}{}$ such that 
  $g \circ f = \id{(X,x)}$ and $f \circ g = \id{(Y,y)}$.
\end{prop}
\begin{proof}(Shorter proof that does not go through Yoneda).

  By the universal property of $(U,u)$,
  There exists a unique $f : \map{(U,u)}{(V,v)}{X\darrow G}{}$.
  Similarly, there exists a unique $g : \map{(Y,y)}{(X,x)}{F\darrow u}{}$.
  But then $g \circ f : \map{(U,u)}{(U,u)}{X\darrow G}{}$.
  By applying the universal property of $(U,u)$ with itself,
  we see that $\id{(U,u)}$ is the unique $\map{(U,u)}{(U,u)}{X\darrow G}{}$.
  In particular, we have $g \circ f = \id{(U,u)}$.
  Similarly, we have $f \circ g = \id{(V,v)}$.
  Since $f$ and $g$ are the \emph{only} morphisms between $(U,u)$, $(V,v)$,
  they are \emph{the} unique isomorphism between $(U,u)$ and $(V,v)$.
\end{proof}

\begin{rmk}[``Canonically Isomorphic'']
  \hypertarget{canonical}{}

  It is common in category theory and maths at large to \emph{equate} 
  two objects that satisfy the same universal property, 
  since they are not only isomorphic, but also isomorphic in a unique way. 
  Some also call these \emph{canonically isomorphic}.
\end{rmk}

\begin{prop}[Isomorphic to Universal implies Universal]
  \link{iso_uni_implies_uni}
  
  Let $G : \map{\CC}{\DD}{\CAT}{}$, $X \in \obj{\DD}$, 
  $(U,u), (V,v) \in X\darrow G$ where 
  $(U,u) \iso[X\darrow G] (V,v)$ and 
  $(U,u)$ is a universal morphism from $X$ to $G$.
  Then $(V,v)$ is a universal morphism from $X$ to $G$.

  Dually, let $F : \map{\DD}{\CC}{\CAT}{}$, $U \in \obj{\CC}$,
  $(X,x), (Y,y) \in F\darrow U$ where $(X,x) \iso[F\darrow U] (Y,y)$ and 
  $(X,x)$ is a universal morphism from $F$ to $U$. 
  Then $(Y,y)$ is a universal morphism from $F$ to $U$. 
\end{prop}
\begin{proof}
  Let $f : \map{(U,u)}{(V,v)}{X\darrow G}{\sim}$.
  Let $(W,w)\in\obj{X\darrow G}$. 
  Then $f$ induces a bijection between $\mor{(U,u)}{(W,w)}{X\darrow G}$
  and $\mor{(V,v)}{(W,w)}{X\darrow G}$. 
  Since the former is singleton, so is the latter. 

  The dual has a similar argument. 
\end{proof}

% \begin{proof}
%  By \hyperlink{uni_iff_rep}{universal iff represents},
%  let $\al : \map{h^U}{\mor{X}{G(\star)}{\DD}}{\SET^\CC}{\sim}$
%  and $\be : \map{h^V}{\mor{X}{G(\star)}{\DD}}{\SET^\CC}{\sim}$ 
%  with $\al_{U}(\id{U}) = u$ and $\be_V(\id{V}) = v$.
%  Then $\be\inv \circ \al$ and $\al\inv \circ \be$ are 
%  natural isomorphisms between $h^U$ and $h^V$.
%  So by \hyperlink{yoneda}{Yoneda's lemma}, 
%  there exists unique $f : \map{U}{V}{\CC}{}$ and $g : \map{V}{U}{\CC}{}$
%  such that the following diagram commutes. 
%  \begin{figure}[H]
%    \centering
%    \begin{tikzcd}[sep = large]
%      \mor{X}{G(\star)}{\DD} 
%      & h^V \ar[l,"\be"{swap},"\sim"] 
%        \ar["h^f",ld,rightharpoonup,xshift=1mm,yshift=-1mm] \\
%      h^U \ar[u,"\al","\sim"{swap}] 
%        \ar["h^g",ru,rightharpoonup] 
%      &  
%    \end{tikzcd}
%  \end{figure}
%  Since $h^{\id{U}} = h^f \circ h^g = h^{g \circ f}$,
%  applying \hyperlink{yoneda}{Yoneda's lemma} again, 
%  we obtain $g \circ f = \id{U}$, and thus $g \circ f = \id{(U,u)}$.
%  Similarly, we have $f \circ g = \id{(V,v)}$.
%  We claim that for any 
%  $f : \map{U}{V}{\CC}{}$ and $g : \map{V}{U}{\CC}{}$,
%  they make the above diagram commute if and only if 
%  $f : \map{(U,u)}{(V,v)}{X\darrow G}{}$ and 
%  $g : \map{(V,v)}{(U,u)}{X\darrow G}{}$.
%  This shows that $f, g$ are as desired. 
%
%  Since the arguments are analogous, 
%  we just prove it for $f$.
%  If the above diagram commutes for $f$,
%  then $v = \be_V(\id{V}) = \al_V \circ \brkt{h^f}_V (\id{V})
%  = \al_V(f) = G(f) \circ u$ as desired. 
%  Now suppose $v = G(f) \circ u$.
%  To show $\al \circ h^f = \be$, 
%  let $W \in \obj{\CC}$ and $g \in h^V(W)$.
%  Consider the following diagram. 
%  \begin{figure}[H]
%    \centering
%    \begin{tikzcd}[sep = huge]
%      \mor{X}{G(W)}{\DD}
%      & h^V(W) \ar["\be_W"{swap},l] \ar["\brkt{h^f}_W"{near end},ld]
%      & \\
%      h^U(W) \ar["\al_W",u]
%      & \mor{X}{G(V)}{\DD} \ar["\mor{X}{G(g)}{\DD}"{swap,near start},lu] 
%      & h^V(V) \ar["h^V(g)"{swap},lu] \ar["\be_V"{swap},l] 
%        \ar["\brkt{h^f}_V",ld] \\
%      & h^U(V) \ar["h^U(g)",lu] \ar["\al_V",u]
%      & 
%    \end{tikzcd}
%  \end{figure}
%  Everything except the triangles commute. 
%  From this, we compute \begin{align*}
%    \be_W(g) 
%    &= \be_W \circ h^V(g)\brkt{\id{V}}
%      = \mor{X}{G(g)}{\DD} \circ \be_V \brkt{\id{V}} 
%      = \mor{X}{G(g)}{\DD}\brkt{v} \\
%    &= \mor{X}{G(g)}{\DD} \brkt{G(f) \circ u}
%      = \mor{X}{G(g)}{\DD} \circ \al_V(f) 
%      = \mor{X}{G(g)}{\DD} \circ \al_V \circ \brkt{h^f}_V \brkt{\id{V}} \\
%    &= \al_W \circ h^U(g) \circ \brkt{h^f}_V \brkt{\id{V}}
%      = \al_W \circ \brkt{h^f}_W \circ h^F(g) \brkt{\id{V}} 
%      = \al_W \circ \brkt{h^f}_W \brkt{g}
%  \end{align*}
%  Thus $\be_W = \al_W \circ \brkt{h^f}_W$,
%  and hence $\be = \al \circ \brkt{h^f}$.
%  This concludes the proof.
% \end{proof}
\section{Yoneda's Lemma}
\begin{dfn}[Dual Categories]\link{dual}{}
  
  Let $\CC \in \obj{\CAT}$. 
  Then the \emph{dual category of $\CC$}, denoted $\CC\op$,
  is defined by : 
  \begin{enumerate}
    \item $\obj{\CC\op} := \obj{\CC}$. 
    \item For all $U, V \in \obj{\CC\op}$,
    $\mor{U}{V}{\CC\op} := \mor{V}{U}{\CC}$.
  \end{enumerate}
\end{dfn}

\begin{dfn}[Contravariant Functors]\link{contravar}{}
  
  Let $\CC, \DD \in \obj{\CAT}$. 
  Then a \emph{contravariant functor from $\CC$ to $\DD$} is 
  just a functor $\map{\CC\op}{\DD}{\CAT}{}$. 
  Functors $\map{\CC}{\DD}{\CAT}{}$ are henceforth called 
  \emph{covariant functors from $\CC$ to $\DD$}. 
\end{dfn}

% \begin{dfn}[Category of Presheaves on a Category]\link{presheaf}
%   
%   Let $\CC \in \obj{\CAT}$. 
%   Then the \emph{category of presheaves on $\CC$} is defined as 
%   the category of contravariant functors from $\CC$ to $\SET$. 
%   It is denoted $\PSH{\CC}$. 
% \end{dfn}

\begin{dfn}[Morphism Functor]\link{mor_funk}{}
  
  Let $\CC$ be a category and $U \in \obj{\CC}$. 
  Then $h_U : \map{\CC\op}{\SET}{\CAT}{}$ is defined as : 
  \begin{enumerate}
    \item For all $V \in \obj{\CC\op}$, 
    $h_U(V) := \mor{V}{U}{\CC}$. 
    \item For all $V, W \in \obj{\CC\op}$ and $f : \map{V}{W}{\CC\op}{}$, 
    $h_U(f) : \map{h_U(V)}{h_U(W)}{}{}, g \mapsto g \circ f$. 
  \end{enumerate}
  Similarly, $h^U : \map{\CC}{\SET}{\CAT}{}$ is defined as : 
  \begin{enumerate}
    \item For all $V \in \obj{\CC}$, 
    $h^U(V) := \mor{U}{V}{\CC}$. 
    \item For all $V, W \in \obj{\CC}$ and $f : \map{V}{W}{\CC}{}$, 
    $h^U(f) : \map{h^U(V)}{h^U(W)}{}{}, 
    g \mapsto f \circ g$. 
  \end{enumerate}
\end{dfn}

\begin{prop}[Morphism Functor is Functorial]\link{mor_funk_funk}{}
  
  Let $\CC$ be a category. 
  Then $h_{\star} : \map{\CC}{\SET^{\CC\op}}{\CAT}{}$. 
  Similarly, $h^\star : \map{\CC\op}{\SET^\CC}{\CAT}{}$. 
\end{prop}

\begin{rmk}[Functor of Points]\link{funk_pts}{}
  
  Because of its relevance in algebraic geometry, 
  $h_U$ is called the \emph{functor of points of $U$}. 
\end{rmk}

\begin{prop}[Yoneda's Lemma]\link{yoneda}{}
  
  Let $\CC$ be a category. 
  Then $h_{\star} : \map{\CC}{\SET^{\CC\op}}{\CAT}{}$ is fully faithful.
  Since \hyperlink{full_faith_inj}{fully faithful functors are injective}, 
  $h_\star$ is called the \emph{Yoneda embedding}. 

  More generally, for any $U \in \obj{\CC}$ and $F \in \obj{\SET^{\CC\op}}$,
  $\mor{h_U}{F}{\SET^{\CC\op}}$ bijects with $F(U)$ via 
  $s \mapsto s_U(\id{U})$ and this bijection is natural in 
  both $U$ and $F$.

  Dually, $h^\star : \map{\CC\op}{\SET^{\CC}}{\CAT}{}$ is fully faithful
  and more generally, 
  for any $U \in \obj{\CC\op}$ and $F \in \obj{\SET^\CC}$,
  $\mor{h^U}{F}{\SET^\CC}$ naturally bijects with $F(U)$ 
  via $s \mapsto s_U(\id{U})$.

\end{prop}
%\begin{proof}
%
%  We first prove the general statement.
%  Let $U \in \obj{\CC}$ and $F \in \obj{\SET^{\CC\op}}$.
%  To show injectivity, let $s, t \in \mor{h_U}{F}{\SET^{\CC\op}}$ and
%  assume $s_U(\id{U}) = t_U(\id{U})$. 
%  To show $s = t$, let $W \in \obj{\CC}$, $f \in h_U(W)$ and
%  consider the following commutative diagram, 
%  \begin{figure}[H]
%    \centering
%    \begin{tikzcd}
%      h_U(U) \arrow[rrr,"s_U"] \arrow[ddd,"h_U(f)"{swap}] 
%      & 
%      & 
%      & F(U) \arrow[ddd,"F(f)"] 
%      \\
%      & \id{U} \ar[r,mapsto] \ar[d,mapsto]& s_U(\id{U}) \ar[d,mapsto]& \\
%      & f \ar[r,mapsto] & s_W(f) = F(f)(s_U(\id{U})) & \\
%      h_U(W) \arrow[rrr,"s_W"{swap}] 
%      & 
%      & 
%      & F(W) 
%    \end{tikzcd}
%  \end{figure}
%  By considering an analogous for $t$,
%  we get $s_W(f) = t_W(f)$.
%  So $s_W = t_W$, and hence $s = t$. 
%  To show surjectivity, let $x \in F(U)$. 
%  Define $s \in \mor{h_U}{F}{\SET^{\CC\op}}$ by 
%  for all $V \in \obj{\CC}$, \[
%    s_V : f \in h_U(V) \mapsto F(f)(x) \in F(V)
%  \]
%  Then by the above diagram, $s_U(\id{U}) = x$. 
%  This proves the desired bijection.
%
%  For naturality in the first component,
%  let $f : \map{U}{V}{\CC}{}$.
%  Then we have the following commutative diagram. 
%  \begin{figure}[H]
%    \centering
%    \begin{tikzcd}
%      \mor{h_U}{F}{\SET^{\CC\op}} \ar[rrr]
%      & & & F(U) \\
%      & s\circ h^f \ar[r,mapsto]
%      & (s\circ h^f)_U (\id{U}) 
%        = s_V(f)
%        = F(f) \circ s_V (\id{V}) & \\
%      & s \ar[u,mapsto] \ar[r,mapsto] & s_V(\id{V}) \ar[u,mapsto] & \\
%      \mor{h_V}{F}{\SET^{\CC\op} } \ar[uuu,"(\star\circ h^f)"] \ar[rrr,""]
%      & & & F(V) \ar[uuu,"F(f)"{swap}]
%    \end{tikzcd}
%  \end{figure}
%  For naturality in the second component, 
%  let $\al : \map{F}{G}{\SET^{\CC\op}}{}$.
%  Then we have the following commutative diagram.
%  \begin{figure}[H]
%    \centering
%    \begin{tikzcd}
%      \mor{h_U}{F}{\SET^{\CC\op}} \ar[rrr]
%        \ar[ddd,"(\al\circ \star)"{swap}]
%      & & & F(U) \ar[ddd,"\al_U"] \\
%      & s \ar[r,mapsto] \ar[d,mapsto]
%      & s_U (\id{U}) \ar[d,mapsto] & \\
%      & \al\circ s \ar[r,mapsto] 
%      & (\al\circ s)_U (\id{U}) = \al_U \circ s_U(\id{U})
%      & \\
%      \mor{h_U}{G}{\SET^{\CC\op} } \ar[rrr,""]
%      & & & G(U)
%    \end{tikzcd}
%  \end{figure}
%  We thus have the desired result. 
%
%  To show $h_\star$ is fully faithful, 
%  let $U, V \in \obj{\CC}$ and apply the above to $F := h_V$.
%\end{proof}
\begin{proof}
  We first prove the general statement. 
  Let $U \in \obj{\CC}, F \in \obj{\SET^{\CC\op}}$.
  Given an element $s \in F(U)$,
  we are tasked with constructing a natural transformation $h_U \to F$.
  For $V \in \CC$ we want to map elements 
  $f \in h_U(V)$ to some element of $F(V)$.
  Well, $f$ is a morphism from $V$ to $U$,
  so $F(f)$ is a morphism from $F(U)$ to $F(V)$,
  and we are given an element $s \in F(U)$.
  So define $\al^s_V : h_U(V) \to F(V) := f \mapsto F(f)(s)$.
  For the collection of $\al^s_V$ to form a natural transformation,
  we need naturality. 
  So given $f \in \CC(V,W)$,
  we need the following diagram to commute : 
  \begin{cd}
    h_U(W) \ar[d,"h_U(f)"] \ar[r,"\al^s_W"] & F(W) \ar[d,"F(f)"]\\
    h_U(V) \ar[r,"\al^s_V"] & F(V)
  \end{cd}
  For $g \in h_U(W)$, then we have as desired \[
    \al^s_V \circ h_U(f) (g) = \al^s_V(g \circ f)
    = F(g \circ f) (x) = F(f) \circ F(g) (x)
    = F(f) \circ \al^s_W (g)
  \]
  So $\al^s : h_U \to F$ is a natural transformation.
  
  Note that we can recover $s$ from $\al^s$ by $\al^s_U(\id{U}) = s$.
  This motivates us to define the inverse map by 
  $\al \in \SET^{\CC\op}(h_U,F) \mapsto \al_U(\id{U})$.
  To show these two maps are indeed inverses, 
  first consider the following diagram where 
  $\al : h_U \to F$ is a natural tranformation,
  $W \in \obj{\CC}$ and $f \in h_U(W)$ : 
  \begin{figure}[H]
    \centering
    \begin{tikzcd}
      h_U(U) \arrow[rrr,"a\l_U"] \arrow[ddd,"h_U(f)"{swap}] 
      & 
      & 
      & F(U) \arrow[ddd,"F(f)"] 
      \\
      & \id{U} \ar[r,mapsto] \ar[d,mapsto]& \al_U(\id{U}) \ar[d,mapsto]& \\
      & f \ar[r,mapsto] & \al_W(f) = F(f)(\al_U(\id{U})) & \\
      h_U(W) \arrow[rrr,"\al_W"{swap}] 
      & 
      & 
      & F(W) 
    \end{tikzcd}
  \end{figure}
  The above diagram commutes by naturality of $\al$.
  What it shows is that $\al_W$ is completely determined by $\al_U(\id{U})$,
  and hence $\al$ is completely determined by $\al_U(\id{U})$.
  This proves one side of the inverse situation. 
  The other side is clear.
  Thus we have a bijection between $\SET^{\CC\op}(h_U,F) \iso F(U)$.

  At this point, we can already get $h_\star$ fully faithful by
  applying the above bijection to $F = h_\star$ itself and noting
  the bijection turns $f \in h_V(U)$ into $h_f$. 

  For naturality in the first component,
  let $f : \map{U}{V}{\CC}{}$.
  Then we have the following commutative diagram. 
  \begin{figure}[H]
    \centering
    \begin{tikzcd}
      \mor{h_U}{F}{\SET^{\CC\op}} \ar[rrr]
      & & & F(U) \\
      & \al\circ h^f \ar[r,mapsto]
      & ( \al\circ h^f)_U (\id{U}) 
        = \al_V(f)
        = F(f) \circ \al_V (\id{V}) & \\
      & \al \ar[u,mapsto] \ar[r,mapsto] & \al_V(\id{V}) \ar[u,mapsto] & \\
      \mor{h_V}{F}{\SET^{\CC\op} } \ar[uuu,"(\star\circ h^f)"] \ar[rrr,""]
      & & & F(V) \ar[uuu,"F(f)"{swap}]
    \end{tikzcd}
  \end{figure}
  For naturality in the second component, 
  let $\phi : \map{F}{G}{\SET^{\CC\op}}{}$.
  Then we have the following commutative diagram.
  \begin{figure}[H]
    \centering
    \begin{tikzcd}
      \mor{h_U}{F}{\SET^{\CC\op}} \ar[rrr]
        \ar[ddd,"(\phi\circ \star)"{swap}]
      & & & F(U) \ar[ddd,"\phi_U"] \\
      & \al \ar[r,mapsto] \ar[d,mapsto]
      & \al_U (\id{U}) \ar[d,mapsto] & \\
      & \phi\circ \al \ar[r,mapsto] 
      & (\phi\circ \al)_U (\id{U}) = \phi_U \circ \al_U(\id{U})
      & \\
      \mor{h_U}{G}{\SET^{\CC\op} } \ar[rrr,""]
      & & & G(U)
    \end{tikzcd}
  \end{figure}
  We thus have the desired result. 
\end{proof}

\begin{dfn}[Representable Functors]
  \link{rep}{}
  
  Let $G : \map{\CC}{\SET}{\CAT}{}$ be a covariant functor. 
  Then a \emph{representation of $G$} is a $(U,u) \in h^\star\darrow G$
  where $u : \map{h^\star}{G}{\SET^\CC}{\sim}$.

  Dually, let $F : \map{\CC\op}{\SET}{}{}$ be a contravariant functor. 
  Then a \emph{representation of $F$} is a $(U,u) \in h_\star\darrow F$
  where $u : \map{h_\star}{F}{\SET^{\CC\op}}{\sim}$.

  A functor (covariant or contravariant) that has a representation is called 
  \emph{representable}.
\end{dfn}

\begin{rmk}
  If a functor has a representation,
  Yoneda's lemma implies it is canonical.
  This is the \hyperlink{canonical_rep}{next result}. 

  Before this, we first relate universal morphisms to representable functors.
  This is important as it leads to the notion of \emph{adjunction}. 
\end{rmk}

\begin{prop}[Universal iff Represents]
  \link{uni_iff_rep}{}
  
  Let $R : \map{\CC}{\DD}{\CAT}{}$, $X \in \obj{\DD}$,
  $(L(X),\eta_X) \in \obj{X\darrow R}$. 
  Then the following are equivalent : 
  \begin{enumerate}
    \item $(L(X),\eta_X)$ is a universal morphism from $X$ to $R$.
    \item $L(X)$ represents the covariant functor $\mor{X}{R(\star)}{\DD}$
    and $\id{L(X)}$ corresponds to $\eta_X$. 
  \end{enumerate}

  Dually, let $L : \map{\DD}{\CC}{\CAT}{}$, $U \in \obj{\CC}$,
  $(R(U),\ep_U) \in \obj{L\darrow U}$. 
  Then the following are equivalent : 
  \begin{enumerate}
    \item $(R(U),\ep_U)$ is a universal morphism from $L$ to $U$.
    \item $R(U)$ represents the contravariant functor $\mor{L(\star)}{U}{\CC}$
    and $\id{R(U)}$ corresponds to $\ep_U$.
  \end{enumerate}
\end{prop}
\begin{proof}
  (Universal implies Represents) 
  Let $(L(X),\eta_X)$ be a universal morphism from $X$ to $R$. 
  Define the following natural transformation,
  \begin{align*}
    &\map{h^{L(X)}}{\mor{X}{R(\star)}{\DD}}{\SET^\CC}{} := \\
    &W \in \obj{\CC} \mapsto \sqbrkt{
      f \in h^{L(X)}(W) \mapsto R(f) \circ \eta_X \in \mor{X}{R(W)}{\DD}
    }
  \end{align*}
  Then for every $W \in \obj{\CC}$, 
  this is an isomorphism between $h^{L(X)}(W)$ and $\mor{X}{R(W)}{\DD}$,
  and hence a natural isomorphism. 
  Indeed, $\id{L(X)}$ corresponds to $\eta_X$ under this natural isomorphism.

  (Represents implies Universal)
  Let $\al : \map{h^{L(X)}}{\mor{X}{R(\star)}{\DD}}{\SET^\CC}{}$
  be a natural isomorphism where at $L(X)$,
  $\al_{L(X)}(\id{L(X)}) = \eta_X$.
  Let $(V,v) \in \obj{X\darrow R}$.
  For any $f : \map{L(X)}{V}{\CC}{}$,
  consider the following commutative diagram. 
  \begin{figure}[H]
    \centering
    \begin{tikzcd}
      h^{L(X)}(L(X)) \arrow[rrr,"\al_{L(X)}"] \arrow[ddd,"h^{L(X)}(f)"{swap}] 
      & 
      & 
      & \mor{X}{RL(X)}{\DD} \arrow[ddd,"\mor{X}{R(f)}{\DD}"] \\
      & \id{L(X)} \ar[r,mapsto] \ar[d,mapsto]
      & \al_{L(X)}(\id{L(X)}) = \eta_X \ar[d,mapsto] 
      & \\
      & f \ar[r,mapsto] 
      & \al_V(f) = R(f) \circ \eta_X 
      & \\
      h^{L(X)}(V) \arrow[rrr,"\al_V"{swap}] 
      & 
      & 
      & \mor{X}{R(V)}{\DD}
    \end{tikzcd}
  \end{figure}
  Thus $f : \map{(L(X),\eta_X)}{(V,v)}{X\darrow R}{}$
  if and only if $\al_V(f) = v$.
  Then $\al_V\inv(v)$ is 
  the unique morphism $\map{(L(X),\eta_X)}{(V,v)}{X\darrow R}{}$.
  Since there exists a unique $\map{(L(X),\eta_X)}{(V,v)}{X\darrow R}{}$,
  $(L(X),\eta_X)$ is universal.

  The dual equivalence has an analogous proof.
\end{proof}

\begin{prop}[Canonical Representation]
  \link{canonical_rep}{}
  
  Let $G : \map{\CC}{\DD}{\CAT}{}$ and $(U,u) \in h^\star\darrow G$.
  Then the following are equivalent : 
  \begin{enumerate}
    \item $(U,u)$ is a representation of $G$.
    \item $(U,u)$ is a universal morphism from $h^\star$ to $G$.
  \end{enumerate}
  In particular, 
  representations of $G$ are canonically isomorphic. 

  Dually, let $F : \map{\CC\op}{\DD}{\CAT}{}$ and $(V,v) \in h_\star\darrow F$.
  Then the following are equivalent : 
  \begin{enumerate}
    \item $(V,v)$ is a representation of $F$.
    \item $(V,v)$ is a universal morphism from $h_\star$ to $F$.
  \end{enumerate}
  In particular, 
  representations of $F$ are canonically isomorphic. 
\end{prop}
\begin{proof}
  (Representation implies Universal)
  Let $(W,w) \in \obj{h^\star\darrow G}$.
  Then $u\inv \circ w : \map{h^W}{h^U}{\SET^\CC}{}$.
  By \hyperlink{yoneda}{Yoneda's lemma},
  there exists a unique $u(W,w) : \map{U}{W}{\CC}{}$ such that 
  $u\inv \circ w = h^{u(W,w)}$.
  Hence $u(W,w)$ is the unique morphism 
  $\map{(W,w)}{U,u}{h^\star\darrow G}{}$.

  (Universal implies Representation)
  By \hyperlink{uni_iff_rep}{universal iff represents}
  and \hyperlink{yoneda}{Yoneda's lemma},
  we have the following diagram. 
  \begin{figure}[H]
    \centering
    \begin{tikzcd}[sep = huge]
      \mor{h^\star}{G}{\SET^\CC} 
        \ar[r,"V \in \obj{\CC} \mapsto \sqbrkt{
          s \in \mor{h^V}{G}{\SET^\CC} \mapsto s_V(\id{V})
        }"{yshift = 3mm},"\sim"{swap}]
      & G \\
      h^U \ar[u,
        "V \in \obj{\CC} \mapsto \sqbrkt{
          f \in h^U(V) \mapsto u \circ h^f
        }",
        "\sim"{swap}
        ]
        \ar[ur,"u"]
      &
    \end{tikzcd}
  \end{figure}
  The claim is that the above commutes, and hence $u$ is an isomorphism.
  Let $V \in \obj{\CC}$ and $f \in h^U(V)$.
  Then \begin{align*}
    (h^f \circ u)_V(\id{V})
    = u_V \circ \brkt{h^f}_V \brkt{\id{V}}
    = u_V (f)
  \end{align*}
  So the above diagram commutes. 

  For the dual, the argument is similar. 
\end{proof}
\section{Adjoint Functors}
\begin{dfn}[Adjoint Functors]
  \link{adjoint}
  
  Let $R : \map{\CC}{\DD}{\CAT}{}$.
  Then $R$ is a \emph{right adjoint} when 
  there exists $L : \map{\obj{\DD}}{\obj{\CC}}{}{}$ and 
  $(\eta_X) \in \prod_{X \in \obj{\DD}} \mor{X}{RL(X)}{\DD}$ such that 
  for all $X\in\obj{\DD}$, 
  $(L(X),\eta(X))$ is a universal morphism from $X$ to $R$.
  In this case, $L$ is called the \emph{left adjoint of $R$}.
  \newline 
  
  Dually, let $L : \map{\DD}{\CC}{\CAT}{}$.
  Then $L$ is a \emph{left adjoint} when 
  there exists $R : \map{\obj{\CC}}{\obj{\CC}}{}{}$ and 
  $(\ep_U) \in \prod_{U \in \obj{\CC}} \mor{LR(U)}{U}{\CC}$ such that 
  for all $U\in\obj{\CC}$,
  $(R(U),\ep(U))$ is a universal morphism from $L$ to $U$. 
  In this case, $R$ is called the \emph{right adjoint of $L$}.
\end{dfn}

\begin{dfn}[Product Category]
  \link{product_cat}
  
  Let $\CC, \DD$ be categories. 
  Then the \emph{product category of $\CC, \DD$} is denoted 
  $\CC\times\DD$ and is defined as follows. 
  \begin{enumerate}
    \item $\obj{\CC\times\DD} := \obj{\CC}\times\obj{\DD}$.
    \item For $(U,X), (V,Y) \in \obj{\CC\times\DD}$, 
    $\mor{(U,X)}{(V,Y)}{\CC\times\DD} := \mor{U}{V}{\CC}\times\mor{X}{Y}{\DD}$.
  \end{enumerate}
\end{dfn}

\begin{prop}[Natural Transformations on Product Category]
  \link{nat_trans_prod_cat}
  
  Let $F, G : \map{\CC\times\DD}{\EE}{}{}$, 
  $(\al_{U,X}) \in \prod_{(U,X) \in \obj{\CC\times\DD}} 
    \mor{F(U,X)}{G(U,X)}{\EE}$.
  Then the following are equivalent. 
  \begin{enumerate}
    \item $\al : \map{F}{G}{}{}$.
    \item For all $(U,X) \in \obj{\CC\times\DD}$, 
    $\al_{U,-} : \map{F(U,-)}{G(U,-)}{}{}$ and 
    $\al_{-,X} : \map{F(-,X)}{G(-,X)}{}{}$.
  \end{enumerate}
\end{prop}
\begin{proof}
  Straight forward.
\end{proof}

\begin{dfn}[Adjunction]
  \link{adjunction}
  
  Let $R : \map{\CC}{\DD}{\CAT}{}$ and $L : \map{\DD}{\CC}{\CAT}{}$.
  We have the two functors 
  $\mor{L(\star)}{-}{\CC}, \mor{\star}{R(-)}{\DD} : 
  \map{\DD\op\times\CC}{\SET}{\CAT}{}$.
  Then $(L,R)$ is an \emph{adjunction} when 
  $\mor{L(\star)}{-}{\CC}, \mor{\star}{R(-)}{\DD}$
  are naturally isomorphic. 
  
  In this case, 
  $R$ is called the \emph{right adjoint of $L$}
  and $L$ is called the \emph{left adjoint of $R$}.
  The isomorphism is called the \emph{adjunction isomorphism}. 
  For all $f : \map{L(X)}{U}{\CC}{}$,
  the image of $f$ under the adjunction isomorphism is called 
  the \emph{adjunct of $f$}, denoted $f^\bot$. 
  Similarly for $g : \map{X}{R(U)}{\DD}{}$, 
  we have the \emph{adjunct of $g$}, denoted $g^\bot$.
\end{dfn}

\begin{prop}[Universal Morphism Characterisation of Adjunction]
  \link{uniprop_char_adj}{}

  Let $R : \map{\CC}{\DD}{\CAT}{}$.
  Then the following are equivalent : 
  \begin{enumerate}
    \item $R$ is a right adjoint. 
    \item There exists $L : \map{\DD}{\CC}{\CAT}{}$ such that 
    $(L,R)$ is an adjunction. 
  \end{enumerate}

  Dually, let $L : \map{\DD}{\CC}{\CAT}{}$.
  Then the following are equivalent : 
  \begin{enumerate}
    \item $L$ is a left adjoint. 
    \item There exists $R : \map{\CC}{\DD}{\CAT}{}$ such that 
    $(L,R)$ is an adjunction. 
  \end{enumerate}
\end{prop}
\begin{proof}
  $(\implies)$
  Let $R$ be a right adjoint, $L : \map{\obj{\DD}}{\obj{\CC}}{}{}$, 
  $\eta \in \Pi X \in \obj{\DD}, \mor{X}{RL(X)}{\DD}$,
  for all $X \in \obj{\DD}$, $(L(X),\eta(X))$ universal morphism from 
  $X$ to $R$.

  The universal properties at every $X \in \obj{\DD}$ implies 
  $L$ is functorial.
  By \hyperlink{uni_iff_rep}{universal iff represents},
  for all $X \in \obj{\DD}$, 
  we have $\mor{L(X)}{-}{\CC} \cong \mor{X}{R(-)}{\DD}$
  as functors $\map{\CC}{\SET}{}{}$.
  Let $f : \map{X}{Y}{\DD}{}$ and $U \in \obj{\CC}$.
  Then we have the following commutative diagram. 
  \begin{figure}[H]
    \centering
    \begin{tikzcd}
      \mor{L(X)}{U}{\CC} \ar[rrr,"R(-)\circ\eta(X)"]
      & & & \mor{X}{R(U)}{\DD} \\
      & g \circ L(f) \ar[r,mapsto]
      & R(g \circ L(f)) \circ \eta(X) = R(g) \circ \eta(Y) \circ f
      & \\
      & g \ar[u,mapsto] \ar[r,mapsto] & R(g)\circ\eta(Y) \ar[u,mapsto] & \\
      \mor{L(Y)}{U}{\CC} \ar[uuu,"(h^{L(f)})_U"]
        \ar[rrr,"R(-)\circ\eta(Y)"{swap}]
      & & & \mor{Y}{R(U)}{\DD} \ar[uuu,"(h^f)_U"{swap}]
    \end{tikzcd}
  \end{figure}
  Thus the isomorphism $\mor{L(X)}{-}{\CC} \cong \mor{X}{R(-)}{\DD}$
  is functorial in $X$, 
  and hence an isomorphism between
  $\mor{L(\star)}{-}{\CC} \cong \mor{\star}{R(-)}{\DD}$.

  $(\limplies)$ 
  Let $L : \map{\obj{\DD}}{\CC}{\CAT}{}$ such that 
  $(L,R)$ is an adjunction. 
  Then for each $X \in \obj{\DD}$, 
  $\mor{L(X)}{-}{\CC} \cong \mor{X}{R(-)}{\DD}$.
  Let $\eta(X)$ be the adjunct of $\id{L(X)}$.
  By \hyperlink{uni_iff_rep}{universal iff represents},
  $(L(X),\eta(X))$ is a universal morphism from $X$ to $R$.

  The dual has a similar argument. 
\end{proof}

\begin{prop}[Uniqueness of Adjoints]
  \link{adj_unique}{}

  Let $R, R_1 : \map{\CC}{\DD}{\CAT}{}$, $L, L_1: \map{\DD}{\CC}{\CAT}{}$.
  Then \begin{enumerate}
    \item If $(L,R)$ and $(L,R_1)$ are both adjunctions, 
    then $R \cong R_1$ as functors. 
    \item If $(L,R)$ and $(L_1, R)$ are both adjunctions, 
    then $L \cong L_1$ as functors. 
  \end{enumerate}
\end{prop}
\begin{proof}
  $(1)$ Let $(L,R), (L,R_1)$ both be adjunctions. 
  Let $f : \map{U}{V}{\CC}{}$. 
  We have an isomorphism between the functors 
  $\mor{-}{R(U)}{\DD}$ and $\mor{-}{R_1(U)}{\DD}$
  for all $U \in \obj{\CC}$. 
  By \hyperlink{yoneda}{Yoneda's lemma}, 
  these isomorphisms are equal to $h_{\al_U}$ 
  for some unique morphism $\al_U : \map{R(U)}{R_1(U)}{\DD}{}$.
  So we have the following commutative diagram. 
  \begin{figure}[H]
    \centering
    \begin{tikzcd}
      \mor{-}{R(U)}{\DD} \ar[r,"h_{\al_U}","\sim"{swap}] 
        \ar[d,"h_{R(f)}"{swap}]
      & \mor{-}{R_1(U)}{\DD} \ar[d,"h_{R_1(f)}"] \\
      \mor{-}{R(V)}{\DD} \ar[r,"h_{\al_V}"{swap},"\sim"] 
      & \mor{-}{R_1(V)}{\DD}
    \end{tikzcd}
  \end{figure}
  Again by Yoneda, 
  we have $R_1(f) \circ \al_U = \al_V \circ R(f)$.
  The fact that $h_{\al_U}$ is an isomorphism implies 
  $\al_U$ is an isomorphism. 
  Thus $\al$ is a natural isomorphism between $R, R_1$.

  $(2)$ Analogous. 
\end{proof}

\begin{rmk}
  There is another characterisation of adjunctions.
  This may be skipped on first reading since 
  in practice, the special case of Galois connection happens more often and 
  the proofs become much easier. 
\end{rmk}

\begin{prop}[Unit/Counit Characterisation of Adjunction]
  \link{unit_char_adj}{}
  
  Let $R : \map{\CC}{\DD}{\CAT}{}$ and $L : \map{\DD}{\CC}{\CAT}{}$.
  Then the following are equivalent : 
  \begin{enumerate}
    \item (Morphism Isomorphism) $(R,L)$ is an adjunction. 
    \item (Unit-Counit) 
    There exists $\eta : \map{\id{\DD}}{RL}{}{}$ and 
    $\ep : \map{LR}{\id{\CC}}{}{}$ such that 
    \begin{enumerate}
      \item $\id{L} = \ep L \circ L \eta$,
      that is to say for all $X \in \obj{D}$, 
      we have the following commutative diagram. 
      \begin{figure}[H]
        \centering
        \begin{tikzcd}
          L(X) \ar[rd,"\id{L(X)}"{swap}] \ar[r,"L(\eta(X))"]
          & LRL(X) \ar[d,"\ep(L(X))"] \\
          & L(X)
        \end{tikzcd}
      \end{figure}
      \item $\id{R} = R\ep \circ \eta R$, i.e. 
      for all $U \in \obj{\CC}$, 
      we have the following commutative diagram. 
      \begin{figure}[H]
        \centering
        \begin{tikzcd}
          R(U) \ar[rd,"\id{R(U)}"{swap}] \ar[r,"\eta(R(U))"] 
          & RLR(U) \ar[d,"R(\ep(U))"] \\
          & R(U) 
        \end{tikzcd}
      \end{figure}
    \end{enumerate}
    The above two equations are often called \emph{triangle-identities}.
  \end{enumerate}
\end{prop}
\begin{proof}
  $(1\implies 2)$
  For all $X \in \obj{\DD}$, 
  the adjunction isomorphism gives an isomorphism of functors
  \[
    \mor{L(X)}{-}{\CC} \iso \mor{X}{R(-)}{\DD}
  \]
  Define $\eta(X) := \id{L(X)}^\bot$. 
  Then by \hyperlink{uni_iff_rep}{universal iff represents}, 
  $(L(X),\eta(X))$ is a universal morphism from $X$ to $R$. 
  We claim that $\eta : \map{\id{\DD}}{RL}{}{}$. 

  Let $f : \map{X}{Y}{\DD}{}$. 
  Then by the universal property of $(L(X),\eta(X))$,
  we have the following commutative diagram. 
  \begin{figure}[H]
    \centering
    \begin{tikzcd}
      X \ar[r,"\eta(X)"] \ar[d,"f"{swap}] & RL(X) \ar[d,"RL(f)"] \\
      Y \ar[r,"\eta(Y)"{swap}] & RL(Y)
    \end{tikzcd}
  \end{figure}
  i.e. $\eta$ is a natural transformation as desired. 
  We similarly define $\ep(U) := \id{R(U)}^\bot$ for $U \in \obj{\CC}$
  and see that $\ep : \map{LR}{\id{\CC}}{}{}$. 

  To prove $(a)$, let $X \in \obj{\DD}$. 
  Then \begin{align*}
    \id{L(X)} 
    = \brkt{\id{L(X)}^\bot}^\bot 
    = \brkt{\eta(X)}^\bot 
    = \ep(L(X)) \circ L(\eta(X)) 
  \end{align*}
  where the last equality follows from 
  the universal property of $(RL(X),\ep(L(X)))$. 
  Similarly for $(b)$, we have for $U \in \obj{\CC}$, 
  \begin{align*}
    \id{R(U)} 
    = \brkt{\id{R(U)}^\bot}^\bot 
    = \brkt{\ep(U)}^\bot 
    = R(\ep(U)) \circ \eta(R(U))
  \end{align*}
  where the last equality is by
  the universal property of $(LR(U),\eta(R(U)))$.

  $(2\implies 1)$ 
  Let $(X,U)\in\obj{\DD\op \times \CC}$. 
  Since $(L(X),\eta(X))$ is supposed to be 
  a universal morphism from $X$ to $R$, 
  we define the adjunction map to be 
  \begin{align*}
    \mor{L(X)}{U}{\CC} &\overset{\bot}{\longleftrightarrow} 
    \mor{X}{R(U)}{\DD} \\
    f &\longmapsto R(f) \circ \eta(X) \\
    \ep(U) \circ L(g) &\longmapsfrom g
  \end{align*}
  Then for $f : \map{L(X)}{U}{\CC}{}$, 
  \begin{align*}
    \brkt{f^\bot}^\bot 
    &= \ep(U) \circ L(f^\bot) = \ep(U) \circ L\brkt{R(f) \circ \eta(X)} \\
    &= \ep(U) \circ LR(f) \circ L(\eta(X)) 
    = f \circ \ep(L(X)) \circ L(\eta(X)) = f
  \end{align*}
  Similarly, $\brkt{g^\bot}^\bot = g$.
  So $\bot$ is an isomorphism at all $(X,U)$.
  
  It remains to show naturality. 
  \hyperlink{nat_trans_prod_cat}{It suffices} to show that 
  the isomorphism is natural in both components. 
  Let $f : \map{X}{Y}{\DD\op}{}$. 
  Then we have the following diagram. 
  \begin{figure}[H]
    \centering
    \begin{tikzcd}
      \mor{L(X)}{U}{\CC} \ar[r,"\bot"] \ar[d,"h^{L(f)}"{swap}]
      & \mor{X}{R(U)}{\DD} \ar[d,"h^f"] \\
      \mor{L(Y)}{U}{\CC} \ar[r,"\bot"{swap}] & \mor{Y}{R(U)}{\DD}
    \end{tikzcd}
  \end{figure}
  It follows from $\eta : \map{\id{\DD}}{RL}{}{}$ that the above commutes.
  Similarly, naturality of $\ep$ implies naturality in the second component. 
  Hence $\bot$ is a natural isomorphism as desired. 
\end{proof}

\begin{rmk}
  The following is a special case of adjunction that is worth noting. 
\end{rmk}
\begin{dfn}[Galois Connection]
  
  Let $I, J$ be partially ordered sets. 
  Then $I, J$ can be seen as categories. 
  A \emph{monotone Galois connection between $I,J$} is 
  an adjunction between $I, J$.
  A \emph{antitone Galois connection between $I,J$} is 
  an adjunction between $I\op, J$. 
\end{dfn}
\begin{rmk}
  The \hyperlink{unit_char_adj}{unit/counit characterisation of adjunctions}
  shows that if $(R,L)$ is a Galois connection (mono or anti) between 
  partially ordered sets $I,J$, 
  then $R$ and $L$ are bijective on their images. 
\end{rmk}

\begin{dfn}[Free Functors]\link{free}
  
  Let $\CC$ be a subcategory of $\DD$ and 
  $L : \DD \to \CC$ be a functor. 
  Then $L$ is called a \emph{free functor from $\DD$ to $\CC$} when 
  it is left adjoint to the forgetful functor from $\CC$ to $\DD$. 
  Note that if it exists, $L$ is then unique up to natural isomorphism,
  so it is customary to called it \emph{the} free functor. 
\end{dfn}

\begin{eg}[Adjunctions]
  
  The adjunctions hinted at in \linkto{eg:functors}{section 2}.

  First, a list of free, forgetful adjunctions.
  \begin{itemize}
    \item (Free Group) 
    What we previously said about taking free groups is precisely 
    the a fact that it's a universal morphism from $S$ to the forgetful functor.
    The forgetful functor is a right adjoint.
    Its left adjoint is taking free groups. 
    \[
      \GRP(\<-\>,\star) \iso \SET(-,\star)
    \]
    \item (Free Left Module over a Ring) Similarly, 
    for a ring $R$, 
    taking free left modules over a set is a left adjoint to forgetful 
    $R\MOD \to \SET$. 
    \[
      R\MOD(R^{\oplus(-)},\star) \iso \SET(-,\star)
    \]
    \item (Free Commutative Algebra over a Ring)
    For a commutative ring $K$, \[
      \CALG(K)(K[-],\star) \iso \SET(-,\star)
    \]
    \item (Extension and Contraction of Scalars)
    Let $B$ be a commutative algebra over $A$ 
    ($A$ is commutative by definition).
    Then extension of scalars is the free functor left adjoint to
    the forgetful $B\MOD \to A\MOD$. \[
      B\MOD((-) \otimes_A B, \star) \iso A\MOD(-,\star)
    \]
    \item (Localization of Modules)
    Let $A$ be a commutative ring and $S \subs A$ a multiplicative set. 
    Then localization with respect to $S$ is the free functor
    left adjoint to the forgetful $S\inv A \MOD \to A\MOD$. \[
      S\inv A\MOD(S\inv(-), \star) \iso A\MOD(-,\star)
    \]
    Note that by \linkto{adj_unique}{uniqueness of adjoints},
    this can be seen as a special case of extension and contraction of scalars
    via the commutative $A$-algebra $S\inv A$.

    \item (Symmetric Algebra)
    Let $A$ be a commutative ring. 
    Then taking symmetric algebas of modules is the free functor 
    left adjoint to the forgetful $\CALG(A) \to A\MOD$.
    \[
      \CALG(A)(Symm(-),\star) \iso A\MOD(-,\star)
    \]

    \item (Discrete Topology)
    Giving sets the discrete topology is the free functor left adjoint to 
    the forgetful $\TOP \to \SET$. 
    \[
      \TOP((-,2^-),\star) \iso \SET(-,\star)
    \]

    \item (Free Category on a Pre-Ordered Set)
    As \linkto{eg:cat_ord}{previously discuessed},
    every pre-ordered set can be made into a category by considering 
    the relation $\leq$ as morphisms. 
    This is the free functor left adjoint to the forgetful 
    $\CAT \to \ORD$, where we see any category $\CC$ as a pre-ordered set by
    declaring for $U, V \in \obj{\CC}$, $U \leq V := \CC(U,V) \neq \nothing$.

  \end{itemize}

  A list of adjunctions, 
  themed ``moving structures on objects across morphisms''. 
  \begin{itemize}
    \item (Image, Preimage of ``Substructures'')
    Let $f : U \to V$ be a morphism of sets. 
    Then \[
      2^U(-,f\inv(\star)) \iso 2^V(f(-),\star)
    \]
    If $f$ is instead a morphism of groups,
    then this upgrades to \[
      \mathbf{Sub}\GRP(U)(-,f\inv(\star)) \iso 
      \mathbf{Sub}\GRP(V)(f(-),\star)
    \]
    If $f$ is instead a morphism of rings, 
    \[
      \mathbf{Sub}\RING(U)(-,f\inv(\star)) \iso 
      \mathbf{Sub}\RING(V)(f(-),\star)
    \]
    If $f$ is instead a morphism of left $R$-modules for some 
    ring $R$, \[
      \mathbf{Sub}R\MOD(U)(-,f\inv(\star)) \iso
      \mathbf{Sub}R\MOD(V)(f(-),\star)
    \]
    If $f$ is instead a morphism of $K$-algebras for some commutative ring $K$,
    \[
      \mathbf{Sub}K\ALG(U)(-,f\inv(\star)) \iso 
      \mathbf{Sub}K\ALG(V)(f(-),\star)
    \]
    \item (Image, Preimage of Filters)
    Let $f : U \to V$ be a morphism of sets. 
    Then \[
      \fil(U)(-,{f}\inv(\star)) \iso \fil(V)(f_{\fil}(-),\star)
    \]
    where for a filter $F$ on $U$,
    $f_{\fil}(F) := \set{W \subs V \st f\inv W \in F}$.
  \end{itemize}

  % Given a category $\CC$ with finite products,
  % for a fixed $Y \in \obj{\CC}$, 
  % $(-)\times Y : \CC \to \CC$ is functorial. 
  % Fixing another $Z \in \obj{\CC}$,
  % an \emph{exponential object} $(Z^Y,ev)$ is a universal morphism 
  % from $(-)\times Y$ to $Z$.
  The following is a list of instances of ``exponential objects''.
  \begin{itemize}
    \item Let $Y$ be a set. 
    Taking the cartesian product with $Y$, 
    $(-)\times Y : \SET \to \SET$, is functorial.
    Then we have the adjunction between $(-)\times Y$ and $h^Y$ : 
    \[
      \SET((-)\times Y, \star) \iso \SET(-, \SET(Y, \star))
    \]
    \item Let $N$ be an $A$-module where $A$ is a commutative ring. 
    Then $(-) \otimes_A N : A\MOD \to A\MOD$ is functorial.
    On the other hand, the coYoneda embedding $h^N$ of $N$ is naturally 
    a functor $A\MOD \to A\MOD$. 
    Then we have the adjunction between $(-)\otimes_A N$ and $h^N$ : 
    \[
      A\MOD((-)\otimes_A N, \star) \iso A\MOD(-,h^N(\star))
    \]
    \item Let $\DD$ be a category. 
    Then taking the product of categories with $\DD$ yields a 
    functor $(-) \times \DD : \CAT \to \CAT$.
    On the other hand, the coYoneda embedding of $\DD$ is naturally a functor
    $h^\DD : \CAT \to \CAT$. 
    Then we have the adjunction : \[
      \CAT((-)\times \DD, \star) \iso \CAT(-,h^\DD(\star))
    \]
  \end{itemize}

  (TODO : Add to this list)

\end{eg}
\section{Limits and Colimits}
\begin{dfn}[(Co)Diagrams]\link{diagrams}
  
  Let $\II, \CC$ be categories. 
  Then an \emph{$\II$-diagram in $\CC$} is 
  a covariant functor from $\II$ to $\CC$. 
  Dually, an \emph{$\II$-codiagram} is a contravariant functor 
  from $\II$ to $\CC$,
  i.e. an $\II\op$-diagram. 
\end{dfn}

\begin{rmk}
  Often, it is easier to take $\II$ to be a subcategory of $\CC$. 
\end{rmk}

\begin{dfn}[Constant (Co)Diagrams]\link{const}
  
  Let $\II, \CC$ be categories and $U \in \obj{\CC}$. 
  Then define the \emph{constant diagram $\De(U)$} as follows. 
  \begin{enumerate}
    \item For all $i \in \II$, $\De(U)(i) := U$. 
    \item For all $\phi : \map{i}{j}{\II}{}$, 
    $\De(U)(\phi) := \id{U}$.
  \end{enumerate}
  Dually, we have the \emph{constant codiagram $\De\op(U)$} defined as : 
  \begin{enumerate}
    \item For all $i \in \obj{\II}$, $\De\op(U)(i) := U$. 
    \item For all $\phi : \map{i}{j}{\II\op}{}$, 
    $\De(U)(\phi) := \id{U}$.
  \end{enumerate}
\end{dfn}

\begin{prop}[Functoriality of Constant (Co)Diagrams]\link{const_funk}
  
  Let $\II, \CC$ be categories. 
  Then $\De : \map{\CC}{\CC^\II}{}{}$.
  Dually, $\De\op : \map{\CC}{\CC^{\II\op}}{}{}$.
\end{prop}

\begin{dfn}[(Co)Limits of (Co)Diagrams]\link{limit}
  
  Let $\II, \CC$ be categories, $X$ a $\II$-diagram in $\CC$,
  and $Y$ a $\II$-codiagram in $\CC$.

  Then a \emph{limit of $X$} is a universal morphism 
  from $\De$ to $X$. 
  If a limit of $X$ exists, 
  it is \linkto{canonical}{canonical} and 
  referred to as \emph{the} limit,
  denoted $(\LIM{X}{}, \pi_X)$.
  
  Dually, a \emph{colimit of $Y$} is a universal morphism from $Y$ to $\De\op$. 
  If a colimit of $Y$ exists, 
  it is canonical and referred to as \emph{the} colimit, 
  denoted with $(\COLIM{Y}{}, \io_Y)$.
\end{dfn}

\begin{rmk}
  Sometimes limits are also called \emph{projective limits},
  and colimits are called \emph{injective limits}. 
\end{rmk}

\begin{dfn}[(Co)Completeness]\link{complete}
  
  Let $\CC$ be a category. 
  Then it is called \emph{complete} when 
  for all ``small'' categories $\II$ and diagrams $X : \map{\II}{\CC}{\CAT}{}$,
  there exists the limit of $X$.

  Dually, it is called \emph{cocomplete} when 
  for all ``small'' categories $\II$ and 
  codiagrams $Y : \map{\II\op}{\CC}{\CAT}{}$,
  there exists the colimit of $Y$.
\end{dfn}

\begin{rmk}
  We now cover important examples of limits and colimits. 
\end{rmk}

\begin{dfn}[Discrete Category]\link{discrete}
  
  For $I \in \obj{\SET}$, 
  $I$ can be turned into a category by having elements as objects 
  and the only morphisms being identity morphisms. 
  Categories obtained in this way are called \emph{discrete categories}.
\end{dfn}

\begin{rmk}
  Note that for a discrete category $\II$,
  $\II$ and $\II\op$ are isomorphic in an obvious way. 
  Consequently, it is best to think of 
  $\II$-diagrams and $\II$-codiagrams as ``the same''.
\end{rmk}

\begin{dfn}[(Co)Products]\link{prod}
  
  Let $\CC$ be a category and $\II$ a discrete category. 

  Let $X$ be an $\II$-diagram in $\CC$.
  Then the limit of $X$ is called the \emph{product of $X(i)$}.

  Dually, let $Y$ be an $\II$-codiagram in $\CC$.
  Then the colimit of $Y$ is called the \emph{coproduct of $Y(i)$}.

  In the special case of $I = \nothing$, 
  the product is called the \emph{final object of $\CC$}.
  Dually, the coproduct is called the \emph{initial object of $\CC$}.
\end{dfn}

\begin{eg}[Final Objects]\link{eg_final}
  
\end{eg}

\begin{eg}[Initial Objects]\link{eg_initial}
  
\end{eg}

\begin{eg}[Products]\link{eg_prod}
  
\end{eg}

\begin{eg}[Coproducts]\link{eg_coprod}
  
\end{eg}

\begin{dfn}[(Co)Equalizers]\link{equalizer}
  
  Let $\CC$ be a category. 
  Let $I$ be an arbitrary set and $\II$ be the following category. 
  \begin{figure}[H]
    \centering
    \begin{tikzcd}
      0 \ar[loop above,"\id{0}"] \ar[r,"i"]
      & 1 \ar[loop above,"\id{1}"] 
    \end{tikzcd}
  \end{figure}
  where there is a morphism $i : \map{0}{1}{\II}{}$ for all $i \in I$. 

  Let $X$ be an $\II$-diagram in $\CC$. 
  Then the limit of $X$ is called 
  the \emph{equalizer of $X(i)$'s}. 
  Dually, let $Y$ be an $\II$-codiagram in $\CC$. 
  Then the colimit of $Y$ is called the 
  \emph{coequalizer of $Y(i)$'s}.
\end{dfn}

\begin{eg}[Equalizers]
  
\end{eg}

\begin{eg}[Coequalizers]
  
\end{eg}

\begin{dfn}[Pullbacks and Pushouts]\link{pullback}
  
  Let $\CC$ be a category, $U \in \obj{\CC}$.
  Then a \emph{pullback over $U$} is a product in the 
  category $\CC\darrow U$.
  Dually, a \emph{pushout under $U$} is a coproduct in the 
  category $U\darrow \CC$.

  Let $I$ be an arbitrary set and $\II$ the following category. 
  \begin{figure}[H]
    \centering
    \begin{tikzcd}
      i \ar[loop above,"\id{0}"] \ar[r,"\phi(i)"]
      & * \ar[loop above,"\id{1}"] 
    \end{tikzcd}
  \end{figure}
  \begin{enumerate}
    \item $\obj{\II} = I \sqcup \set{*}$.
    \item For all $x \in \obj{\II}$, $\mor{x}{x}{\II} = \set{\id{x}}$.
    \item For all $i \in I$, $\mor{i}{*}{\II} = \set{\phi(i)}$.
  \end{enumerate}
  Then a pullback over $U$ is equivalently the limit of 
  an $\II$-diagram $X$ with $X(*) = U$. 
  Dually, a pushout under $U$ is equivalently the colimit of 
  an $\II$-codiagram $Y$ with $Y(*) = U$.
\end{dfn}
\section{Completeness}
\begin{dfn}[Finite Categories]\link{fin}
  
  Let $\CC$ be a category. 
  Then it is called \emph{finite} when 
  $\obj{\CC}$ is finite and for any $U, V \in \obj{\CC}$,
  $\CC(U,V)$ is finite. 
\end{dfn}

\begin{dfn}[(Finite) (Co)Completeness]\link{complete}
  
  Let $\CC$ be a category. 
  Then it is called \emph{(finitely) complete} when 
  for all (finite) categories $\II$ and diagrams $X : \map{\II}{\CC}{\CAT}{}$,
  there exists the limit of $X$ in $\CC$.

  Dually, it is called \emph{(finitely) cocomplete} when 
  for all (finite) categories $\II$ and 
  diagrams $X : \map{\II}{\CC}{\CAT}{}$,
  there exists the colimit of $X$ in $\CC$.
\end{dfn}

\begin{rmk}
  If a category $\CC$ has nice constructions,
  then for any category $\II$, $\CC^\II$ will often inherit these constructions.
  This is the following result. 
\end{rmk}

\begin{prop}[Characterisation of (Co)Limits of Functors]
  \link{char_lim_funk}
  
  Let $\CC, \DD$ be categories.
  Let $X : \II \to \DD^\CC$ be an $\II$-shaped diagram of functors from 
  $\CC$ to $\DD$.

  Let $(F,\phi) \in \obj{\De\darrow X}$.
  For each $U \in \obj{\CC}$,
  we obtain a functor $X_U : \II \to \DD, i \in \obj{\II} \mapsto X(i,U)$
  by restricting the second argument to $U$
  (mapping morphisms is obvious). 
  In a similar way, we obtain $(F(U),\phi_U) \in \obj{\De\darrow X_U}$.
  Then the following are equivalent : 
  \begin{itemize}
    \item For all $U \in \obj{\CC}$, 
    $(F(U),\phi_U)$ is a limit of $X_U$.
    \item $(F,\phi)$ is a limit of $X$ and for every $U\in\obj{\CC}$,
    the limit of $X_U$ exists in $\DD$.
  \end{itemize}
  Paraphrasing : if $\DD$ has limits of $\II$-shaped diagrams,
  then so does $\DD^\CC$.

  Dually, let $(F,\phi) \in \obj{X\darrow\De}$.
  For each $U \in \obj{\CC}$,
  we obtain a functor $X^U : \II \to \DD, i \in \obj{\II} \mapsto X(i,U)$
  (mapping morphisms is obvious). 
  In a similar way, we obtain $(F(U),\phi^U) \in \obj{X^U\darrow\De}$.
  Then the following are equivalent : 
  \begin{itemize}
    \item For all $U \in \obj{\CC}$, 
    $(F(U),\phi^U)$ is a colimit of $X^U$.
    \item $(F,\phi)$ is a colimit of $X$ and for every $U \in \obj{\CC}$,
    the colimit of $X^U$ exists in $\DD$.
  \end{itemize}
  Paraphrasing : if $\DD$ has colimits of $\II$-shaped diagrams,
  then so does $\DD^\CC$.
\end{prop}
\begin{proof}
  
  $(\implies)$
  Suppose for all $U \in \obj{\CC}$, 
  $(F(U),\phi_U)$ is a limit of $X_U$.
  We show that $(F,\phi)$ has the desired UP.
  Let $(G,\ga) \in \obj{\De\darrow X}$.
  For each $U \in \obj{\CC}$,
  define $(G(U),\ga_U) \in \obj{\De\darrow X_U}$
  by restricting the second argument to $U$.
  Then by the UP of limits, 
  we have a unique morphism $\LIM \ga_U \in 
  \De\darrow X_U((G(U),\ga_U),(F(U),\phi_U))$.
  \begin{cd}
    \De(G(U)) \ar[r,"\De(\LIM \ga_U)", dashed] \ar[rd,"\ga_U"{swap}] & 
    \De(F(U)) \ar[d,"\phi_U"] &
    \De(G) \ar[r,"\De(\LIM \ga)", dashed] \ar[rd,"\ga"{swap}] & 
    \De(F) \ar[d,"\phi"] \\
    & 
    X_U &
    & 
    X
  \end{cd}
  Then again by the UP of limits,
  the left triangle is functorial in $U$,
  so the collection of $(\LIM \ga_U)_{U \in \obj{\CC}}$ forms a 
  $\LIM \ga \De\darrow X((G,\ga),(F,\phi))$.
  The uniqueness of $\LIM \ga$ comes from 
  the componentwise uniqueness of $(\LIM \ga_U)_{U \in \obj{\CC}}$.

  $(\limplies)$
  Let $(F,\phi)$ be a limit of $X$ and 
  suppose for every $U \in \obj{\CC}$,
  the limit of $X_U$ exists. 
  Then we can use the forward implication to explicitly construct a limit 
  $(\bar{F},\bar{\phi})$ of $X$ by defining
  the components $(\bar{F}(U),\bar{\phi}_U)$ to be limits of $X_U$.
  Functoriality of $\bar{F}$ comes from the UP of limits in each component.
  Since limits are canonical,
  we have a unique isomorphism between $(F,\phi)$ and $(\bar{F},\bar{\phi})$.
  Restricting to each component yields an isomorphism of cones 
  $(F(U),\phi_U) \iso (\bar{F}(U),\bar{\phi}_U)$ for every $U \in \obj{\CC}$.
  Since $(\bar{F}(U),\bar{\phi}_U)$ is by construction a limit of $X_U$,
  and being isomorphic to the limit \linkto{iso_uni_implies_uni}{implies}
  being a limit.
  We conclude that $(F(U),\phi_U)$ is a limit of $X_U$ for every $U$.

  The dual argument is similar. 
\end{proof}

\begin{rmk}
  The following result gives an easier check for (co)completeness.
\end{rmk}

\begin{prop}[Characterisation of Completeness, Cocompleteness]
  \link{char_complete}
  
  Let $\CC$ be a category. 
  Then the following are equivalent : 
  \begin{itemize}
    \item $\CC$ is (finitely) complete.
    \item $\CC$ has (finite) products and (finite) equalizers.
  \end{itemize}
  Dually, the following are equivalent : 
  \begin{itemize}
    \item $\CC$ is (finitely) cocomplete.
    \item $\CC$ has (finite) coproducts and (finite) coequalizers. 
  \end{itemize}
\end{prop}
\begin{proof}
  For fun, we prove the dual argument instead for once.
  It suffices to prove the reverse implication. 
  Let $X : \II \to \CC$ be a diagram. 
  Define the \emph{arrow category of $\II$} as follows : 
  take the free category on the partially ordered set $\2 := \set{0,1}$ and 
  $\arr(\II) := \II^{\2}$.
  Define \emph{source} and \emph{target} functions 
  $s, t : \obj{\arr(\II)} \to \obj{\II}, (\phi : i \to j) \mapsto i, j$
  respectively. 
  Let $(\coprod_{\phi \in \obj{\arr\II}} X_{s(\phi)},\io)$
  be the coproduct of $X_{s(\phi)}$ across all arrows $\phi$ in $\II$ and 
  $(\coprod_{i \in \obj{\II}} X_i,\lift{}{})$ the coproduct of 
  $X(i)$ across all $i$ in $\II$.
  Now consider the following diagram : 
  \begin{cd}
    \coprod_{\phi \in \obj{\arr\II}} X_{s(\phi)} 
      \ar[r,"\al",yshift = 1mm] \ar[r,"\be"{swap},yshift = -1mm]&
    \coprod_{i \in \obj{\II}} X_i
  \end{cd}
  where $\al$ is defined using the UP of the coproduct via 
  for all $\phi \in \obj{\arr\II}$, $\al \circ \io_\phi := \lift{s(\phi)}{}$;
  $\be$ is also defined using the UP of the coproduct, but via 
  for all $\phi \in \obj{\arr\II}$, 
  $\be\circ \io_\phi := \lift{t(a)}{} \circ X(\phi)$. 
  It is then straightforward to check that 
  the coequalizer of $\al,\be$ with obvious morphisms from $X$ into it
  is a colimit of $X$. 
  Note that the above proof still works for finite categories $\II$. 

  The non-dual argument is similar. 
\end{proof}

\begin{rmk}
  We now proceed to show the (co)completeness of some standard categories.
  As this is unimportant to the theory, 
  the reader may skip to the next section. 
  
  For fun, we will not use the above characterisation,
  though the constructions are of similar nature. 
\end{rmk}

\begin{prop}[$\SET$ (Co)Complete]\link{set_complete}
  
  Let $I$ be a category and $X$ an $I$-shaped diagram in $\SET$.

  Consider the set $\SET^\II(\De(1),X)$ where $1$ is a singleton set.
  There is an obvious $\pi_X : \De(\SET^\II(\De(1),X)) \to X$.
  Then $(\SET^\II(\De(1),X),\pi_X)$ is a limit of $X$ in $\SET$.

  Dually, we construct the colimit of $X$ as follows. 
  For every $(i,x_i) \in \bigsqcup_{i \in I} X_i$,
  define $[(i,x_i)]$ to be the subset of $(j,x_j) \in \bigsqcup_{i \in I} X_i$
  such that there exists 
  a ``zig zag of elements and morphisms in $X$ joining them''
  (unenlightening to write down fully) : 
  \begin{cd}[cramped]
    (i,x_i) \ar[rd] & & 
    (k_1, x_1) \ar[ld] \ar[rd] & & 
    (k_n, x_n) \ar[ld] \ar[rd] & & 
    (j,x_j) \ar[ld] \\
    & (l_0,y_0) & 
    & \cdots & 
    & (l_n,y_n) & 
  \end{cd}
  Then $Q := \set{[(i,x_i)] \st (i,x_i) \in \bigsqcup_{i \in I} X(i)}$ forms
  a partition of $\bigsqcup_{i \in I} X(i)$,
  corresponding to the minimal equivalence relation generated by 
  $(i,x_i) \sim (j,x_j) := \exists \phi \in I(i,j), X(\phi)(x_i) = x_j$.
  There is an obvious morphism $\io_X : X \to \De(Q)$.
  Then $(Q,\io_X)$ is a colimit of $X$.

\end{prop}
\begin{proof}
  Straightforward.
\end{proof}

\begin{prop}[$\TOP$ Complete]\link{top_complete}
  
  Let $I$ be a category and $X$ an $I$-shaped diagram in $\TOP$.

  Let $(\LIM_\SET X,\pi_X)$ be the limit of $X_\SET$,
  obtained by forgeting from $\TOP$ to $\SET$.
  Let $\LIM X$ be endowed with 
  the topology generated by preimages of opens in components. 
  Then for all $(Y,y) \in \obj{\De\darrow X_\SET}$ where $Y \in \obj{\TOP}$
  and $\bar{y} \in \De\darrow X((Y,y),(\LIM_\SET X,\pi_X))$,
  $\bar{y} \in \TOP(Y,\LIM X)$ if and only if 
  $y \in \TOP^\II(\De(Y),X)$.
  Hence, $\pi_X \in \TOP^\II(\De(\LIM X), X)$ and 
  $(\LIM X, \pi_X)$ is a limit of $X$ in $\TOP$.

  Dually, similar trick to turn the colimit in $\SET$ to colimit in $\TOP$ : 
  take the topology generated by 
  sets whose preimages are open in every component of the diagram $X$.
\end{prop}
\begin{proof}
  Straightforward.
\end{proof}

\begin{prop}[$\GRP$ Complete]\link{grp_complete}
  
  Let $I$ be a category and $X$ an $I$-shaped diagram in $\GRP$.

  The construction $\SET^\II(\De(1),X)$ from 
  the limit of $X$ as a diagram in $\SET$ has a group structure by
  multiplying ``objectwise'' in the diagram $X$.
  This makes $\pi_X \in \GRP^{\II}(\De(\SET^\II(\De(1),X)),X)$,
  and hence $\SET^\II(\De(1),X)$ a limit of $X$ in $\GRP$.

  Dually, we imitate the colimit construction in $\SET$ and consider 
  the free product quotiented by the subgroup generated by 
  ``eventually being equal in $X$'' : 
  \[
    \brkt{\bigsqcup_{i \in\II} X_i} /
    \<(i,x_i)(j,x_j)\inv \st \exists\,\phi\in\II(i,j), X(\phi)(x_i) = x_j\>
  \]
  Then the above is a colimit of $X$.

\end{prop}
\begin{proof}
  Straightforward. 
\end{proof}

\begin{rmk}
  We will show (co)completeness of modules over a ring first,
  then show (co)completeness of rings,
  since for the case of colimits, 
  we will make use of the fact that $\RING$ forgets into $\AB = \Z\MOD$. 
\end{rmk}

\begin{prop}[$R\MOD$ Complete]\link{mod_complete}
  
  Let $R \in \RING$, $\II$ a category, 
  $X$ an $\II$-diagram in $R\MOD$.

  The construction $\SET^\II(\De(1),X)$ from 
  the limit of $X$ as a diagram in $\SET$ has an $R\MOD$ structure by
  adding and scalar multiplying ``objectwise'' in the diagram $X$.
  This gives the morphism $\pi_X \in R\MOD^{\II}(\De(\SET^\II(\De(1),X)),X)$,
  and hence $\SET^\II(\De(1),X)$ a limit of $X$ in $R\MOD$.

  Dually, we imitate the colimit construction in $\SET$ and consider 
  the direct sum quotiented by the left $R$-submodule generated by 
  ``eventually being equal in $X$'' : 
  \[
    \brkt{\bigoplus_{i\in\II} X_i} / 
    R\set{(i,x_i) - (j,x_j) \st \exists\, \phi\in\II(i,j), X(\phi)(x_i) = x_j}
  \]
  Then the above is a colimit for $X$ in $R\MOD$.
\end{prop}
\begin{proof}
  Straightforward.
\end{proof}

\begin{prop}[$\RING$ Complete]\link{ring_complete}
  
  Let $\II$ be a category and $X$ an $\II$-shaped diagram in $\RING$.

  The construction $\SET^\II(\De(1),X)$ from 
  the limit of $X$ as a diagram in $\SET$ has a ring structure by
  adding and multiplying ``objectwise'' in the diagram $X$.
  Then $\pi_X \in \RING^{\II}(\De(\SET^\II(\De(1),X)),X)$,
  and hence $\SET^\II(\De(1),X)$ a limit of $X$ in $\RING$.

  Dually, things get nasty. 
\end{prop}
\begin{proof}
  (Nasty things go here)
  (TODO)
\end{proof}

\begin{rmk}
  One may wonder : 
  (1) Why are all the limits for these categories the same as in $\SET$? 
  (2) Why do the colimits differ to $\SET$ for everything except $\TOP$?

  The \linkto{adjoint_cts}{answer} is the related to adjunctions,
  and is the focus of the next section. 
  In a nutshell : 
  all of the following categories are subcategories of $\SET$
  which admit free functors. 
  This implies that the forgetful functor preserves limits.
  This answers the first question. 
  This is also why colimits differ to the ones in $\SET$
  \emph{except} for the case of $\TOP$,
  since the forgetful functor also admits a \emph{right adjoint},
  namely taking the indiscrete topology. 
  This then implies the forgetful functor also preserves colimits.
  This answers the second question.
\end{rmk}
\section{Abelian Categories}
\begin{dfn}[Zero Objects]\label{zero}
  
\end{dfn}

\begin{dfn}[Kernels and Cokernels]\label{ker}
  
\end{dfn}

\end{document}