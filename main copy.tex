\documentclass{article}

\usepackage[left=1in,right=1in]{geometry}
\usepackage{subfiles}
\usepackage{amsmath, amsthm, amssymb, verbatim} % thms
\usepackage{hyperref,nameref,cleveref,thmtools,enumitem} % Easy to use thms
\usepackage[dvipsnames]{xcolor} % Fancy Colours
\usepackage{mathrsfs} % Fancy font
\usepackage{tikz, tikz-cd, float} % Commutative Diagrams
\usepackage{mdframed} % Customizable Boxes
\usepackage{perpage}
\usepackage{parskip} % So that paragraphs look nice
\usepackage{ifthen,xargs} % For defining better commands

% Shortcuts

% % Misc
\newcommand{\brkt}[1]{\left(#1\right)}
\newcommand{\sqbrkt}[1]{\left[#1\right]}
\newcommand{\dash}{\text{-}}

% % Logic
\renewcommand{\implies}{\Rightarrow}
\renewcommand{\iff}{\Leftrightarrow}
\newcommand{\limplies}{\Leftarrow}

% % Sets
\DeclareMathOperator{\supp}{supp}
\newcommand{\set}[1]{\left\{#1\right\}}
\newcommand{\st}{\,|\,}
\newcommand{\minus}{\setminus}
\newcommand{\subs}{\subseteq}
\newcommand{\ssubs}{\subsetneq}
\DeclareMathOperator{\im}{Im}
\newcommand{\nothing}{\varnothing}

% % Greek 
\newcommand{\al}{\alpha}
\newcommand{\be}{\beta}
\newcommand{\ga}{\gamma}
\newcommand{\de}{\delta}
\newcommand{\ep}{\varepsilon}
\newcommand{\io}{\iota}
\newcommand{\ka}{\kappa}
\newcommand{\la}{\lambda}
\newcommand{\om}{\omega}

% % Mathbb
\newcommand{\N}{\mathbb{N}}
\newcommand{\Z}{\mathbb{Z}}
\newcommand{\Q}{\mathbb{Q}}
\newcommand{\R}{\mathbb{R}}
\newcommand{\C}{\mathbb{C}}
\newcommand{\F}{\mathbb{F}}
\newcommand{\bP}{\mathbb{P}}

% % Mathcal
\newcommand{\CC}{\mathcal{C}}
\newcommand{\DD}{\mathcal{D}}
\newcommand{\EE}{\mathcal{E}}

% % Mathfrak
\newcommand{\f}[1]{\mathfrak{#1}}

% % Mathrsfs
\newcommand{\s}[1]{\mathscr{#1}}

% % Category Theory
\newcommand{\obj}[1]{\mathrm{Obj}\left(#1\right)}
\newcommand{\Hom}[3]{\mathrm{Hom}_{#3}(#1, #2)\,}
\newcommand{\mor}[3]{\mathrm{Mor}_{#3}(#1, #2)\,}
\newcommand{\End}[2]{\mathrm{End}_{#2}#1\,}
\newcommand{\aut}[2]{\mathrm{Aut}_{#2}#1\,}
\newcommand{\CAT}{\mathbf{Cat}}
\newcommand{\SET}{\mathbf{Set}}
\newcommand{\TOP}{\mathbf{Top}}
\newcommand{\GRP}{\mathbf{Grp}}
\newcommand{\RING}{\mathbf{Ring}}
\newcommand{\MOD}[1][R]{#1\text{-}\mathbf{Mod}}
\newcommand{\VEC}[1][K]{#1\text{-}\mathbf{Vec}}
\newcommand{\ALG}[1][R]{#1\text{-}\mathbf{Alg}}
\newcommand{\PSH}[1]{\mathbf{PSh}\brkt{#1}}
\newcommand{\map}[4]{#1 \yrightarrow[#4][4pt]{#3}[-1pt] #2}
\newcommand{\op}{^{op}}
\newcommand{\darrow}{\downarrow}

% % Algebra
\newcommand{\iso}{\cong}
\newcommand{\nsub}{\trianglelefteq}
\newcommand{\id}[1]{\mathrm{id}_{#1}}
\newcommand{\inv}{^{-1}}

% % Analysis
\newcommand{\abs}[1]{\left\vert #1 \right\vert}
\newcommand{\norm}[1]{\left\Vert #1 \right\Vert}
\renewcommand{\bar}[1]{\overline{#1}}
\newcommand{\<}{\langle}
\renewcommand{\>}{\rangle}
\renewcommand{\hat}[1]{\widehat{#1}}
\renewcommand{\check}[1]{\widecheck{#1}}

% % Galois
\newcommand{\Gal}[2]{\mathrm{Gal}_{#1}(#2)}
\DeclareMathOperator{\Orb}{Orb}
\DeclareMathOperator{\Stab}{Stab}
\newcommand{\emb}[3]{\mathrm{Emb}_{#1}(#2, #3)}
\newcommand{\Char}[1]{\mathrm{Char}#1}

% Arrows with text above and below with adjustable displacement
% (Stolen from Stackexchange)
\newcommandx{\yaHelper}[2][1=\empty]{
\ifthenelse{\equal{#1}{\empty}}
  % no offset
  { \ensuremath{ \scriptstyle{ #2 } } } 
  % with offset
  { \raisebox{ #1 }[0pt][0pt]{ \ensuremath{ \scriptstyle{ #2 } } } }  
}

\newcommandx{\yrightarrow}[4][1=\empty, 2=\empty, 4=\empty, usedefault=@]{
  \ifthenelse{\equal{#2}{\empty}}
  % there's no text below
  { \xrightarrow{ \protect{ \yaHelper[ #4 ]{ #3 } } } } 
  % there's text below
  {
    \xrightarrow[ \protect{ \yaHelper[ #2 ]{ #1 } } ]
    { \protect{ \yaHelper[ #4 ]{ #3 } } } 
  } 
}

%% code from mathabx.sty and mathabx.dcl to get some symbols from mathabx
\DeclareFontFamily{U}{mathx}{\hyphenchar\font45}
\DeclareFontShape{U}{mathx}{m}{n}{
      <5> <6> <7> <8> <9> <10>
      <10.95> <12> <14.4> <17.28> <20.74> <24.88>
      mathx10
      }{}
\DeclareSymbolFont{mathx}{U}{mathx}{m}{n}
\DeclareFontSubstitution{U}{mathx}{m}{n}
\DeclareMathAccent{\widecheck}{0}{mathx}{"71}

\MakePerPage{footnote}

% xcolor
\definecolor{lightgrey}{gray}{0.90}
\definecolor{slightgrey}{gray}{0.95}

% hyperref
\hypersetup{
      colorlinks = true,
      linkcolor = {blue},
      citecolor = {blue}
}

% Boxes
\mdfdefinestyle{Definitions}{
    leftmargin=0cm,
    rightmargin=0cm,
    linecolor=gray!70,
    topline=false,
    bottomline=false,
    rightline=false,
    backgroundcolor=gray!4,
    footnoteinside=true}

% thmtool 

% % custom theoremstyles
\declaretheoremstyle[
spaceabove=10pt, spacebelow=10pt,
notebraces = {- }{},
headpunct = {.\vspace{1mm}\newline}
]{mydfn}

\declaretheoremstyle[
spaceabove=10pt, spacebelow=10pt,
notebraces = {- }{},
bodyfont = \itshape, % Italics body font
headpunct = {.\vspace{1mm}\newline}
]{mythm}

% % Theorems
\declaretheorem[
  name = Theorem,
  style = mythm, 
  refname = {theorem,theorems},
  Refname = {Theorem,Theorems},
  numbered = no,
  % shaded = {rulecolor = lightgrey, rulewidth = 1.5mm, 
  %   bgcolor = white, textwidth = 46em}
]{thm}
\declaretheorem[
  name = Lemma,
  style = mythm, 
  refname = {lemma,lemmas},
  numbered = no]{lem}
\declaretheorem[
  name = Proposition,
  style = mythm, 
  refname = {proposition,propositions},
  numbered = no]{prop}
\declaretheorem[
  name = Corollary,
  style = mythm, 
  refname = {corollary,corollaries},
  numbered = no]{cor}
\declaretheorem[
  name = Remark, 
  style = remark, 
  numbered = no
]{rmk}
\declaretheorem[
  style = mydfn, 
  name = Definition, 
  numbered = no, 
  shaded = {bgcolor = slightgrey, margin = 2mm, textwidth = 46em}
]{dfn}
\declaretheorem[
  name = Example, 
  style = remark, 
  numbered = no
]{eg}

% tikzcd
% % Substituting symbols for arrows in tikz comm-diagrams.
\tikzset{
  symbol/.style={
    draw=none,
    every to/.append style={
      edge node={node [sloped, allow upside down, auto=false]{$#1$}}}
  }
}

\renewcommand{\listtheoremname}{List of Definitions and Theorems}

\begin{document}
\title{Category Theory Intuitively}
\author{Ken Lee}
\date{Date}
\maketitle

\tableofcontents

Notations : 
\begin{enumerate}
  \item For a collection of sets $U_i$ indexed by a set $I$, 
  $\Pi i \in I, U_i$ denotes $\prod_{i \in I} U_i$.
\end{enumerate}

\section{Categories}
\begin{dfn}[Categories]\link{cat}
  
  A \emph{category $\CC$} is defined by the following data : 
  \begin{enumerate}
    \item A set of \emph{objects}, $\obj{\CC}$. 
    \item For every $U, V \in \obj{\CC}$, 
    a set of \emph{$\CC$-morphisms} from $U$ to $V$,
    denoted $\mor{U}{V}{\CC}$.
    We denote $f : \map{U}{V}{\CC}{}$ for $f \in \mor{U}{V}{\CC}$. 
    \item For every $U, V, W \in \obj{\CC}$, 
    $f : \map{U}{V}{\CC}{}$ and $g : \map{V}{W}{\CC}{}$, 
    a $\CC$-morphism called the \emph{composition of $f$ with $g$},
    denoted $g \circ f : \map{U}{W}{\CC}{}$.
    \item Associativity of $\circ$.
    \item For every $U \in \obj{\CC}$, 
    an \emph{identity morphism} $\id{U} : \map{U}{U}{\CC}{}$.
    \item For all $U, V, W \in \obj{\CC}$, 
    $f : \map{U}{V}{\CC}{}$ and $g : \map{W}{U}{\CC}{}$, 
    we have $f \circ \id{U} = f$ and $\id{U} \circ g = g$. 
  \end{enumerate}
\end{dfn}

\begin{rmk}
Morphisms in a category do \emph{not} have to be functions.
See the example of preordered sets as categories at
\linkto{eg:cat_ord}{end of this section}.
\end{rmk}

\begin{dfn}[Isomorphisms]\link{iso}
  
  Let $\CC$ be a category, $U, V \in \obj{\CC}$, $f : \map{U}{V}{\CC}{}$.
  Then $f$ is called an \emph{isomorphism} when 
  there exists $g : \map{V}{U}{\CC}{}$ such that 
  $g \circ f = \id{U}$ and $f \circ g = \id{V}$.
  In this case, we denote $f : \map{U}{V}{\CC}{\sim}$.
  When there exists an isomorphism from $U$ to $V$, 
  we say they are \emph{isomorphic} and write $U \iso V$. 
\end{dfn}

\begin{dfn}[Subcategories]\link{subcat}
  
  Let $\CC,\DD$ be categories. 
  Then $\DD$ is called a \emph{subcategory of $\CC$} when 
  $\obj{\DD} \subs \obj{\CC}$ and 
  for all $U,V \in \obj{\DD}$, $\mor{U}{V}{\DD} \subs \mor{U}{V}{\CC}$.
\end{dfn}

\begin{eg}[Standard Categories]\link{eg:cat}~
  \begin{enumerate}
    \item $\SET$ denotes the \emph{category of sets}, where 
    $\obj{\SET}$ contains sets and for $U, V \in \obj{\SET}$, 
    $\SET(U,V)$ is the set of maps from $U$ to $V$. 
    \item $\TOP$ denotes the \emph{category of topological spaces}, where
    $\obj{\TOP}$ contains topological spaces and for $U, V \in \obj{\TOP}$, 
    $\TOP(U,V)$ is the set of continuous maps from $U$ to $V$. 
    $\TOP$ is a subcategory of $\SET$. 
    \item The \emph{category of groups} $\GRP$ has 
    $\obj{\GRP}$ containing groups and 
    $\GRP(U,V)$ containing group homomorphisms from $U$ to $V$. 
    $\GRP$ is a subcategory of $\SET$.  
    \item The \emph{category of abelian groups} $\AB$ has 
    $\obj{\AB}$ containing abelian groups and 
    $\AB(U,V)$ containing group homomorphisms from $U$ to $V$. 
    $\AB$ is a subcategory of $\GRP$. 
    \item The \emph{category of rings} $\RING$ has 
    $\obj{\RING}$ containing rings and 
    $\RING(U,V)$ containing ring homomorphisms from $U$ to $V$.
    $\RING$ is a subcategory of $\SET$. 
    \item The \emph{category of commutative rings} $\CRING$ has 
    $\obj{\CRING}$ containing commutative rings and 
    $\CRING(U,V)$ containing ring homomorphisms from $U$ to $V$.
    $\CRING$ is a subcategory of $\RING$. 
    \item Let $R$ be a commutative ring. 
    Then the \emph{category of $R$-modules} $\MOD$ has 
    $\obj{\MOD}$ containing $R$-modules and 
    $\MOD(U,V)$ contains $R$-linear maps from $U$ to $V$.
    This is a subcategory of $\AB$. 
    \item Let $R$ be a commutative ring. 
    Then the \emph{category of $R$-algebras} $\ALG$ has 
    $\obj{\ALG}$ containing pairs $(S,\si)$ where $\si : \map{R}{S}{\CRING}{}$.
    $\ALG((U,u),(V,v))$ contains $f : \map{U}{V}{\CRING}{}$ such that 
    $f \circ u = v$. 
  \end{enumerate}
\end{eg}

\begin{eg}[Preordered Sets as Categories]\link{eg:cat_ord}
  
  Let $I$ be a set, $\leq$ a relation on $I$. 
  Then $(I,\leq)$ is called a \emph{preordered set} when 
  $\leq$ satisfies all of the following : 
  \begin{enumerate}
    \item (Reflexivity) For all $i \in I$, $i \leq i$. 
    \item (Transitivity) For all $i, j, k \in I$, 
    $i \leq j$ and $j \leq k$ implies $i \leq k$. 
  \end{enumerate}
  If $(I,\leq)$ is a preordered set where $\leq$ is clear, 
  we abbreviate to $I$.

  Let $I$ be a preordered set. 
  Then we can turn $I$ into a category as follows : 
  \begin{enumerate}
    \item $\obj{I}$ is $I$. 
    \item For $i, j \in \obj{I}$, $I(i,j)$ is 
    singleton when $i \leq j$ and empty otherwise.  
  \end{enumerate}

  Things get meta. 
  We can form the \emph{category of preordered sets} $\ORD$ where 
  $\obj{\ORD}$ contains preoredered sets and 
  $\ORD(I,J)$ contains $f : \map{I}{J}{\SET}{}$ such that 
  for all $i, j \in I$, $i \leq j$ implies $f(i) \leq f(j)$.
\end{eg}

\begin{eg}[Category of Partially Ordered Sets]\link{eg:cat_poset}
  
  Let $I \in \obj{\ORD}$. 
  Then $I$ is called a \emph{partially ordered set} when 
  $\leq$ is \emph{antisymmetric}, i.e.
  for all $i, j \in I$, $i \leq j$ and $j \leq i$ implies $i = j$. 
  We thus have the \emph{category of partially ordered sets} $\POSET$ where 
  $\obj{\POSET}$ contains partially ordered sets and 
  $\POSET(I,J) = \ORD(I,J)$. 
  We see that $\POSET$ is a subcategory of $\ORD$.
\end{eg}

\begin{eg}[Partially Ordered Sets]\link{eg:poset}~
  \begin{enumerate}
    \item Let $X$ be a set. Then its powerset $(2^X,\leq) \in \obj{\POSET}$. 
    \item Let $X$ be a topological space. 
    Then the set of its open sets $\mathrm{Open}\,X$ is a partially ordered set.
    \item Let $G$ be a group. 
    Then the set of its subgroups is in $\obj{\POSET}$.
    \item Let $R$ be a commutative ring and $M$ be an $R$-module.
    Then the set of $R$-submodules of $M$ is in $\obj{\POSET}$.
    \item Let $R$ be a commutative ring and $(S,\si)$ an $R$-algebra. 
    Then the set of all $R$-subalgebras of $S$ is in $\obj{\POSET}$. 
    \item Consider the relation on $\N$ that is $a \dvd b$. 
    This is a partial order on $\N$. 
  \end{enumerate}
\end{eg}

\begin{eg}[A Group as a Category]\link{eg:cat_group}
  
  A group $G$ is equivalent to 
  a category $G$ where there is only one object $\bullet$ and 
  all morphisms are isomorphisms. 

  A direct generalization is a \emph{groupoid} : 
  a category where every morphism is an isomorphism. 
\end{eg}
\section{Functors}
\begin{dfn}[Functors]\link{functors}
  
  Let $\CC, \DD$ be categories. 
  Then a \emph{functor $F$ from $\CC$ to $\DD$} is defined by 
  the following data :
  \begin{enumerate}
    \item A map of objects $\obj{\CC} \to \obj{\DD}$,
    which we will denote by the same name $F$. 
    \item A map of morphisms for all $U, V \in \obj{\CC}$, 
    $\mor{U}{V}{\CC} \to \mor{F(U)}{F(V)}{\DD}$,
    which we will also denote by the same name $F$. 
    \item (Compositions are Preserved)
    For all $f : \map{U}{V}{\CC}{}$ and $g : \map{V}{W}{\CC}{}$, 
    $F(g \circ f) = F(g) \circ F(f)$. 
    \item (Identity Morphisms are Preserved)
    For all $U \in \obj{\CC}$, $F(\id{U}) = \id{F(U)}$.
  \end{enumerate}
\end{dfn}

\begin{dfn}[Category of Categories]\hypertarget{bigcat}{}
  
  We define the \emph{category of categories} $\CAT$, 
  \begin{enumerate}
    \item $\obj{\CAT}$ consists of categories. 
    \item For $\CC, \DD \in \obj{\CAT}$, 
    $\mor{\CC}{\DD}{\CAT}$ consists of functors from $\CC$ to $\DD$.  
    \item For $\CC \in \obj{\CAT}$, $\id{\CC}$ is the obvious thing.
  \end{enumerate}
\end{dfn}

\begin{dfn}[Faithful, Full, Fully Faithful]\link{fully_faithful}
  
  Let $F : \map{\CC}{\DD}{\CAT}{}$.
  Then $F$ is called 
  \begin{enumerate}
    \item \emph{faithful} when for all $U, V \in \obj{\CC}$,
    $F : \mor{U}{V}{\CC} \to \mor{F(U)}{F(V)}{\DD}$ is injective. 
    \item \emph{full} when for all $U, V \in \obj{\CC}$,
    $F : \mor{U}{V}{\CC} \to \mor{F(U)}{F(V)}{\DD}$ is surjective.
    \item \emph{fully faithful} when for all $U, V \in \obj{\CC}$,
    $F : \mor{U}{V}{\CC} \to \mor{F(U)}{F(V)}{\DD}$ is bijective.
  \end{enumerate}
\end{dfn}

\begin{prop}[Fully Faithful Functors are Injective]
  \link{full_faith_inj}
  
  Let $F : \map{\CC}{\DD}{\CAT}{}$ be fully faithful,
  $U, V \in \obj{\CC}$ such that $F(U) \iso F(V)$.
  Then $U \iso V$. 
\end{prop}
\begin{proof}
  Let $f_1 \in \DD(F(U),F(V))$ and $f_2 \in \DD(F(V),F(U))$ such that 
  $\id{F(U)} = f_2 \circ f_1$ and $\id{F(V)} = f_1 \circ f_2$. 
  Then $f_1, f_2$ corresponds respectively to $g_1, g_2 \in \CC(U,V), \CC(V,U)$
  through $F$. 
  We thus have 
  \[ 
    F(g_2 \circ g_1) = F(g_2) \circ F(g_1) = f_2 \circ f_1 = \id{F(U)}
    = F(\id{U})
  \]
  which by $F$ fully faithful gives $g_2 \circ g_1 = \id{U}$. 
  Similarly, $g_1 \circ g_2 = \id{V}$.
\end{proof}

\begin{dfn}[Natural Transformations]\link{natural}
  
  Let $F, G : \map{\CC}{\DD}{\CAT}{}$. 
  Then a \emph{natural transformation $\eta$ from $F$ to $G$} is defined by 
  the following data : 
  \begin{enumerate}
    \item For all $U \in \obj{\CC}$, $\eta_U : \map{F(U)}{G(U)}{\DD}{}$. 
    \item (Naturality) 
    For all $U, V \in \obj{\CC}$ and $f : \map{U}{V}{\CC}{}$, 
    we have the following commutative diagram. \begin{figure}[H]
      \centering
      \begin{tikzcd}
        F(U) \arrow[r,"\eta_U"] \arrow[d,"F(f)"{swap}] & G(U) \arrow[d,"G(f)"] \\
        F(V) \arrow[r,"\eta_V"{swap}] & G(V) 
      \end{tikzcd}
    \end{figure}
  \end{enumerate}
\end{dfn}

\begin{dfn}[Category of Functors]\link{cat_functor}
  
  Let $\CC, \DD \in \obj{\CAT}$. 
  Then the \emph{category of functors from $\CC$ to $\DD$},
  denoted $\DD^\CC$, is defined by 
  \begin{enumerate}
    \item $\obj{\DD^\CC} := \mor{\CC}{\DD}{\CAT}$. 
    \item For all $F, G \in \obj{\DD^\CC}$, 
    $\mor{F}{G}{\DD^\CC} := $
    the set of natural transformations from $F$ to $G$. 
    \item For all $F \in \obj{\DD^\CC}$, $\id{F}$ is the obvious thing.
    \item The obvious way to define composition of natural tranformations is ``component-wise''.
  \end{enumerate}
\end{dfn}

\begin{dfn}[Equivalence of Categories]\link{equiv}
  
  Let $\CC, \DD$ be categories, $F \in \CAT(\CC,\DD)$.
  Then $F$ is called an \emph{equivalence of categories} when 
  there exists $G \in \CAT(\DD,\CC)$ such that  
  $G \circ F \iso \id{\CC}$ and $F\circ G \iso \id{\DD}$. 
\end{dfn}

\begin{dfn}[Essentially Surjective]\link{surj}
  
  Let $\CC,\DD$ be categories and $F : \map{\CC}{\DD}{\CAT}{}$. 
  The \emph{essential image of $F$} is defined as
  the set of $X \in \DD$ such that 
  there exists $U \in \CC$ where $F(U) \iso X$.
  $F$ is called \emph{essentially surjective} when
  its essential image is the whole of $\DD$.
\end{dfn}

\begin{prop}[Characterisation of Equivalence of Categories]\link{char_equiv}
  
  Let $F : \map{\CC}{\DD}{\CAT}{}$. 
  Then $F$ is an equivalence of categories if and only if 
  $F$ is fully faithful and essentially surjective. 
\end{prop}
\begin{proof}~
  $(\implies)$
  Let $G : \map{\DD}{\CC}{\CAT}{}$, $\ep : \map{\id{\CC}}{G \circ F}{}{\sim}$,
  $\eta : \map{F \circ G}{\id{\DD}}{}{\sim}$.
  It is clear that $F$ is essentially surjective. 
  For faithful, let $f, g \in \CC(U,V)$ such that 
  $F(f) = F(g)$.
  Then by naturality of $\ep$, we have the following commutative diagram : 
  \begin{cd}
    U \ar[r,"f"] \ar[d,"\ep_U"] & V \ar[d,"\ep_V"] \\
    GF(U) \ar[r,"GF(f)"] & GF(V)
  \end{cd}
  Then $f = g$ follows from $\ep_U, \ep_V$ being isomorphisms. 

  For fullness, let $f \in \DD(F(U), F(V))$.
  The guess is that by mapping $f$ back to $\CC$, 
  we get the morphism that maps to $f$.
  That is, we claim that $F(\ep_V\inv \circ G(f) \circ \ep_U) = f$.
  Since the arguments of the above paragraph also applies to $G$,
  we have $G$ is faithful and hence 
  it suffices to show that $GF(\ep_V\inv \circ G(f) \circ \ep_U) = G(f)$.
  We first show that perhaps unsurprisingly, 
  $GF(\ep_V\inv) = \ep_{GF(V)}\inv$. 
  By functoriality of $GF$, it suffices to show that
  $GF(\ep_V) = \ep_{GF(V)}$. 
  This follows from $\ep_V$ being an isomorphism and 
  the following commutative diagram due to
  the naturality of $\ep$ : 
  \begin{cd}
    V \ar[r,"\ep_V"] \ar[d,"\ep_V"]& GF(V) \ar[d,"GF(\ep_V)"] \\
    GF(V) \ar[r,"\ep_{GF(V)}"] & GFGF(V) 
  \end{cd}
  It now remains to show $GF(G(f) \circ \ep_U) = \ep_{GF(V)} \circ G(f)$.
  This follows from naturality of $\ep$ and $\ep_U$ being an isomorphism : 
  \begin{cd}
    U \ar[r,"G(f) \circ \ep_U"] \ar[d,"\ep_U"] & GF(V) \ar[d,"\ep_{GF(V)}"]
    \\
    GF(U) \ar[r,"GF(G(f) \circ \ep_U)"{swap,yshift = -1mm}] & GFGF(V)
  \end{cd}

  $(\limplies)$
  Using the axiom of choice, 
  for each $X \in \obj{\DD}$, let $G(X) \in \obj{\CC}$ and 
  $\eta_X \in \DD(FG(X), X)$ such that $\eta_X$ is an isomorphism. 
  For $f \in \DD(X, Y)$, 
  by full faithfulness of $F$ let $G(f) \in \CC(G(X),G(Y))$ be
  the unique morphism such that $FG(f) = \eta_Y\inv \circ f \eta_X$. 
  It then follows from uniqueness of the above morphisms and 
  functoriality of $F$ that 
  $G$ is a functor. 
  Note that by construction, 
  the collection of $\eta_X$ gives a natural isomorphism 
  $\eta : F \circ G \to \id{\DD}$.
  
  It remains to give a natural isomorphism $\ep : \id{\CC} \to G \circ F$.
  For $U \in \obj{\CC}$, we are looking for a morphism $U \to GF(U)$.
  Feeling optimistic, we use full and faithfulness of $F$ to define 
  $\ep_U \in \CC(U, GF(U))$ as the unique morphism such that 
  $F(\ep_U) = \eta_{F(U)} \inv$. 
  From $\eta_{F(U)}$ being an isomorphism and full faithfulness of $F$, 
  it follows that $\ep_U$ is also an isomorphism. 
  Finally, to check naturality of $\ep$, 
  let $f \in \CC(U,V)$.
  We need $\ep_V \circ f = GF(f) \circ \ep_U$.
  But since $F$ is faithful, 
  it suffices that $F$ applied to these morphisms are equal. 
  Well, indeed we have it \[
    F(\ep_V \circ f) = \eta_{F(V)}\inv \circ F(f)
    = \eta_{F(V)}\inv \circ F(f) \circ \eta_U \circ \eta_U\inv 
    = \eta_{F(V)}\inv \circ F(f) \circ \eta_U \circ F(\ep_U)
    = F(GF(f) \circ \ep_U)
  \]
\end{proof}

\begin{eg}[Functors and Natural Transformations]\link{eg:functors}
  
  The following is but a small sample of the vast sea of functors that
  appear in mathematics. 
  There is no need to ``memorize'' these.
  You will spot them when they appear. 

  The first list gives constructions of ``structures''
  between subcategories of sets. 
  The theme is that these are all 
  \linkto{free}{free functors} adjoint to some kind of forgetful functor.
  Details of this are explained in the section on adjunctions. 
  \begin{itemize}
    \item Forgetful functor
    Given any subcategory $\CC$ of $\DD$, 
    there is an ``obvious'' functor from $\CC$ to $\DD$ that maps 
    $\obj{\CC} \to \obj{\DD}$ by doing nothing and 
    morphisms in $\CC$ to morphisms in $\DD$ by doing nothing. 
    Functors of this form are often called 
    the \emph{forgetful functor}.

    Here is a graph showing subcategories of set and their ``inclusions''.
    \begin{cd}
      \SET & 
      \GRP \ar[l] & 
      \AB \ar[l] & 
      \MOD(R) \ar[l] \\
      \TOP \ar[u] & 
      \RING \ar[u] \ar[ur] & 
      \CRING \ar[u] \ar[l] & 
      \ALG(R) \ar[l] \ar[u]
    \end{cd}
    in the diagram, $R$ is a commutative ring with unity. 
    The maps from $\RING, \CRING$ into $\AB$ take 
    (commutative) rings to their underlying abelian groups. 

    \item (Free Group)
    For each set $S$, the free group over $S$ is an object $\<S\> \in \GRP$
    the comes with a morphism of sets $\lift{}{} : S \to \<S\>$ such that
    for any group $G$ and $\phi \in \SET(S,G)$,
    there is a unique morphism of groups $\<\phi\> \in \GRP(\<S\>,G)$ such that
    $\<\phi\> \circ \lift{}{} = \phi$.
    This makes $G \mapsto \<G\>$ into a functor from $\SET$ to $\GRP$.

    \item (Free Module over a Ring)
    Let $A$ be a commutative ring. 
    For each set $S$, 
    the free $A$-module over $S$ is an object $A^{\oplus S} \in \MOD(A)$ 
    that comes with a morphism of sets $\lift{}{} : S \to A^{\oplus S}$
    such that for any $A$-module $M$ and $f \in \SET(S,M)$,
    there is a unique $A$-linear map 
    $A^{\oplus f} : \oplus_{s \in S} A \to M$ such that 
    $A^{\oplus f} \circ \lift{}{} = f$. 
    This makes $S \mapsto A^{\oplus S}$ into a functor 
    from $\SET$ to $\MOD(A)$.

    In particular, for a fixed $M \in \MOD(A)$ and $S \subs M$,
    $S$ is called respectively linearly independent, spanning, a basis
    if and only if $A^{\oplus f}$ is injective, surjective, an isomorphism.

    Note that the above covers $\AB$, since 
    $\AB$ is nothing more than $\MOD(\Z)$.

    \item (Free Algebra over a Ring)
    Let $K$ be a commutative ring. 
    For each set $S$, 
    the free $K$-algebra over $S$ is an object $K[S] \in \ALG(K)$
    that comes with a morphism of sets $\lift{}{} : S \to K[S]$ such that
    for any $K$-algebra $A$ and $a \in \SET(S,A)$,
    there exists a unique $K$-algebra morphism $ev_a : K[S] \to A$
    such that $ev_a \circ \lift{}{} = a$.
    This makes $S \mapsto K[S]$ into a functor 
    from $\SET$ to $\ALG(K)$.

    These free algebras are not unfamiliar.
    For instance, 
    the polynomial ring in $K[T]$ over $K$ is
    precisely $K[\set{*}]$ where $\set{*}$ is the singleton set.
    For any $K$-algebra $A$,
    a set morphism $a : \set{*} \to A$ is nothing more than
    an element in $A$.
    So as suggested by the notation, $ev_a$ is precisely evaluation 
    of polynomials $f \mapsto f(a)$ where we have identified 
    the set morphism $a$ with the unique element in its image. 
    Generalizing, for an arbitrary set $S$,
    $K[S]$ is precisely the $K$-algebra of polynomials 
    with variables indexed by $S$. 
    In particular, for a $K$-algebra $A$ and $S \subs A$,
    $S$ is called respectively algebraically independent over $A$, generating
    when $ev_S : K[S] \to A$ is injective, surjective. 

    \item (Tensor Product, Extension and Contraction of Scalars)
    Let $B$ be an $A$-algebra where $A$ is a commutative ring. 
    Every $B$-module $N$ already has an $A$-module structure. 
    This gives a forgetful functor from $\MOD(B)$ to $\MOD(A)$.

    ``Conversely'', for any $A$-module $M$, 
    $B \otimes_A M$ has an obvious $B$-module structure.
    Then for any $A$-linear map $f \in \MOD(A)(M,N)$, 
    $\id{B} \otimes_A f \in \MOD(B)(B\otimes_A M, B\otimes_A N)$.
    This makes $B \otimes_A (-)$ into a functor from 
    $\MOD(A)$ to $\MOD(B)$.
    In analogy with the prior examples,
    extension of scalars can be seen as 
    ``taking the free $B$-module over an $A$-module''.

    \item (Localization of Modules)
    Let $A$ be a commutative ring and $S \subs A$ multiplicative. 
    Define the category $\MOD(A_S)$ as the full subcategory of $\MOD(A)$ 
    with objects consisting of $M \in \obj{\MOD(A)}$ such that 
    for all $f \in S$, scalar multiplication by $f$ on $M$ is an isomorphism,
    i.e. $f$ is an ``invertible'' scalar for $M$.
    There is an obvious forgetful functor from $\MOD(A_S)$ to $\MOD(A)$.

    ``Conversely'', for an $A$-module $M$, 
    the localization of $M$ with respect to $S$ is 
    is an object $M_S$ of $\MOD(A_S)$ that comes with 
    an $A$-linear map $\lift{}{} : M \to M_S$ such that 
    for all $N \in \MOD(A_S)$ and $f \in \MOD(A)(M,N)$,
    there is a unique $f_S \in \MOD(A_S)(M_S,N)$ where 
    $f_S \circ \lift{}{} = f$.
    This gives a functor $\MOD(A) \to \MOD(A_S), M \mapsto M_S$
    and morphisms are mapped to induced morphisms. 
    In particular, the localization $A_S$ of $A$ itself 
    has an obvious ring structure.
    This realizes $\MOD(A_S)$ as the category of modules over $A_S$.

    \item (Group Algebra)
    The following is similar to the free algebra construction. 
    Let $K$ be a commutative ring.
    Then for $A \in \ALG(K)$, $A^\times \in \AB$.
    Any $f \in \ALG(K)(A,B)$, 
    let $f^\times$ denote the restriction of $f$ onto $A^\times$.
    Then $f^\times$ is automatically a morphism of abelian groups
    (keeping in mind the group operation is multiplication).
    This gives a ``forgetful functor'' from $\ALG(K)$ to $\AB$.
    
    ``Conversely'',
    for any abelian group $G$,
    the group $K$-algebra over $G$ is a $K$-algebra $K[G]$ that comes with
    a morphism of abelian groups $\lift{}{} : G \to K[G]^\times$ such that
    for any other $K$-algebra $A$ and $\phi \in \AB(G,A^\times)$,
    there exists a unique $K[\phi] \in \ALG(K)(K[G],A)$ such that 
    $K[\phi] \circ \lift{}{} = \phi$. 
    This property makes $G \mapsto K[G]$ into a functor from 
    $\AB$ to $\ALG(K)$.
    In analogy to the prior examples,
    this may be seen as taking the ``free $K$-algebra on $G$''.
    In particular, $K[\Z]$ is precisely the localization $K[T,T\inv]$. 

    \item (Symmetric Algebra)
    The following is similar to the group algebra construction.
    Let $A$ be a commutative ring.
    Then for any $A$-module $M$,
    the symmetric algebra $Symm\,M$ is an object in $\ALG(A)$ 
    that comes with an $A$-linear map $\lift{}{} : M \to Symm\,M$
    such that for any $A$-algebra $B$ and $\phi \in \MOD(A)(M,B)$,
    there exists a unique $A$-algebra morphism $Symm\,\phi : Symm\,M \to B$
    such that $Symm\,\phi \circ \lift{}{} = \phi$.
    In analogy with the prior examples,
    this may be seen as taking the ``free $A$-algebra over $M$''.

    \item (Discrete Topology)
    For any set $X$, $(X,2^X)$ where $2^X$ is the powerset of $X$
    is a topological space. 
    Then for any topological space $Y$ and $f \in \SET(X,Y)$,
    $f$ is automatically continuous with respect to the discrete topology $2^X$.
    This gives rise to a functor $\SET \to \TOP$.
    In analogy with the prior examples, 
    this seen as taking the ``free topological space on $X$''.

  \end{itemize}

  The next list is themed ``moving structures on objects across morphisms''.
  \begin{itemize}
    \item Image, Preimage of subsets
    \item Image Preimage of subgroups
    \item Image Preimage of Subrings
    \item Image Preimage of Submodules
    \item Image, Preimage of filters
  \end{itemize}

  This list contains more exotic ``algebraic constructions''.
  \begin{itemize}
    \item Fundamental groups
    \item Singular Complex
    \item Classical Galois Correspondence
    \item Vanishing, Ideal
    \item Spec of a commutative ring
  \end{itemize}

  The final list consists of miscellaneous ``algebraic constructions'' :
  \begin{itemize}
    \item (Vector Spaces with a Basis)
    Let $K$ be a field. 
    Define a category $\CC$ as follows : 
    \begin{itemize}
      \item objects are pairs $(V,B)$ where $V$ is a $K$-vector space and 
      $B$ is a basis of $V$. 
      \item For $(V,B_V), (W,B_W)$ objects in $\CC$, 
      define $\CC((V,B_V),(W,B_W))$ as the set of $K$-linear maps 
      from $V$ to $W$ such that maps $B_V$ into $B_W$.
      \item For every $K$-vector space with a basis $(V,B_V)$, 
      $\id{(V,B_V)}$ is defined to be the identity map of $V$. 
      \item Composition of the underlying $K$-linear maps of morphisms yields
      another morphism in this category. 
    \end{itemize}
    Define the functor $F : \CC \to \CC$ that ``takes components'' as follows : 
    \begin{itemize}
      \item For $(V,B_V)$ in $\CC$, 
      let $F((V,B_V)) := (K^{\oplus B_V}, E)$ where 
      $K^{\oplus B_V}$ is the free $K$-vector space on $B_V$ and 
      $E$ is the standard basis.
      \item For a morphism $f \in \CC((V,B_V),(W,B_W))$,
      since $f B_V \subs B_W$, 
      this determines a map from the standard basis of $K^{\oplus B_V}$ to
      the standard basis of $K^{\oplus B_W}$, 
      thus extending to a unique 
      $K$-linear map $F(f) : K^{\oplus B_V} \to K^{\oplus B_W}$.
      \item Identity morphisms are clearly respected.
      \item Composition of morphisms are clearly respected. 
    \end{itemize}
    There is a natural isomorphism between 
    ``taking components'' and the identity functor : 
    For each $(V,B_V)$ in $\CC$, 
    consider the $K$-linear map $[-]_{B_V} : V \to K^{\oplus B_V}$ that 
    that takes vectors to their components with respect to $B_V$.
    This is well-defined and an isomorphism by $B_V$ being a basis. 
    (In fact, this can serve as a \emph{definition} of $B_V$ being a basis.)
    Then we have naturality : 
    \begin{cd}
      (V,B_V) \ar[r,"f"] \ar[d,"\sqbrkt{-}_{B_V}"{swap},"\sim"] &
      (W,B_W) \ar[d,"\sqbrkt{-}_{B_W}","\sim"{swap}] \\
      (K^{\oplus B_V},E) \ar[r,"F(f)"] &
      (K^{\oplus B_W},E) \\
    \end{cd}
    In particular for a fixed $K$-vector space $V$ and 
    two finite bases $B,B_1$,
    any total ordering on $B, B_1$ gives rise to a unique 
    $f \in \CC((V,B),(V,B_1))$.
    Then the (iso)morphism $F(f)$ is 
    what is usually known as \emph{change of basis}.

    \item  Dual Module
  
    \item (Power Set as an $\F_2$-Algebra)
    For any set $X$, we can see the power set $2^X$ as $\F_2^X$
    the set of set morphisms from $X$ to the field with two elements $\F_2$.
    Then $\F_2^X$ naturally has a structure of an $\F_2$-algebra.
    Explicitly, for two subsets $f, g \in \F_2^X$,
    $f g = f \cap g$ and $f + g = (f \cup g) \minus (f \cap g)$.
    The additive identity is $\nothing$ and the multiplicative identity is $X$.
    One can see that preimage functor $X \mapsto 2^X = \F_2^X$
    upgrades to a contravariant functor from $\SET$ to $\ALG(\F_2)$.

    \item  Tangent space of pointed differentiable manifold
  \end{itemize}
\end{eg}
\section{Universal Morphisms}
\begin{dfn}[Comma Category]\hypertarget{comma}{}
  
  Let $G : \map{\CC}{\DD}{\CAT}{}$ and $X \in \obj{\DD}$.
  Then the \emph{comma category $X\darrow G$} is defined as follows. 
  \begin{enumerate}
    \item $\obj{X\darrow G}$ consists of pairs $(U,u)$
    where $U \in \obj{\CC}$ and $u : \map{X}{G(U)}{\DD}{}$.
    \item For $(U,u), (V,v) \in \obj{X\darrow G}$, 
    $\mor{(U,u)}{(V,v)}{X\darrow G}$ consists of $f : \map{U}{V}{\CC}{}$
    such that \begin{figure}[H]
      \centering
      \begin{tikzcd}
        X \ar[r,"u"] \ar[rd,"v"{swap}] & G(U) \ar[d,"G(f)"] \\
        & G(V)
      \end{tikzcd}
    \end{figure}
  \end{enumerate}

  Dually, let $F : \map{\DD}{\CC}{\CAT}{}$ and $U \in \obj{\CC}$.
  Then the \emph{comma category $F\darrow U$} is defined as follows. 
  \begin{enumerate}
    \item $\obj{F\darrow U}$ consists of pairs $(X,x)$
    where $X \in \obj{\DD}$ and $x : \map{F(X)}{U}{\CC}{}$.
    \item For $(X,x), (Y,y) \in \obj{F\darrow U}$, 
    $\mor{(X,x)}{(Y,y)}{X\darrow F}$ consists of $g : \map{X}{Y}{\DD}{}$
    such that \begin{figure}[H]
      \centering
      \begin{tikzcd}
        F(X) \ar[rd,"x"] \ar[d,"F(g)"{swap}]& \\
        F(Y) \ar[r,"y"{swap}] & U
      \end{tikzcd}
    \end{figure}
  \end{enumerate}
\end{dfn}

\begin{rmk}
  Here is a special case of the comma category worth noting. 
\end{rmk}

\begin{dfn}[Over Category]\link{over_cat}
  
  Let $\CC$ be a category and $U \in \obj{\CC}$.
  Then the \emph{over category $\CC\darrow U$} is defined as 
  $\id{\CC}\darrow U$. 

  Dually, the \emph{under category $U \darrow \CC$} is defined as 
  $U\darrow \id{\CC}$.
\end{dfn}

\begin{eg}[Over and Under Categories]\link{eg:over_cat}
  \begin{enumerate}
    \item Let $R\in\obj{\RING}$. Then $\ALG(R) = R\darrow\RING$.
    \item Let $X\in\obj{\TOP}$. Then we have the category of covering spaces 
    of $X$ which is the subcategory of $\TOP\darrow X$ where 
    objects are $(\tilde{X},p)$ with $p$ a covering map.
  \end{enumerate}
\end{eg}

\begin{dfn}[Universal Morphism]\hypertarget{universal}{}
  
  Let $G : \map{\CC}{\DD}{\CAT}{}$ and $X \in \obj{\DD}$.
  Then a \emph{universal morphism from $X$ to $G$} is the following data. 
  \begin{enumerate}
    \item An object $(F(X),\eta_X)$ of the comma category $X\darrow G$.
    \item (Universal Property) For all $(V,v) \in \obj{X\darrow G}$,
    there exists a unique morphism $\map{(F(X),\eta_X)}{(V,v)}{X\darrow G}{}$.
  \end{enumerate}

  Dually, let $F : \map{\DD}{\CC}{\CAT}{}$ and $U \in \obj{\CC}$.
  Then a \emph{universal morphism from $F$ to $U$} is the following data. 
  \begin{enumerate}
    \item An object $(G(U),\ep_U)$ of the comma category $F\darrow U$.
    \item (Universal Property)For all $(Y,y) \in \obj{F\darrow U}$,
    there exists a unique morphism $\map{(Y,y)}{(G(U),\ep_U)}{F\darrow U}{}$.
  \end{enumerate}
\end{dfn}

\begin{prop}[Unique up to Unique Isomorphism]\hypertarget{uniprop}{}
  
  Let $G : \map{\CC}{\DD}{\CAT}{}$, $X \in \obj{\DD}$, 
  $(U,u), (V,v) \in X\darrow G$ both universal morphisms from $X$ to $G$. 
  Then there exist unique $f : \map{(U,u)}{(V,v)}{X\darrow G}{}$
  and $g : \map{(V,v)}{(U,u)}{X\darrow G}{}$ such that 
  $g \circ f = \id{(U,u)}$ and $f \circ g = \id{(V,v)}$.
  Thus, if a universal morphism exists, 
  we say it is \emph{unique up to unique isomorphism}.

  Dually, let $F : \map{\DD}{\CC}{\CAT}{}$, $U \in \obj{\CC}$,
  $(X,x), (Y,y) \in F\darrow U$ both universal morphisms from $F$ to $U$.
  Then there exists a unique $f : \map{(X,x)}{(Y,y)}{F\darrow U}{}$
  and $g : \map{(Y,y)}{(X,x)}{F\darrow u}{}$ such that 
  $g \circ f = \id{(X,x)}$ and $f \circ g = \id{(Y,y)}$.
\end{prop}
\begin{proof}(Shorter proof that does not go through Yoneda).

  By the universal property of $(U,u)$,
  There exists a unique $f : \map{(U,u)}{(V,v)}{X\darrow G}{}$.
  Similarly, there exists a unique $g : \map{(Y,y)}{(X,x)}{F\darrow u}{}$.
  But then $g \circ f : \map{(U,u)}{(U,u)}{X\darrow G}{}$.
  By applying the universal property of $(U,u)$ with itself,
  we see that $\id{(U,u)}$ is the unique $\map{(U,u)}{(U,u)}{X\darrow G}{}$.
  In particular, we have $g \circ f = \id{(U,u)}$.
  Similarly, we have $f \circ g = \id{(V,v)}$.
  Since $f$ and $g$ are the \emph{only} morphisms between $(U,u)$, $(V,v)$,
  they are \emph{the} unique isomorphism between $(U,u)$ and $(V,v)$.
\end{proof}

\begin{rmk}[``Canonically Isomorphic'']
  \hypertarget{canonical}{}

  It is common in category theory and maths at large to \emph{equate} 
  two objects that satisfy the same universal property, 
  since they are not only isomorphic, but also isomorphic in a unique way. 
  Some also call these \emph{canonically isomorphic}.
\end{rmk}

\begin{prop}[Isomorphic to Universal implies Universal]
  \link{iso_uni_implies_uni}
  
  Let $G : \map{\CC}{\DD}{\CAT}{}$, $X \in \obj{\DD}$, 
  $(U,u), (V,v) \in X\darrow G$ where 
  $(U,u) \iso[X\darrow G] (V,v)$ and 
  $(U,u)$ is a universal morphism from $X$ to $G$.
  Then $(V,v)$ is a universal morphism from $X$ to $G$.

  Dually, let $F : \map{\DD}{\CC}{\CAT}{}$, $U \in \obj{\CC}$,
  $(X,x), (Y,y) \in F\darrow U$ where $(X,x) \iso[F\darrow U] (Y,y)$ and 
  $(X,x)$ is a universal morphism from $F$ to $U$. 
  Then $(Y,y)$ is a universal morphism from $F$ to $U$. 
\end{prop}
\begin{proof}
  Let $f : \map{(U,u)}{(V,v)}{X\darrow G}{\sim}$.
  Let $(W,w)\in\obj{X\darrow G}$. 
  Then $f$ induces a bijection between $\mor{(U,u)}{(W,w)}{X\darrow G}$
  and $\mor{(V,v)}{(W,w)}{X\darrow G}$. 
  Since the former is singleton, so is the latter. 

  The dual has a similar argument. 
\end{proof}

% \begin{proof}
%  By \hyperlink{uni_iff_rep}{universal iff represents},
%  let $\al : \map{h^U}{\mor{X}{G(\star)}{\DD}}{\SET^\CC}{\sim}$
%  and $\be : \map{h^V}{\mor{X}{G(\star)}{\DD}}{\SET^\CC}{\sim}$ 
%  with $\al_{U}(\id{U}) = u$ and $\be_V(\id{V}) = v$.
%  Then $\be\inv \circ \al$ and $\al\inv \circ \be$ are 
%  natural isomorphisms between $h^U$ and $h^V$.
%  So by \hyperlink{yoneda}{Yoneda's lemma}, 
%  there exists unique $f : \map{U}{V}{\CC}{}$ and $g : \map{V}{U}{\CC}{}$
%  such that the following diagram commutes. 
%  \begin{figure}[H]
%    \centering
%    \begin{tikzcd}[sep = large]
%      \mor{X}{G(\star)}{\DD} 
%      & h^V \ar[l,"\be"{swap},"\sim"] 
%        \ar["h^f",ld,rightharpoonup,xshift=1mm,yshift=-1mm] \\
%      h^U \ar[u,"\al","\sim"{swap}] 
%        \ar["h^g",ru,rightharpoonup] 
%      &  
%    \end{tikzcd}
%  \end{figure}
%  Since $h^{\id{U}} = h^f \circ h^g = h^{g \circ f}$,
%  applying \hyperlink{yoneda}{Yoneda's lemma} again, 
%  we obtain $g \circ f = \id{U}$, and thus $g \circ f = \id{(U,u)}$.
%  Similarly, we have $f \circ g = \id{(V,v)}$.
%  We claim that for any 
%  $f : \map{U}{V}{\CC}{}$ and $g : \map{V}{U}{\CC}{}$,
%  they make the above diagram commute if and only if 
%  $f : \map{(U,u)}{(V,v)}{X\darrow G}{}$ and 
%  $g : \map{(V,v)}{(U,u)}{X\darrow G}{}$.
%  This shows that $f, g$ are as desired. 
%
%  Since the arguments are analogous, 
%  we just prove it for $f$.
%  If the above diagram commutes for $f$,
%  then $v = \be_V(\id{V}) = \al_V \circ \brkt{h^f}_V (\id{V})
%  = \al_V(f) = G(f) \circ u$ as desired. 
%  Now suppose $v = G(f) \circ u$.
%  To show $\al \circ h^f = \be$, 
%  let $W \in \obj{\CC}$ and $g \in h^V(W)$.
%  Consider the following diagram. 
%  \begin{figure}[H]
%    \centering
%    \begin{tikzcd}[sep = huge]
%      \mor{X}{G(W)}{\DD}
%      & h^V(W) \ar["\be_W"{swap},l] \ar["\brkt{h^f}_W"{near end},ld]
%      & \\
%      h^U(W) \ar["\al_W",u]
%      & \mor{X}{G(V)}{\DD} \ar["\mor{X}{G(g)}{\DD}"{swap,near start},lu] 
%      & h^V(V) \ar["h^V(g)"{swap},lu] \ar["\be_V"{swap},l] 
%        \ar["\brkt{h^f}_V",ld] \\
%      & h^U(V) \ar["h^U(g)",lu] \ar["\al_V",u]
%      & 
%    \end{tikzcd}
%  \end{figure}
%  Everything except the triangles commute. 
%  From this, we compute \begin{align*}
%    \be_W(g) 
%    &= \be_W \circ h^V(g)\brkt{\id{V}}
%      = \mor{X}{G(g)}{\DD} \circ \be_V \brkt{\id{V}} 
%      = \mor{X}{G(g)}{\DD}\brkt{v} \\
%    &= \mor{X}{G(g)}{\DD} \brkt{G(f) \circ u}
%      = \mor{X}{G(g)}{\DD} \circ \al_V(f) 
%      = \mor{X}{G(g)}{\DD} \circ \al_V \circ \brkt{h^f}_V \brkt{\id{V}} \\
%    &= \al_W \circ h^U(g) \circ \brkt{h^f}_V \brkt{\id{V}}
%      = \al_W \circ \brkt{h^f}_W \circ h^F(g) \brkt{\id{V}} 
%      = \al_W \circ \brkt{h^f}_W \brkt{g}
%  \end{align*}
%  Thus $\be_W = \al_W \circ \brkt{h^f}_W$,
%  and hence $\be = \al \circ \brkt{h^f}$.
%  This concludes the proof.
% \end{proof}
\section{Yoneda's Lemma}
\begin{dfn}[Dual Categories]\hypertarget{dual}{}
  
  Let $\CC \in \obj{\CAT}$. 
  Then the \emph{dual category of $\CC$}, denoted $\CC\op$,
  is defined by : 
  \begin{enumerate}
    \item $\obj{\CC\op} := \obj{\CC}$. 
    \item For all $U, V \in \obj{\CC\op}$,
    $\mor{U}{V}{\CC\op} := \mor{V}{U}{\CC}$.
  \end{enumerate}
\end{dfn}

\begin{dfn}[Contravariant Functors]\hypertarget{contravar}{}
  
  Let $\CC, \DD \in \obj{\CAT}$. 
  Then a \emph{contravariant functor from $\CC$ to $\DD$} is 
  just a functor $\map{\CC\op}{\DD}{\CAT}{}$. 
  Functors $\map{\CC}{\DD}{\CAT}{}$ are henceforth called 
  \emph{covariant functors from $\CC$ to $\DD$}. 
\end{dfn}

% \begin{dfn}[Category of Presheaves on a Category]\hypertarget{presheaf}
%   
%   Let $\CC \in \obj{\CAT}$. 
%   Then the \emph{category of presheaves on $\CC$} is defined as 
%   the category of contravariant functors from $\CC$ to $\SET$. 
%   It is denoted $\PSH{\CC}$. 
% \end{dfn}

\begin{dfn}[Morphism Functor]\hypertarget{mor_funk}{}
  
  Let $\CC$ be a category and $U \in \obj{\CC}$. 
  Then $h_U : \map{\CC\op}{\SET}{\CAT}{}$ is defined as : 
  \begin{enumerate}
    \item For all $V \in \obj{\CC\op}$, 
    $h_U(V) := \mor{V}{U}{\CC}$. 
    \item For all $V, W \in \obj{\CC\op}$ and $f : \map{V}{W}{\CC\op}{}$, 
    $h_U(f) : \map{h_U(V)}{h_U(W)}{}{}, g \mapsto g \circ f$. 
  \end{enumerate}
  Similarly, $h^U : \map{\CC}{\SET}{\CAT}{}$ is defined as : 
  \begin{enumerate}
    \item For all $V \in \obj{\CC}$, 
    $h^U(V) := \mor{U}{V}{\CC}$. 
    \item For all $V, W \in \obj{\CC}$ and $f : \map{V}{W}{\CC}{}$, 
    $h^U(f) : \map{h^U(V)}{h^U(W)}{}{}, 
    g \mapsto f \circ g$. 
  \end{enumerate}
\end{dfn}

\begin{prop}[Morphism Functor is Functorial]\hypertarget{mor_funk_funk}{}
  
  Let $\CC$ be a category. 
  Then $h_{\star} : \map{\CC}{\SET^{\CC\op}}{\CAT}{}$. 
  Similarly, $h^\star : \map{\CC\op}{\SET^\CC}{\CAT}{}$. 
\end{prop}

\begin{rmk}[Functor of Points]\hypertarget{funk_pts}{}
  
  Because of its relevance in algebraic geometry, 
  $h_U$ is called the \emph{functor of points of $U$}. 
\end{rmk}

\begin{prop}[Yoneda's Lemma]\hypertarget{yoneda}{}
  
  Let $\CC$ be a category. 
  Then $h_{\star} : \map{\CC}{\SET^{\CC\op}}{\CAT}{}$ is fully faithful.
  Since \hyperlink{full_faith_inj}{fully faithful functors are injective}, 
  $h_\star$ is called the \emph{Yoneda embedding}. 

  More generally, for any $U \in \obj{\CC}$ and $F \in \obj{\SET^{\CC\op}}$,
  $\mor{h_U}{F}{\SET^{\CC\op}}$ bijects with $F(U)$ via 
  $s \mapsto s_U(\id{U})$ and this bijection is natural in 
  both $U$ and $F$.

  Dually, $h^\star : \map{\CC\op}{\SET^{\CC}}{\CAT}{}$ is fully faithful
  and more generally, 
  for any $U \in \obj{\CC\op}$ and $F \in \obj{\SET^\CC}$,
  $\mor{h^U}{F}{\SET^\CC}$ naturally bijects with $F(U)$ 
  via $s \mapsto s_U(\id{U})$.

\end{prop}
%\begin{proof}
%
%  We first prove the general statement.
%  Let $U \in \obj{\CC}$ and $F \in \obj{\SET^{\CC\op}}$.
%  To show injectivity, let $s, t \in \mor{h_U}{F}{\SET^{\CC\op}}$ and
%  assume $s_U(\id{U}) = t_U(\id{U})$. 
%  To show $s = t$, let $W \in \obj{\CC}$, $f \in h_U(W)$ and
%  consider the following commutative diagram, 
%  \begin{figure}[H]
%    \centering
%    \begin{tikzcd}
%      h_U(U) \arrow[rrr,"s_U"] \arrow[ddd,"h_U(f)"{swap}] 
%      & 
%      & 
%      & F(U) \arrow[ddd,"F(f)"] 
%      \\
%      & \id{U} \ar[r,mapsto] \ar[d,mapsto]& s_U(\id{U}) \ar[d,mapsto]& \\
%      & f \ar[r,mapsto] & s_W(f) = F(f)(s_U(\id{U})) & \\
%      h_U(W) \arrow[rrr,"s_W"{swap}] 
%      & 
%      & 
%      & F(W) 
%    \end{tikzcd}
%  \end{figure}
%  By considering an analogous for $t$,
%  we get $s_W(f) = t_W(f)$.
%  So $s_W = t_W$, and hence $s = t$. 
%  To show surjectivity, let $x \in F(U)$. 
%  Define $s \in \mor{h_U}{F}{\SET^{\CC\op}}$ by 
%  for all $V \in \obj{\CC}$, \[
%    s_V : f \in h_U(V) \mapsto F(f)(x) \in F(V)
%  \]
%  Then by the above diagram, $s_U(\id{U}) = x$. 
%  This proves the desired bijection.
%
%  For naturality in the first component,
%  let $f : \map{U}{V}{\CC}{}$.
%  Then we have the following commutative diagram. 
%  \begin{figure}[H]
%    \centering
%    \begin{tikzcd}
%      \mor{h_U}{F}{\SET^{\CC\op}} \ar[rrr]
%      & & & F(U) \\
%      & s\circ h^f \ar[r,mapsto]
%      & (s\circ h^f)_U (\id{U}) 
%        = s_V(f)
%        = F(f) \circ s_V (\id{V}) & \\
%      & s \ar[u,mapsto] \ar[r,mapsto] & s_V(\id{V}) \ar[u,mapsto] & \\
%      \mor{h_V}{F}{\SET^{\CC\op} } \ar[uuu,"(\star\circ h^f)"] \ar[rrr,""]
%      & & & F(V) \ar[uuu,"F(f)"{swap}]
%    \end{tikzcd}
%  \end{figure}
%  For naturality in the second component, 
%  let $\al : \map{F}{G}{\SET^{\CC\op}}{}$.
%  Then we have the following commutative diagram.
%  \begin{figure}[H]
%    \centering
%    \begin{tikzcd}
%      \mor{h_U}{F}{\SET^{\CC\op}} \ar[rrr]
%        \ar[ddd,"(\al\circ \star)"{swap}]
%      & & & F(U) \ar[ddd,"\al_U"] \\
%      & s \ar[r,mapsto] \ar[d,mapsto]
%      & s_U (\id{U}) \ar[d,mapsto] & \\
%      & \al\circ s \ar[r,mapsto] 
%      & (\al\circ s)_U (\id{U}) = \al_U \circ s_U(\id{U})
%      & \\
%      \mor{h_U}{G}{\SET^{\CC\op} } \ar[rrr,""]
%      & & & G(U)
%    \end{tikzcd}
%  \end{figure}
%  We thus have the desired result. 
%
%  To show $h_\star$ is fully faithful, 
%  let $U, V \in \obj{\CC}$ and apply the above to $F := h_V$.
%\end{proof}
\begin{proof}
  We first prove the general statement. 
  Let $U \in \obj{\CC}, F \in \obj{\SET^{\CC\op}}$.
  Given an element $s \in F(U)$,
  we are tasked with constructing a natural transformation $h_U \to F$.
  For $V \in \CC$ we want to map elements 
  $f \in h_U(V)$ to some element of $F(V)$.
  Well, $f$ is a morphism from $V$ to $U$,
  so $F(f)$ is a morphism from $F(U)$ to $F(V)$,
  and we are given an element $s \in F(U)$.
  So define $\al^s_V : h_U(V) \to F(V) := f \mapsto F(f)(s)$.
  For the collection of $\al^s_V$ to form a natural transformation,
  we need naturality. 
  So given $f \in \CC(V,W)$,
  we need the following diagram to commute : 
  \begin{cd}
    h_U(W) \ar[d,"h_U(f)"] \ar[r,"\al^s_W"] & F(W) \ar[d,"F(f)"]\\
    h_U(V) \ar[r,"\al^s_V"] & F(V)
  \end{cd}
  For $g \in h_U(W)$, then we have as desired \[
    \al^s_V \circ h_U(f) (g) = \al^s_V(g \circ f)
    = F(g \circ f) (x) = F(f) \circ F(g) (x)
    = F(f) \circ \al^s_W (g)
  \]
  So $\al^s : h_U \to F$ is a natural transformation.
  
  Note that we can recover $s$ from $\al^s$ by $\al^s_U(\id{U}) = s$.
  This motivates us to define the inverse map by 
  $\al \in \SET^{\CC\op}(h_U,F) \mapsto \al_U(\id{U})$.
  To show these two maps are indeed inverses, 
  first consider the following diagram where 
  $\al : h_U \to F$ is a natural tranformation,
  $W \in \obj{\CC}$ and $f \in h_U(W)$ : 
  \begin{figure}[H]
    \centering
    \begin{tikzcd}
      h_U(U) \arrow[rrr,"a\l_U"] \arrow[ddd,"h_U(f)"{swap}] 
      & 
      & 
      & F(U) \arrow[ddd,"F(f)"] 
      \\
      & \id{U} \ar[r,mapsto] \ar[d,mapsto]& \al_U(\id{U}) \ar[d,mapsto]& \\
      & f \ar[r,mapsto] & \al_W(f) = F(f)(\al_U(\id{U})) & \\
      h_U(W) \arrow[rrr,"\al_W"{swap}] 
      & 
      & 
      & F(W) 
    \end{tikzcd}
  \end{figure}
  The above diagram commutes by naturality of $\al$.
  What it shows is that $\al_W$ is completely determined by $\al_U(\id{U})$,
  and hence $\al$ is completely determined by $\al_U(\id{U})$.
  This proves one side of the inverse situation. 
  The other side is clear.
  Thus we have a bijection between $\SET^{\CC\op}(h_U,F) \iso F(U)$.

  At this point, we can already get $h_\star$ fully faithful by
  applying the above bijection to $F = h_\star$ itself and noting
  the bijection turns $f \in h_V(U)$ into $h_f$. 

  For naturality in the first component,
  let $f : \map{U}{V}{\CC}{}$.
  Then we have the following commutative diagram. 
  \begin{figure}[H]
    \centering
    \begin{tikzcd}
      \mor{h_U}{F}{\SET^{\CC\op}} \ar[rrr]
      & & & F(U) \\
      & \al\circ h^f \ar[r,mapsto]
      & ( \al\circ h^f)_U (\id{U}) 
        = \al_V(f)
        = F(f) \circ \al_V (\id{V}) & \\
      & \al \ar[u,mapsto] \ar[r,mapsto] & \al_V(\id{V}) \ar[u,mapsto] & \\
      \mor{h_V}{F}{\SET^{\CC\op} } \ar[uuu,"(\star\circ h^f)"] \ar[rrr,""]
      & & & F(V) \ar[uuu,"F(f)"{swap}]
    \end{tikzcd}
  \end{figure}
  For naturality in the second component, 
  let $\phi : \map{F}{G}{\SET^{\CC\op}}{}$.
  Then we have the following commutative diagram.
  \begin{figure}[H]
    \centering
    \begin{tikzcd}
      \mor{h_U}{F}{\SET^{\CC\op}} \ar[rrr]
        \ar[ddd,"(\phi\circ \star)"{swap}]
      & & & F(U) \ar[ddd,"\phi_U"] \\
      & \al \ar[r,mapsto] \ar[d,mapsto]
      & \al_U (\id{U}) \ar[d,mapsto] & \\
      & \phi\circ \al \ar[r,mapsto] 
      & (\phi\circ \al)_U (\id{U}) = \phi_U \circ \al_U(\id{U})
      & \\
      \mor{h_U}{G}{\SET^{\CC\op} } \ar[rrr,""]
      & & & G(U)
    \end{tikzcd}
  \end{figure}
  We thus have the desired result. 
\end{proof}

\begin{dfn}[Representable Functors]
  \hypertarget{rep}{}
  
  Let $G : \map{\CC}{\SET}{\CAT}{}$ be a covariant functor. 
  Then a \emph{representation of $G$} is a $(U,u) \in h^\star\darrow G$
  where $u : \map{h^\star}{G}{\SET^\CC}{\sim}$.

  Dually, let $F : \map{\CC\op}{\SET}{}{}$ be a contravariant functor. 
  Then a \emph{representation of $F$} is a $(U,u) \in h_\star\darrow F$
  where $u : \map{h_\star}{F}{\SET^{\CC\op}}{\sim}$.

  A functor (covariant or contravariant) that has a representation is called 
  \emph{representable}.
\end{dfn}

\begin{rmk}
  If a functor has a representation,
  Yoneda's lemma implies it is canonical.
  This is the \hyperlink{canonical_rep}{next result}. 

  Before this, we first relate universal morphisms to representable functors.
  This is important as it leads to the notion of \emph{adjunction}. 
\end{rmk}

\begin{prop}[Universal iff Represents]
  \hypertarget{uni_iff_rep}{}
  
  Let $R : \map{\CC}{\DD}{\CAT}{}$, $X \in \obj{\DD}$,
  $(L(X),\eta_X) \in \obj{X\darrow R}$. 
  Then the following are equivalent : 
  \begin{enumerate}
    \item $(L(X),\eta_X)$ is a universal morphism from $X$ to $R$.
    \item $L(X)$ represents the covariant functor $\mor{X}{R(\star)}{\DD}$
    and $\id{L(X)}$ corresponds to $\eta_X$. 
  \end{enumerate}

  Dually, let $L : \map{\DD}{\CC}{\CAT}{}$, $U \in \obj{\CC}$,
  $(R(U),\ep_U) \in \obj{L\darrow U}$. 
  Then the following are equivalent : 
  \begin{enumerate}
    \item $(R(U),\ep_U)$ is a universal morphism from $L$ to $U$.
    \item $R(U)$ represents the contravariant functor $\mor{L(\star)}{U}{\CC}$
    and $\id{R(U)}$ corresponds to $\ep_U$.
  \end{enumerate}
\end{prop}
\begin{proof}
  (Universal implies Represents) 
  Let $(L(X),\eta_X)$ be a universal morphism from $X$ to $R$. 
  Define the following natural transformation,
  \begin{align*}
    &\map{h^{L(X)}}{\mor{X}{R(\star)}{\DD}}{\SET^\CC}{} := \\
    &W \in \obj{\CC} \mapsto \sqbrkt{
      f \in h^{L(X)}(W) \mapsto R(f) \circ \eta_X \in \mor{X}{R(W)}{\DD}
    }
  \end{align*}
  Then for every $W \in \obj{\CC}$, 
  this is an isomorphism between $h^{L(X)}(W)$ and $\mor{X}{R(W)}{\DD}$,
  and hence a natural isomorphism. 
  Indeed, $\id{L(X)}$ corresponds to $\eta_X$ under this natural isomorphism.

  (Represents implies Universal)
  Let $\al : \map{h^{L(X)}}{\mor{X}{R(\star)}{\DD}}{\SET^\CC}{}$
  be a natural isomorphism where at $L(X)$,
  $\al_{L(X)}(\id{L(X)}) = \eta_X$.
  Let $(V,v) \in \obj{X\darrow R}$.
  For any $f : \map{L(X)}{V}{\CC}{}$,
  consider the following commutative diagram. 
  \begin{figure}[H]
    \centering
    \begin{tikzcd}
      h^{L(X)}(L(X)) \arrow[rrr,"\al_{L(X)}"] \arrow[ddd,"h^{L(X)}(f)"{swap}] 
      & 
      & 
      & \mor{X}{RL(X)}{\DD} \arrow[ddd,"\mor{X}{R(f)}{\DD}"] \\
      & \id{L(X)} \ar[r,mapsto] \ar[d,mapsto]
      & \al_{L(X)}(\id{L(X)}) = \eta_X \ar[d,mapsto] 
      & \\
      & f \ar[r,mapsto] 
      & \al_V(f) = R(f) \circ \eta_X 
      & \\
      h^{L(X)}(V) \arrow[rrr,"\al_V"{swap}] 
      & 
      & 
      & \mor{X}{R(V)}{\DD}
    \end{tikzcd}
  \end{figure}
  Thus $f : \map{(L(X),\eta_X)}{(V,v)}{X\darrow R}{}$
  if and only if $\al_V(f) = v$.
  Then $\al_V\inv(v)$ is 
  the unique morphism $\map{(L(X),\eta_X)}{(V,v)}{X\darrow R}{}$.
  Since there exists a unique $\map{(L(X),\eta_X)}{(V,v)}{X\darrow R}{}$,
  $(L(X),\eta_X)$ is universal.

  The dual equivalence has an analogous proof.
\end{proof}

\begin{prop}[Canonical Representation]
  \hypertarget{canonical_rep}{}
  
  Let $G : \map{\CC}{\DD}{\CAT}{}$ and $(U,u) \in h^\star\darrow G$.
  Then the following are equivalent : 
  \begin{enumerate}
    \item $(U,u)$ is a representation of $G$.
    \item $(U,u)$ is a universal morphism from $h^\star$ to $G$.
  \end{enumerate}
  In particular, 
  representations of $G$ are canonically isomorphic. 

  Dually, let $F : \map{\CC\op}{\DD}{\CAT}{}$ and $(V,v) \in h_\star\darrow F$.
  Then the following are equivalent : 
  \begin{enumerate}
    \item $(V,v)$ is a representation of $F$.
    \item $(V,v)$ is a universal morphism from $h_\star$ to $F$.
  \end{enumerate}
  In particular, 
  representations of $F$ are canonically isomorphic. 
\end{prop}
\begin{proof}
  (Representation implies Universal)
  Let $(W,w) \in \obj{h^\star\darrow G}$.
  Then $u\inv \circ w : \map{h^W}{h^U}{\SET^\CC}{}$.
  By \hyperlink{yoneda}{Yoneda's lemma},
  there exists a unique $u(W,w) : \map{U}{W}{\CC}{}$ such that 
  $u\inv \circ w = h^{u(W,w)}$.
  Hence $u(W,w)$ is the unique morphism 
  $\map{(W,w)}{U,u}{h^\star\darrow G}{}$.

  (Universal implies Representation)
  By \hyperlink{uni_iff_rep}{universal iff represents}
  and \hyperlink{yoneda}{Yoneda's lemma},
  we have the following diagram. 
  \begin{figure}[H]
    \centering
    \begin{tikzcd}[sep = huge]
      \mor{h^\star}{G}{\SET^\CC} 
        \ar[r,"V \in \obj{\CC} \mapsto \sqbrkt{
          s \in \mor{h^V}{G}{\SET^\CC} \mapsto s_V(\id{V})
        }"{yshift = 3mm},"\sim"{swap}]
      & G \\
      h^U \ar[u,
        "V \in \obj{\CC} \mapsto \sqbrkt{
          f \in h^U(V) \mapsto u \circ h^f
        }",
        "\sim"{swap}
        ]
        \ar[ur,"u"]
      &
    \end{tikzcd}
  \end{figure}
  The claim is that the above commutes, and hence $u$ is an isomorphism.
  Let $V \in \obj{\CC}$ and $f \in h^U(V)$.
  Then \begin{align*}
    (h^f \circ u)_V(\id{V})
    = u_V \circ \brkt{h^f}_V \brkt{\id{V}}
    = u_V (f)
  \end{align*}
  So the above diagram commutes. 

  For the dual, the argument is similar. 
\end{proof}
\section{Adjoint Functors}
\begin{dfn}[Adjoint Functors]
  \hypertarget{adjoint}{}
  
  Let $R : \map{\CC}{\DD}{\CAT}{}$.
  Then $R$ is a \emph{right adjoint} when 
  there exists $L : \map{\obj{\DD}}{\obj{\CC}}{}{}$ and 
  $\eta \in \Pi {X \in \obj{\DD}}, \mor{X}{RL(X)}{\DD}$ such that 
  for all $X\in\obj{\DD}$, 
  $(L(X),\eta(X))$ is a universal morphism from $X$ to $R$.
  In this case, $L$ is called the \emph{left adjoint of $R$}.
  \newline 
  
  Dually, let $L : \map{\DD}{\CC}{\CAT}{}$.
  Then $L$ is a \emph{left adjoint} when 
  there exists $R : \map{\obj{\CC}}{\obj{\CC}}{}{}$ and 
  $\ep \in \Pi {U \in \obj{\CC}}, \mor{LR(U)}{U}{\CC}$ such that 
  for all $U\in\obj{\CC}$,
  $(R(U),\ep(U))$ is a universal morphism from $L$ to $U$. 
  In this case, $R$ is called the \emph{right adjoint of $L$}.
\end{dfn}

\begin{dfn}[Product Category]
  \hypertarget{product_cat}{}
  
  Let $\CC, \DD$ be categories. 
  Then the \emph{product category of $\CC, \DD$} is denoted 
  $\CC\times\DD$ and is defined as follows. 
  \begin{enumerate}
    \item $\obj{\CC\times\DD} := \obj{\CC}\times\obj{\DD}$.
    \item For $(U,X), (V,Y) \in \obj{\CC\times\DD}$, 
    $\mor{(U,X)}{(V,Y)}{\CC\times\DD} := \mor{U}{V}{\CC}\times\mor{X}{Y}{\DD}$.
  \end{enumerate}
\end{dfn}

\begin{prop}[Natural Transformations on Product Category]
  \hypertarget{nat_trans_prod_cat}{}
  
  Let $F, G : \map{\CC\times\DD}{\EE}{}{}$, 
  $\al \in \Pi (U,X) \in \obj{\CC\times\DD}, \mor{F(U)}{G(X)}{\EE}$.
  Then the following are equivalent. 
  \begin{enumerate}
    \item $\al : \map{F}{G}{}{}$.
    \item For all $(U,X) \in \obj{\CC\times\DD}$, 
    $\al(U,-) : \map{F(U,-)}{G(U,-)}{}{}$ and 
    $\al(-,X) : \map{F(-,X)}{G(-,X)}{}{}$.
  \end{enumerate}
\end{prop}

\begin{dfn}[Adjunction]
  \hypertarget{adjunction}{}
  
  Let $R : \map{\CC}{\DD}{\CAT}{}$ and $L : \map{\DD}{\CC}{\CAT}{}$.
  We have the two functors 
  $\mor{L(\star)}{-}{\CC}, \mor{\star}{R(-)}{\DD} : 
  \map{\DD\op\times\CC}{\SET}{\CAT}{}$.
  Then $(L,R)$ is an \emph{adjunction} when 
  $\mor{L(\star)}{-}{\CC}, \mor{\star}{R(-)}{\DD}$
  are naturally isomorphic. 
  
  In this case, 
  $R$ is called the \emph{right adjoint of $L$}
  and $L$ is called the \emph{left adjoint of $R$}.
  The isomorphism is called the \emph{adjunction isomorphism}. 
  For all $f : \map{L(X)}{U}{\CC}{}$,
  the image of $f$ under the adjunction isomorphism is called 
  the \emph{adjunct of $f$}, denoted $f^\bot$. 
  Similarly for $g : \map{X}{R(U)}{\DD}{}$, 
  we have the \emph{adjunct of $g$}, denoted $g^\bot$.
\end{dfn}

\begin{prop}[Universal Morphism Characterisation of Adjunction]
  \hypertarget{uniprop_char_adj}{}

  Let $R : \map{\CC}{\DD}{\CAT}{}$.
  Then the following are equivalent : 
  \begin{enumerate}
    \item $R$ is a right adjoint. 
    \item There exists $L : \map{\DD}{\CC}{\CAT}{}$ such that 
    $(L,R)$ is an adjunction. 
  \end{enumerate}

  Dually, let $L : \map{\DD}{\CC}{\CAT}{}$.
  Then the following are equivalent : 
  \begin{enumerate}
    \item $L$ is a left adjoint. 
    \item There exists $R : \map{\CC}{\DD}{\CAT}{}$ such that 
    $(L,R)$ is an adjunction. 
  \end{enumerate}
\end{prop}
\begin{proof}
  $(\implies)$
  Let $R$ be a right adjoint, $L : \map{\obj{\DD}}{\obj{\CC}}{}{}$, 
  $\eta \in \Pi X \in \obj{\DD}, \mor{X}{RL(X)}{\DD}$,
  for all $X \in \obj{\DD}$, $(L(X),\eta(X))$ universal morphism from 
  $X$ to $R$.

  The universal properties at every $X \in \obj{\DD}$ implies 
  $L$ is functorial.
  By \hyperlink{uni_iff_rep}{universal iff represents},
  for all $X \in \obj{\DD}$, 
  we have $\mor{L(X)}{-}{\CC} \cong \mor{X}{R(-)}{\DD}$
  as functors $\map{\CC}{\SET}{}{}$.
  Let $f : \map{X}{Y}{\DD}{}$ and $U \in \obj{\CC}$.
  Then we have the following commutative diagram. 
  \begin{figure}[H]
    \centering
    \begin{tikzcd}
      \mor{L(X)}{U}{\CC} \ar[rrr,"R(-)\circ\eta(X)"]
      & & & \mor{X}{R(U)}{\DD} \\
      & g \circ L(f) \ar[r,mapsto]
      & R(g \circ L(f)) \circ \eta(X) = R(g) \circ \eta(Y) \circ f
      & \\
      & g \ar[u,mapsto] \ar[r,mapsto] & R(g)\circ\eta(Y) \ar[u,mapsto] & \\
      \mor{L(Y)}{U}{\CC} \ar[uuu,"h^{L(f)}"]
        \ar[rrr,"R(-)\circ\eta(Y)"{swap}]
      & & & \mor{Y}{R(U)}{\DD} \ar[uuu,"h^f"{swap}]
    \end{tikzcd}
  \end{figure}
  Thus the isomorphism $\mor{L(X)}{-}{\CC} \cong \mor{X}{R(-)}{\DD}$
  is functorial in $X$, 
  and hence an isomorphism between
  $\mor{L(\star)}{-}{\CC} \cong \mor{\star}{R(-)}{\DD}$.

  $(\limplies)$ 
  Let $L : \map{\obj{\DD}}{\CC}{\CAT}{}$ such that 
  $(L,R)$ is an adjunction. 
  Then for each $X \in \obj{\DD}$, 
  $\mor{L(X)}{-}{\CC} \cong \mor{X}{R(-)}{\DD}$.
  Let $\eta(X)$ be the adjunct of $\id{L(X)}$.
  By \hyperlink{uni_iff_rep}{universal iff represents},
  $(L(X),\eta(X))$ is a universal morphism from $X$ to $R$.

  The dual has a similar argument. 
\end{proof}

\begin{prop}[Uniqueness of Adjoints]
  \hypertarget{adj_unique}{}

  Let $R, R_1 : \map{\CC}{\DD}{\CAT}{}$, $L, L_1: \map{\DD}{\CC}{\CAT}{}$.
  Then \begin{enumerate}
    \item If $(L,R)$ and $(L,R_1)$ are both adjunctions, 
    then $R \cong R_1$ as functors. 
    \item If $(L,R)$ and $(L_1, R)$ are both adjunctions, 
    then $L \cong L_1$ as functors. 
  \end{enumerate}
\end{prop}
\begin{proof}
  $(1)$ Let $(L,R), (L,R_1)$ both be adjunctions. 
  Let $f : \map{U}{V}{\CC}{}$. 
  We have an isomorphism between the functors 
  $\mor{-}{R(U)}{\DD}$ and $\mor{-}{R_1(U)}{\DD}$
  for all $U \in \obj{\CC}$. 
  By \hyperlink{yoneda}{Yoneda's lemma}, 
  these isomorphisms are equal to $h_{\al_U}$ 
  for some unique morphism $\al_U : \map{R(U)}{R_1(U)}{\DD}{}$.
  So we have the following commutative diagram. 
  \begin{figure}[H]
    \centering
    \begin{tikzcd}
      \mor{-}{R(U)}{\DD} \ar[r,"h_{\al_U}","\sim"{swap}] 
        \ar[d,"h_{R(f)}"{swap}]
      & \mor{-}{R_1(U)}{\DD} \ar[d,"h_{R_1(f)}"] \\
      \mor{-}{R(V)}{\DD} \ar[r,"h_{\al_V}"{swap},"\sim"] 
      & \mor{-}{R_1(V)}{\DD}
    \end{tikzcd}
  \end{figure}
  Again by Yoneda, 
  we have $R_1(f) \circ \al_U = \al_V \circ R(f)$.
  The fact that $h_{\al_U}$ is an isomorphism implies 
  $\al_U$ is an isomorphism. 
  Thus $\al$ is a natural isomorphism between $R, R_1$.

  $(2)$ Analogous. 
\end{proof}

\begin{rmk}
  There is another characterisation of adjunctions.
\end{rmk}

\begin{prop}[Unit/Counit Characterisation of Adjunction]
  \hypertarget{unit_char_adj}{}
  
  Let $R : \map{\CC}{\DD}{\CAT}{}$ and $L : \map{\DD}{\CC}{\CAT}{}$.
  Then the following are equivalent : 
  \begin{enumerate}
    \item (Morphism Isomorphism) $(R,L)$ is an adjunction. 
    \item (Unit-Counit) 
    There exists $\eta : \map{\id{\DD}}{RL}{}{}$ and 
    $\ep : \map{LR}{\id{\CC}}{}{}$ such that 
    \begin{enumerate}
      \item $\id{L} = \ep L \circ L \eta$,
      that is to say for all $X \in \obj{D}$, 
      we have the following commutative diagram. 
      \begin{figure}[H]
        \centering
        \begin{tikzcd}
          L(X) \ar[rd,"\id{L(X)}"{swap}] \ar[r,"L(\eta(X))"]
          & LRL(X) \ar[d,"\ep(L(X))"] \\
          & L(X)
        \end{tikzcd}
      \end{figure}
      \item $\id{R} = R\ep \circ \eta R$, i.e. 
      for all $U \in \obj{\CC}$, 
      we have the following commutative diagram. 
      \begin{figure}[H]
        \centering
        \begin{tikzcd}
          R(U) \ar[rd,"\id{R(U)}"{swap}] \ar[r,"\eta(R(U))"] 
          & RLR(U) \ar[d,"R(\ep(U))"] \\
          & R(U) 
        \end{tikzcd}
      \end{figure}
    \end{enumerate}
    The above two equations are often called \emph{triangle-identities}.
  \end{enumerate}
\end{prop}
\begin{proof}
  $(1\implies 2)$
  For all $X \in \obj{\DD}$, 
  the adjunction isomorphism gives an isomorphism of functors
  \[
    \mor{L(X)}{-}{\CC} \iso \mor{X}{R(-)}{\DD}
  \]
  Define $\eta(X) := \id{L(X)}^\bot$. 
  Then by \hyperlink{uni_iff_rep}{universal iff represents}, 
  $(L(X),\eta(X))$ is a universal morphism from $X$ to $R$. 
  We claim that $\eta : \map{\id{\DD}}{RL}{}{}$. 

  Let $f : \map{X}{Y}{\DD}{}$. 
  Then by the universal property of $(L(X),\eta(X))$,
  we have the following commutative diagram. 
  \begin{figure}[H]
    \centering
    \begin{tikzcd}
      X \ar[r,"\eta(X)"] \ar[d,"f"{swap}] & RL(X) \ar[d,"RL(f)"] \\
      Y \ar[r,"\eta(Y)"{swap}] & RL(Y)
    \end{tikzcd}
  \end{figure}
  i.e. $\eta$ is a natural transformation as desired. 
  We similarly define $\ep(U) := \id{R(U)}^\bot$ for $U \in \obj{\CC}$
  and see that $\ep : \map{LR}{\id{\CC}}{}{}$. 

  To prove $(a)$, let $X \in \obj{\DD}$. 
  Then \begin{align*}
    \id{L(X)} 
    = \brkt{\id{L(X)}^\bot}^\bot 
    = \brkt{\eta(X)}^\bot 
    = \ep(L(X)) \circ L(\eta(X)) 
  \end{align*}
  where the last equality follows from 
  the universal property of $(RL(X),\ep(L(X)))$. 
  Similarly for $(b)$, we have for $U \in \obj{\CC}$, 
  \begin{align*}
    \id{R(U)} 
    = \brkt{\id{R(U)}^\bot}^\bot 
    = \brkt{\ep(U)}^\bot 
    = R(\ep(U)) \circ \eta(R(U))
  \end{align*}
  where the last equality is by
  the universal property of $(LR(U),\eta(R(U)))$.

  $(2\implies 1)$ 
  Let $(X,U)\in\obj{\DD\op \times \CC}$. 
  Since $(L(X),\eta(X))$ is supposed to be 
  a universal morphism from $X$ to $R$, 
  we define the adjunction map to be 
  \begin{align*}
    \mor{L(X)}{U}{\CC} &\overset{\bot}{\longleftrightarrow} 
    \mor{X}{R(U)}{\DD} \\
    f &\longmapsto R(f) \circ \eta(X) \\
    \ep(U) \circ L(g) &\longmapsfrom g
  \end{align*}
  Then for $f : \map{L(X)}{U}{\CC}{}$, 
  \begin{align*}
    \brkt{f^\bot}^\bot 
    &= \ep(U) \circ L(f^\bot) = \ep(U) \circ L\brkt{R(f) \circ \eta(X)} \\
    &= \ep(U) \circ LR(f) \circ L(\eta(X)) 
    = f \circ \ep(L(X)) \circ L(\eta(X)) = f
  \end{align*}
  Similarly, $\brkt{g^\bot}^\bot = g$.
  So $\bot$ is an isomorphism at all $(X,U)$.
  
  It remains to show naturality. 
  \hyperlink{nat_trans_prod_cat}{It suffices} to show that 
  the isomorphism is natural in both components. 
  Let $f : \map{X}{Y}{\DD\op}{}$. 
  Then we have the following diagram. 
  \begin{figure}[H]
    \centering
    \begin{tikzcd}
      \mor{L(X)}{U}{\CC} \ar[r,"\bot"] \ar[d,"h^{L(f)}"{swap}]
      & \mor{X}{R(U)}{\DD} \ar[d,"h^f"] \\
      \mor{L(Y)}{U}{\CC} \ar[r,"\bot"{swap}] & \mor{Y}{R(U)}{\DD}
    \end{tikzcd}
  \end{figure}
  It follows from $\eta : \map{\id{\DD}}{RL}{}{}$ that the above commutes.
  Similarly, naturality of $\ep$ implies naturality in the second component. 
  Hence $\bot$ is a natural isomorphism as desired. 
\end{proof}

\begin{rmk}
  The following is a special case of adjunction that is worth noting. 
\end{rmk}
\begin{dfn}[Galois Connection]
  
  Let $I, J$ be partially ordered sets. 
  Then $I, J$ can be seen as categories. 
  A \emph{monotone Galois connection between $I,J$} is 
  an adjunction between $I, J$.
  A \emph{antitone Galois connection between $I,J$} is 
  an adjunction between $I\op, J$. 
\end{dfn}
\begin{rmk}
  The \hyperlink{unit_char_adj}{unit/counit characterisation of adjunctions}
  shows that if $(R,L)$ is a Galois connection (mono or anti) between 
  partially ordered sets $I,J$, 
  then $R$ and $L$ are bijective on their images. 
\end{rmk}

\begin{dfn}[Free Functors]\link{free}
  
\end{dfn}
\section{Limits and Colimits}
\begin{dfn}[(Co)Diagrams]\link{diagrams}
  
  Let $\II, \CC$ be categories. 
  Then an \emph{$\II$-diagram in $\CC$} is 
  a covariant functor from $\II$ to $\CC$. 
  Dually, an \emph{$\II$-codiagram} is a contravariant functor 
  from $\II$ to $\CC$,
  i.e. an $\II\op$-diagram. 
\end{dfn}

\begin{rmk}
  Often, it is easier to take $\II$ to be a subcategory of $\CC$. 
\end{rmk}

\begin{dfn}[Constant (Co)Diagrams]\link{const}
  
  Let $\II, \CC$ be categories and $U \in \obj{\CC}$. 
  Then define the \emph{constant diagram $\De(U)$} as follows. 
  \begin{enumerate}
    \item For all $i \in \II$, $\De(U)(i) := U$. 
    \item For all $\phi : \map{i}{j}{\II}{}$, 
    $\De(U)(\phi) := \id{U}$.
  \end{enumerate}
  Dually, we have the \emph{constant codiagram $\De\op(U)$} defined as : 
  \begin{enumerate}
    \item For all $i \in \obj{\II}$, $\De\op(U)(i) := U$. 
    \item For all $\phi : \map{i}{j}{\II\op}{}$, 
    $\De(U)(\phi) := \id{U}$.
  \end{enumerate}
\end{dfn}

\begin{prop}[Functoriality of Constant (Co)Diagrams]\link{const_funk}
  
  Let $\II, \CC$ be categories. 
  Then $\De : \map{\CC}{\CC^\II}{}{}$.
  Dually, $\De\op : \map{\CC}{\CC^{\II\op}}{}{}$.
\end{prop}

\begin{dfn}[(Co)Limits of (Co)Diagrams]\link{limit}
  
  Let $\II, \CC$ be categories, $X$ a $\II$-diagram in $\CC$,
  and $Y$ a $\II$-codiagram in $\CC$.

  Then a \emph{limit of $X$} is a universal morphism 
  from $\De$ to $X$. 
  If a limit of $X$ exists, 
  it is \linkto{canonical}{canonical} and 
  referred to as \emph{the} limit,
  denoted $(\LIM{X}{}, \pi_X)$.
  
  Dually, a \emph{colimit of $Y$} is a universal morphism from $Y$ to $\De\op$. 
  If a colimit of $Y$ exists, 
  it is canonical and referred to as \emph{the} colimit, 
  denoted with $(\COLIM{Y}{}, \io_Y)$.
\end{dfn}

\begin{rmk}
  Sometimes limits are also called \emph{projective limits},
  and colimits are called \emph{injective limits}. 
\end{rmk}

\begin{dfn}[(Co)Completeness]\link{complete}
  
  Let $\CC$ be a category. 
  Then it is called \emph{complete} when 
  for all ``small'' categories $\II$ and diagrams $X : \map{\II}{\CC}{\CAT}{}$,
  there exists the limit of $X$.

  Dually, it is called \emph{cocomplete} when 
  for all ``small'' categories $\II$ and 
  codiagrams $Y : \map{\II\op}{\CC}{\CAT}{}$,
  there exists the colimit of $Y$.
\end{dfn}

\begin{rmk}
  We now cover important examples of limits and colimits. 
\end{rmk}

\begin{dfn}[Discrete Category]\link{discrete}
  
  For $I \in \obj{\SET}$, 
  $I$ can be turned into a category by having elements as objects 
  and the only morphisms being identity morphisms. 
  Categories obtained in this way are called \emph{discrete categories}.
\end{dfn}

\begin{rmk}
  Note that for a discrete category $\II$,
  $\II$ and $\II\op$ are isomorphic in an obvious way. 
  Consequently, it is best to think of 
  $\II$-diagrams and $\II$-codiagrams as ``the same''.
\end{rmk}

\begin{dfn}[(Co)Products]\link{prod}
  
  Let $\CC$ be a category and $\II$ a discrete category. 

  Let $X$ be an $\II$-diagram in $\CC$.
  Then the limit of $X$ is called the \emph{product of $X(i)$}.

  Dually, let $Y$ be an $\II$-codiagram in $\CC$.
  Then the colimit of $Y$ is called the \emph{coproduct of $Y(i)$}.

  In the special case of $I = \nothing$, 
  the product is called the \emph{final object of $\CC$}.
  Dually, the coproduct is called the \emph{initial object of $\CC$}.
\end{dfn}

\begin{eg}[Final Objects]\link{eg_final}
  
\end{eg}

\begin{eg}[Initial Objects]\link{eg_initial}
  
\end{eg}

\begin{eg}[Products]\link{eg_prod}
  
\end{eg}

\begin{eg}[Coproducts]\link{eg_coprod}
  
\end{eg}

\begin{dfn}[(Co)Equalizers]\link{equalizer}
  
  Let $\CC$ be a category. 
  Let $I$ be an arbitrary set and $\II$ be the following category. 
  \begin{figure}[H]
    \centering
    \begin{tikzcd}
      0 \ar[loop above,"\id{0}"] \ar[r,"i"]
      & 1 \ar[loop above,"\id{1}"] 
    \end{tikzcd}
  \end{figure}
  where there is a morphism $i : \map{0}{1}{\II}{}$ for all $i \in I$. 

  Let $X$ be an $\II$-diagram in $\CC$. 
  Then the limit of $X$ is called 
  the \emph{equalizer of $X(i)$'s}. 
  Dually, let $Y$ be an $\II$-codiagram in $\CC$. 
  Then the colimit of $Y$ is called the 
  \emph{coequalizer of $Y(i)$'s}.
\end{dfn}

\begin{eg}[Equalizers]
  
\end{eg}

\begin{eg}[Coequalizers]
  
\end{eg}

\begin{dfn}[Pullbacks and Pushouts]\link{pullback}
  
  Let $\CC$ be a category, $U \in \obj{\CC}$.
  Then a \emph{pullback over $U$} is a product in the 
  category $\CC\darrow U$.
  Dually, a \emph{pushout under $U$} is a coproduct in the 
  category $U\darrow \CC$.

  Let $I$ be an arbitrary set and $\II$ the following category. 
  \begin{figure}[H]
    \centering
    \begin{tikzcd}
      i \ar[loop above,"\id{0}"] \ar[r,"\phi(i)"]
      & * \ar[loop above,"\id{1}"] 
    \end{tikzcd}
  \end{figure}
  \begin{enumerate}
    \item $\obj{\II} = I \sqcup \set{*}$.
    \item For all $x \in \obj{\II}$, $\mor{x}{x}{\II} = \set{\id{x}}$.
    \item For all $i \in I$, $\mor{i}{*}{\II} = \set{\phi(i)}$.
  \end{enumerate}
  Then a pullback over $U$ is equivalently the limit of 
  an $\II$-diagram $X$ with $X(*) = U$. 
  Dually, a pushout under $U$ is equivalently the colimit of 
  an $\II$-codiagram $Y$ with $Y(*) = U$.
\end{dfn}
\section{Completeness}
\begin{prop}[Characterisation of Completeness, Cocompleteness]
  \link{char_complete}
  
  
\end{prop}

\begin{prop}[$\SET$ Complete]\label{set_complete}
  
\end{prop}

\begin{prop}[$\TOP$ Complete]\label{top_complete}
  
\end{prop}

\begin{prop}[$\GRP$ Complete]\label{grp_complete}
  
\end{prop}

\begin{prop}[$\RING$ Complete]\label{ring_complete}
  
\end{prop}

\begin{prop}[$\MOD$ Complete]\label{mod_complete}
  
\end{prop}

\begin{prop}[Set-theoretic Characterisation of Limits and Colimits]
  \link{set_char_lim}
  
  Let $X$ be an $\II$-shaped diagram in a category $\CC$ and 
  $(U,u) \in \obj{\De\darrow X}$. 
  We have an $\II$-shaped diagram in $\SET^{\CC\op}$ 
  that is $h_X := h_{\star} \circ X$. 
  We also have $h_{\De(U)} = \De(h_U)$ and 
  the natural transformation $h_u : \map{\De(h_U)}{h_X}{}{}$. 
  So $(h_U,h_u) \in \obj{\De\darrow h_X}$.
  Then the following are equivalent. 
  \begin{enumerate}
    \item $(U,u)$ is a limit of $X$. 
    \item $(h_U,h_u)$ is a limit of $h_X$.
  \end{enumerate}
\end{prop}

\begin{cor}[Right Adjoints commute with Limits,
  Left Adjoints commute with Colimits]\label{cts}
  
\end{cor}

\begin{dfn}[Filtered Sets and Filtered Colimits]\label{filtered}
  
\end{dfn}

\begin{prop}[Filtered Colimits commute with Finite Limits]
  \label{filtered_colimits_commute}
  
\end{prop}
\section{Abelian Categories}
\begin{dfn}[Zero Objects]\label{zero}
  
\end{dfn}

\begin{dfn}[Kernels and Cokernels]\label{ker}
  
\end{dfn}

\end{document}