\begin{rmk}
  The idea that limits and colimits formalize is 
  that of making new objects out of given objects within a category
  in the ``most efficient way possible''. 
\end{rmk}

\begin{dfn}[Diagrams]\link{diagrams}
  
  Let $\II, \CC$ be categories. 
  Then an \emph{$\II$-shaped diagram in $\CC$} is 
  a covariant functor from $\II$ to $\CC$. 
\end{dfn}

\begin{rmk}
  Often, it is easier to take $\II$ to be a subcategory of $\CC$. 
\end{rmk}

\begin{dfn}[Constant (Co)Diagrams]\link{const}
  
  Let $\II, \CC$ be categories and $U \in \obj{\CC}$. 
  Then define the \emph{constant diagram $\De(U)$} as follows. 
  \begin{enumerate}
    \item For all $i \in \II$, $\De(U)(i) := U$. 
    \item For all $\phi : \map{i}{j}{\II}{}$, 
    $\De(U)(\phi) := \id{U}$.
  \end{enumerate}
  Then $\De : \map{\CC}{\CC^\II}{}{}$.
  % Dually, we have the \emph{constant codiagram $\De\op(U)$} defined as : 
  % \begin{enumerate}
  %   \item For all $i \in \obj{\II}$, $\De\op(U)(i) := U$. 
  %   \item For all $\phi : \map{i}{j}{\II\op}{}$, 
  %   $\De(U)(\phi) := \id{U}$.
  % \end{enumerate}
\end{dfn}

% \begin{prop}[Functoriality of Constant (Co)Diagrams]\link{const_funk}
%   
%   Let $\II, \CC$ be categories. 
%   Then $\De : \map{\CC}{\CC^\II}{}{}$.
%   Dually, $\De\op : \map{\CC}{\CC^{\II\op}}{}{}$.
% \end{prop}

\begin{dfn}[(Co)Limits of Diagrams]\link{limit}
  
  Let $\II, \CC$ be categories, $X$ a $\II$-shaped diagram in $\CC$.
  % and $Y$ a $\II\op$-shaped diagram in $\CC$.
  Let $(U,u) \in \obj{\De\darrow X}$. 
  Then the following are equivalent : 
  \begin{itemize}
    \item $(U,u)$ is a universal morphism from $\De$ to $X$.
    \item $U$ represents the functor $\CC^\II(\De(-),X) : \CC\op \to \SET$,
    i.e. $\CC^\II(\De(-),X) \iso \CC(-,U)$ where $\id{U}$ corresponds to $u$.
  \end{itemize}
  When any (and thus all) of these are true,
  $(U,u)$ is called a \emph{limit of $X$}.
  By abuse of language, we sometimes simply say $U$ is a limit of $X$.
  If a limit of $X$ exists, 
  it is \linkto{canonical}{canonical}.
  So we will say \emph{the} limit to refer to \emph{any} limit,
  and denote it using $(\LIM X, \pi_X)$.
  When a limit exists, we will say \emph{the} limit exists.
  
  Dually, let $(V,v) \in \obj{X\darrow \De}$.
  Then the following are equivalent : 
  \begin{itemize}
    \item $(V,v)$ is a universal morphism from $X$ to $\De$.
    \item $V$ represents the functor $\CC^\II(X,\De(-)) : \CC \to \SET$,
    i.e. $\CC(V,-) \iso \CC^\II(X,\De(-))$ where $\id{V}$ corresponds to $v$.
  \end{itemize}
  When any (and thus all) of these are true,
  $(V,v)$ is called a \emph{colimit of $X$}.
  By abuse of language, we sometimes simply say $V$ is a colimit of $X$.
  If a colimit of $X$ exists, it is canonical.
  So we will say \emph{the} colimit to refer to \emph{any} colimit,
  and denote it with $(\COLIM X, \io_X)$.
  When a colimit exists, we will say \emph{the} colimit exists.
\end{dfn}

\begin{rmk}[Some Terminology]
  Sometimes limits are also called \emph{projective limits},
  and colimits are called \emph{injective limits}. 
  Also, objects in $\De\darrow X$ are often called \emph{cones}.
  Dually, objects in $X\darrow\De$ are called \emph{cocones}.
  In my experience,
  being able to ``see the (co)cones'' has been great
  for keeping a clear mind in later proofs.
\end{rmk}

\begin{rmk}
  We now cover important instances of (co)limits.
  These are special in the sense that 
  \linkto{char_complete}{all (co)limits can be built from them}. 
\end{rmk}

\begin{dfn}[Discrete Category]\link{discrete}
  
  For $I \in \obj{\SET}$, 
  $I$ can be turned into a category by having elements as objects 
  and the only morphisms being identity morphisms. 
  Categories obtained in this way are called \emph{discrete categories}.
\end{dfn}

\begin{rmk}
  Taking discrete categories is the free functor adjoint to 
  the forgetful functor from $\CAT$ to $\SET$ by taking the set of objects.
  \footnote{
    Russell cries, but we shall ignore him. 
  }
\end{rmk}

\begin{dfn}[(Co)Products]\link{prod}
  
  Let $\CC$ be a category and $\II$ a discrete category.
  Let $X$ be an $\II$-diagram in $\CC$.
  If a limit of $X$ exists,
  it is called the \emph{product of $X(i)$}.
  Dually, if a colimit of $X$ exists,
  it is called the \emph{coproduct of $X(i)$}.

  In the special case of $I = \nothing$, 
  the product is called the \emph{final object of $\CC$}.
  Dually, the empty coproduct is called the \emph{initial object of $\CC$}.
\end{dfn}

\begin{eg}[Final Objects]\link{eg_final}
  
  (TODO)
\end{eg}

\begin{eg}[Initial Objects]\link{eg_initial}
  
  (TODO)
\end{eg}

\begin{eg}[Products]\link{eg_prod}
  
  (TODO)
\end{eg}

\begin{eg}[Coproducts]\link{eg_coprod}
  
  (TODO)
\end{eg}

\begin{dfn}[(Co)Equalizers]\link{equalizer}
  
  Let $\CC$ be a category. 
  Let $I$ be an arbitrary set and $\II$ be the following category. 
  \begin{figure}[H]
    \centering
    \begin{tikzcd}
      0 \ar[loop above,"\id{0}"] \ar[r,"i"]
      & 1 \ar[loop above,"\id{1}"] 
    \end{tikzcd}
  \end{figure}
  where there is a morphism $i : \map{0}{1}{\II}{}$ for all $i \in I$. 

  Let $X$ be an $\II$-diagram in $\CC$. 
  If a limit of $X$ exists, 
  it is called the \emph{equalizer of $X(i)$'s}. 
  Dually, if a colimit of $X$ exists, 
  it is called the \emph{coequalizer of $X(i)$'s}.
\end{dfn}

\begin{eg}[Equalizers]
  
  (TODO)
\end{eg}

\begin{eg}[Coequalizers]
  
  (TODO)
\end{eg}

\begin{dfn}[Pullbacks and Pushouts]\link{pullback}
  
  Let $\CC$ be a category, $U \in \obj{\CC}$.
  Then a \emph{pullback over $U$} is a product in the 
  category $\CC\darrow U$.
  Dually, a \emph{pushout under $U$} is a coproduct in the 
  category $U\darrow \CC$.

  Let $I$ be an arbitrary set and $\II$ the following category. 
  \begin{figure}[H]
    \centering
    \begin{tikzcd}
      i \ar[loop above,"\id{0}"] \ar[r,"\phi(i)"]
      & * \ar[loop above,"\id{1}"] 
    \end{tikzcd}
  \end{figure}
  \begin{enumerate}
    \item $\obj{\II} = I \sqcup \set{*}$.
    \item For all $x \in \obj{\II}$, $\mor{x}{x}{\II} = \set{\id{x}}$.
    \item For all $i \in I$, $\mor{i}{*}{\II} = \set{\phi(i)}$.
  \end{enumerate}
  Then a pullback over $U$ is equivalently the limit of 
  an $\II$-shaped diagram $X$ with $X(*) = U$. 
  Dually, a pushout under $U$ is equivalently the colimit of 
  an $\II\op$-shaped diagram $Y$ with $Y(*) = U$.
  \footnote{
    We could have define colimits only for $\II\op$-shaped diagrams 
    so that coproduct, coequalizer, pushout are all related to
    their respective duals in the same way. 
    However, other than this, 
    doing things this way has little practical nor theoretical significance,
    hence my choice of defining colimits over $\II$-shaped diagrams.
  }
\end{dfn}