\begin{dfn}[(Co)Diagrams]\link{diagrams}
  
  Let $\II, \CC$ be categories. 
  Then an \emph{$\II$-diagram in $\CC$} is 
  a covariant functor from $\II$ to $\CC$. 
  Dually, an \emph{$\II$-codiagram} is a contravariant functor 
  from $\II$ to $\CC$,
  i.e. an $\II\op$-diagram. 
\end{dfn}

\begin{rmk}
  Often, it is easier to take $\II$ to be a subcategory of $\CC$. 
\end{rmk}

\begin{dfn}[Constant (Co)Diagrams]\link{const}
  
  Let $\II, \CC$ be categories and $U \in \obj{\CC}$. 
  Then define the \emph{constant diagram $\De(U)$} as follows. 
  \begin{enumerate}
    \item For all $i \in \II$, $\De(U)(i) := U$. 
    \item For all $\phi : \map{i}{j}{\II}{}$, 
    $\De(U)(\phi) := \id{U}$.
  \end{enumerate}
  Dually, we have the \emph{constant codiagram $\De\op(U)$} defined as : 
  \begin{enumerate}
    \item For all $i \in \obj{\II}$, $\De\op(U)(i) := U$. 
    \item For all $\phi : \map{i}{j}{\II\op}{}$, 
    $\De(U)(\phi) := \id{U}$.
  \end{enumerate}
\end{dfn}

\begin{prop}[Functoriality of Constant (Co)Diagrams]\link{const_funk}
  
  Let $\II, \CC$ be categories. 
  Then $\De : \map{\CC}{\CC^\II}{}{}$.
  Dually, $\De\op : \map{\CC}{\CC^{\II\op}}{}{}$.
\end{prop}

\begin{dfn}[(Co)Limits of (Co)Diagrams]\link{limit}
  
  Let $\II, \CC$ be categories, $X$ a $\II$-diagram in $\CC$,
  and $Y$ a $\II$-codiagram in $\CC$.

  Then a \emph{limit of $X$} is a universal morphism 
  from $\De$ to $X$. 
  If a limit of $X$ exists, 
  it is \linkto{canonical}{canonical} and 
  referred to as \emph{the} limit,
  denoted $(\LIM{X}{}, \pi_X)$.
  
  Dually, a \emph{colimit of $Y$} is a universal morphism from $Y$ to $\De\op$. 
  If a colimit of $Y$ exists, 
  it is canonical and referred to as \emph{the} colimit, 
  denoted with $(\COLIM{Y}{}, \io_Y)$.
\end{dfn}

\begin{rmk}
  Sometimes limits are also called \emph{projective limits},
  and colimits are called \emph{injective limits}. 
\end{rmk}

\begin{dfn}[(Co)Completeness]\link{complete}
  
  Let $\CC$ be a category. 
  Then it is called \emph{complete} when 
  for all ``small'' categories $\II$ and diagrams $X : \map{\II}{\CC}{\CAT}{}$,
  there exists the limit of $X$.

  Dually, it is called \emph{cocomplete} when 
  for all ``small'' categories $\II$ and 
  codiagrams $Y : \map{\II\op}{\CC}{\CAT}{}$,
  there exists the colimit of $Y$.
\end{dfn}

\begin{rmk}
  We now cover important examples of limits and colimits. 
\end{rmk}

\begin{dfn}[Discrete Category]\link{discrete}
  
  For $I \in \obj{\SET}$, 
  $I$ can be turned into a category by having elements as objects 
  and the only morphisms being identity morphisms. 
  Categories obtained in this way are called \emph{discrete categories}.
\end{dfn}

\begin{rmk}
  Note that for a discrete category $\II$,
  $\II$ and $\II\op$ are isomorphic in an obvious way. 
  Consequently, it is best to think of 
  $\II$-diagrams and $\II$-codiagrams as ``the same''.
\end{rmk}

\begin{dfn}[(Co)Products]\link{prod}
  
  Let $\CC$ be a category and $\II$ a discrete category. 

  Let $X$ be an $\II$-diagram in $\CC$.
  Then the limit of $X$ is called the \emph{product of $X(i)$}.

  Dually, let $Y$ be an $\II$-codiagram in $\CC$.
  Then the colimit of $Y$ is called the \emph{coproduct of $Y(i)$}.

  In the special case of $I = \nothing$, 
  the product is called the \emph{final object of $\CC$}.
  Dually, the coproduct is called the \emph{initial object of $\CC$}.
\end{dfn}

\begin{eg}[Final Objects]\link{eg_final}
  
\end{eg}

\begin{eg}[Initial Objects]\link{eg_initial}
  
\end{eg}

\begin{eg}[Products]\link{eg_prod}
  
\end{eg}

\begin{eg}[Coproducts]\link{eg_coprod}
  
\end{eg}

\begin{dfn}[(Co)Equalizers]\link{equalizer}
  
  Let $\CC$ be a category. 
  Let $I$ be an arbitrary set and $\II$ be the following category. 
  \begin{figure}[H]
    \centering
    \begin{tikzcd}
      0 \ar[loop above,"\id{0}"] \ar[r,"i"]
      & 1 \ar[loop above,"\id{1}"] 
    \end{tikzcd}
  \end{figure}
  where there is a morphism $i : \map{0}{1}{\II}{}$ for all $i \in I$. 

  Let $X$ be an $\II$-diagram in $\CC$. 
  Then the limit of $X$ is called 
  the \emph{equalizer of $X(i)$'s}. 
  Dually, let $Y$ be an $\II$-codiagram in $\CC$. 
  Then the colimit of $Y$ is called the 
  \emph{coequalizer of $Y(i)$'s}.
\end{dfn}

\begin{eg}[Equalizers]
  
\end{eg}

\begin{eg}[Coequalizers]
  
\end{eg}

\begin{dfn}[Pullbacks and Pushouts]\link{pullback}
  
  Let $\CC$ be a category, $U \in \obj{\CC}$.
  Then a \emph{pullback over $U$} is a product in the 
  category $\CC\darrow U$.
  Dually, a \emph{pushout under $U$} is a coproduct in the 
  category $U\darrow \CC$.

  Let $I$ be an arbitrary set and $\II$ the following category. 
  \begin{figure}[H]
    \centering
    \begin{tikzcd}
      i \ar[loop above,"\id{0}"] \ar[r,"\phi(i)"]
      & * \ar[loop above,"\id{1}"] 
    \end{tikzcd}
  \end{figure}
  \begin{enumerate}
    \item $\obj{\II} = I \sqcup \set{*}$.
    \item For all $x \in \obj{\II}$, $\mor{x}{x}{\II} = \set{\id{x}}$.
    \item For all $i \in I$, $\mor{i}{*}{\II} = \set{\phi(i)}$.
  \end{enumerate}
  Then a pullback over $U$ is equivalently the limit of 
  an $\II$-diagram $X$ with $X(*) = U$. 
  Dually, a pushout under $U$ is equivalently the colimit of 
  an $\II$-codiagram $Y$ with $Y(*) = U$.
\end{dfn}