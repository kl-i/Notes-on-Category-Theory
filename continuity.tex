\begin{dfn}[(Co)Continuous Functors]
  \link{cts}

  Let $F : \CC \to \DD$.
  Then $F$ is called \emph{continuous} when 
  for all categories $\II$ and 
  $\II$-shaped diagrams $X$ with a limit $(U,u)$, 
  $(F(U),F(u))$ is a limit of the $\II$-shaped diagram $F(X) := F \circ X$. 
  More succinctly by abuse of notation, $F(\LIM X) = \LIM F(X)$.

  Dually, $F$ is called \emph{cocontinuous} when
  or all categories $\II$ and $\II$-shaped diagrams $X$ with
  a colimit $(U,u)$,
  $(F(U,F(u)))$ is a colimit of the $\II$-shaped diagram $F \circ X$.
  More succinctly by abuse of notation, $F(\COLIM X) = \COLIM F(X)$.
\end{dfn}

\begin{rmk}
  The focus of this section is to prove that
  \emph{right adjoints are continuous} and 
  \emph{left adjoints are cocontinuous}.
  This is achieved by the special case of the (co)Yoneda embedding. 

  What this says is that via adjunctions,
  one can transfer constructions in one category to another.
\end{rmk}

\begin{prop}[Naturality of Taking Constant Diagrams]
  \link{nat_const}
  
  Let $\II$ be a category. 
  Then the collection of $(\De : \CC \to \CC^\II)_{\CC \in \obj{\CAT}}$
  is a natural transformation from $\id{\CAT}$ to $\CAT(\II,-) = h^\II$.
\end{prop}
\begin{proof}
  This is simply the commutativity of the following square given 
  a functor $F : \CC \to \DD$, 
  \begin{cd}
    \CC \ar[r,"F"] \ar[d,"\De"]& \DD \ar[d,"\De"] \\
    \CC^\II \ar[r,"F \circ -"] & \DD^\II
  \end{cd}
\end{proof}

\begin{prop}[(Co)Continuity of the (Co)Yoneda Embedding]
  \link{yoneda_cts}
  
  Let $X$ be an $\II$-shaped diagram in a category $\CC$.
  We can embed $X$ into $\SET^{\CC\op}$ by the Yoneda embedding to get 
  the $\II$-shaped diagram $h_X := \CC(-,X)$.
  There is an obvious morphism 
  $\pi_X \in \brkt{\SET^{\CC\op}}^\II\brkt{\De(\CC^\II(\De(-),X)),h_X}$.

  Then $(\CC^\II(\De(-),X),\pi_X)$ is a limit of $h_X$ and 
  for every $U \in \obj{\CC\op}$, the limit of $h_X(U)$ exists. 
  \footnote{
    Note that technically, 
    we don't need to carry around this second piece of data since 
    completeness of $\SET$ implies completeness of $\SET^\II$.
    I have decided to keep it so that should the reader have decided to skip 
    the completeness results of the previous section, 
    the proofs in this section still carry through
    without completeness assumptions.
  }
  Hence for any $(U,u) \in \obj{\De\darrow X}$,
  \[
    (U,u) \text{ is a limit of } X \iff 
    (h_U,h_u) \text{ is a limit of } h_X
  \]
  More succinctly with abuse of notation, 
  $h_{\LIM X} = \LIM h_X$ if either limit exists
  (then they both exist).

  Dually, we can embed $X$ into $\SET^{\CC}$ by the Yoneda embedding to get 
  the $\II\op$-shaped diagram $h^X := \CC(X,-)$.
  There is again an obvious morphism 
  $\pi_X \in \brkt{\SET^\CC}^{\II}\brkt{\De(\CC^\II(X,\De(-))),h^X}$.

  Then $(\CC^\II(X,\De(-)),\pi_X)$ is a limit of $h^X$ and 
  for every $U \in \obj{\CC}$, the limit of $h^X(U)$ exists. 
  Hence, for any $(U,u) \in \obj{X\darrow\De}$,
  \[
    (U,u) \text{ is a colimit of } X \iff 
    (h^U,h^u) \text{ is a limit of } h^X
  \]
  More succinctly with abuse of notation, 
  $h^{\COLIM X} = \LIM h^X$ if 
  either the colimit or the limit exists (then they both exist).

\end{prop}
\begin{proof}
  By the characterisation of limits of functors,
  it suffices to show that for every $V \in \obj{\CC\op}$,
  we have that 
  $(\CC^\II(\De(V),X),(\pi_X)_V)$ is a limit of $h_X(V)$,
  where $(\pi_X)_V \in \SET^\II(\De(\CC^\II(\De(V),X)),h_X(V))$ is obtained by 
  restricting $\pi_X$ in the second argument to $V$.
  The essence is that ``a cone over $X$ is nothing more than 
  a compatible system of morphisms from the tip to $X$''.

  Let $(S,\si) \in \obj{\De\darrow\CC(V,X)}$.
  We seek a unique morphism $(S,\si) \to (\CC^\II(\De(V),X),(\pi_X)_V)$.
  By function extensionality, 
  a collection of set morphisms $(\set{s} \to \CC^\II(\De(V),X))_{s\in S}$
  glues uniquely to a function $S \to \CC^\II(\De(V),X)$.
  It thus suffice to do the case of $S = \set{*}$, a singleton.
  \begin{cd}
    \De(\set{*}) \ar[r,dashed,"\De(\LIM \si)"] 
      \ar[rd,"\si"{swap}] & 
    \De(\CC^\II(\De(V),X)) \ar[d,"(\pi_X)_V"]\\
    & \CC(V,X)
  \end{cd}
  Then $\si : \De(\set{*}) \to \CC(V,X)$ is equivalent to the data of
  a morphism $\De(V) \to X$, i.e. an element of $\CC^\II(\De(V),X)$.
  This gives a unique $\LIM \si : \set{*} \to \CC^\II(\De(V),X)$ such that 
  $(\pi_X)_V \circ \De(\LIM \si) = \si$ as desired.
  Hence, $(\CC^\II(\De(-),X),\pi_X)$ is a limit of $h_X$.

  % \textit{[UP of $\CC^\II(\De(-),X)$]}
  % Let $(F,\phi) \in \obj{\De\darrow \CC(-,X)}$.
  % Then for every $V \in \obj{\CC\op}$, 
  % the morphism $\phi \in {\SET^{\CC\op}}^\II(\De(F),\CC(-,X))$ restricts 
  % to a morphism $(\phi)_V \in \SET^\II(\De(F(V)),\CC(V,X))$.
  % \begin{cd}
  %   \De(F(V)) \ar[r,"\De(\LIM (\phi)_V)", dashed] \ar[rd,"(\phi)_V"{swap}] & 
  %   \De(\CC^\II(\De(V),X)) \ar[d,"(\pi_X)_V"] &
  %   \De(F) \ar[r,"\De(\LIM \phi)", dashed] \ar[rd,"\phi"{swap}] & 
  %   \De(\CC^\II(\De(-),X)) \ar[d,"\pi_X"] \\
  %   & 
  %   \CC(V,X) &
  %   & 
  %   \CC(-,X)
  % \end{cd}
  % For every $V \in \obj{\CC\op}$,
  % by the universal property of $\CC^\II(\De(V),X)$,
  % we have a unique morphism 
  % $\LIM (\phi)_V : (F(V),(\phi)_V) \to (\CC^\II(\De(V),X),(\pi_X)_V)$.
  % It also follows from the UP of $\CC^\II(\De(V),X)$ that 
  % the left triangle above is (contravariantly) functorial in $V$,
  % and hence $(\LIM (\phi)_V)_{V \in \obj{\CC\op}}$ forms a 
  % morphism $\LIM \phi : (F,\phi) \to (\CC^\II(\De(-),X),\pi_X)$.
  % The uniqueness of $\LIM (\phi)_V$ for every $V$ then implies 
  % the uniqueness of $\LIM \phi$.
  % Hence, $\CC^\II(\De(-),X)$ has the UP required for being the 
  % limit of $\CC(-,X)$.

  Now, $(U,u)$ is a limit of $X$ \linkto{limit}{if and only if} 
  $\CC(-,U) \iso \CC^\II(\De(-),X)$ where $\id{U}$ corresponds to $u$,
  if and only if $(h_U,h_u)$ is isomorphic to $(\CC^\II(\De(-),X),\pi_X)$
  in $\De\darrow h_X$,
  \linkto{iso_uni_implies_uni}{if and only if} $(h_U,h_u)$ is a limit of $h_X$.

  The dual argument is similar, the only minor discrepancy being 
  the conversion of colimits into limits.
\end{proof}

\begin{rmk}
  Note that what we not just shown continuity of the Yoneda embedding,
  but also that ``limits in its image pull back to limits''.
\end{rmk}

\begin{prop}[Right Adjoints Continuous,
  Left Adjoints Cocontinuous]\link{adjoint_cts}
  
  Let $R : \CC \to \DD$, $L : \DD \to \CC$ be functors such that 
  $(L,R)$ forms an adjunctions. 
  Let $\II$ be a category.
  Then 
  \begin{itemize}
    \item Given an $\II$-shaped diagram $X$ in $\CC$ with a limit $(U,u)$,
    $(R(U),R(u))$ is a limit of the diagram $R(X) := R \circ X$.
    More succinctly with abuse of notation, $R(\LIM X) = \LIM R(X)$.
    \item Given an $\II$-shaped diagram $X$ in $\DD$ with colimit $(U,u)$, 
    $(L(U),L(u))$ is a colimit of the diagram $L(X) := L \circ X$.
    More succinctly with abuse of notation, $L(\COLIM X) = \COLIM L(X)$.
  \end{itemize} 
\end{prop}
\begin{proof}
  % Let $X$ be an $\II$-shaped diagram in $\CC$ and 
  % $(\LIM X, \pi_X) \in \obj{\De\darrow X}$ the limit of $X$. 
  % To show $(R(\LIM X),R(\pi_X))$ is a limit of $R(X)$,
  % it \linkto{yoneda_cts}{suffices} to show that 
  % $(h_{R(\LIM X)},h_{R(\pi_X)}) \in \obj{\De\darrow h_{R(X)}}$
  % is a limit of $h_{R(X)}$, the embedding of $R(X)$ under Yoneda. 
  % The proof is contained in the following diagrams : 
  % \begin{cd}
  %   \De(\LIM X) \ar[d,"\pi_X"] &
  %   \De(R(\LIM X)) \ar[d,"R(\pi_X)"] & 
  %   \De(h_{R(\LIM X)}) \ar[d,"h_{R(\pi_X)}"] \ar[r,"\sim"] & 
  %   \De(\CC(L(-),\LIM X)) \ar[d,"{\CC(L(-),\pi_X)}"] \\
  %   X & 
  %   R(X) & 
  %   \DD(-,R(X)) \ar[r,"\sim"] & 
  %   \CC(L(-),X)  
  % \end{cd}
  % Put the left through $R$ to obtain the middle,
  % then embed through Yoneda to obtain the left vertical arrow in the 
  % square on the right. 
  % The right vertical arrow of the square is the image of 
  % $\pi_X$ under the functor $U \mapsto h_U \circ L$.
  % The horizontal isomorphisms of the square are from the adjunction and 
  % the square commutes by naturality of the adjunction. 
  % But by continuity of Yoneda, 
  % $(h_{\LIM X},h_{\pi_X})$ is a limit of $h_X$.
  % In particular, for any $V \in \obj{\CC\op}$,
  % $(\CC(V,\LIM X),\CC(V,\pi_X)$ is a limit of $h_X(V) = \CC(V,X)$.
  % It follows that for any $W \in \obj{\DD\op}$,
  % $(\CC(L(W),\LIM X),\CC(L(W),\pi_X))$ is a limit of $\CC(L(W),X)$. 
  % Hence $(\CC(L(-),\LIM X),\CC(L(-),X)$ is a limit of $\CC(L(-),X)$.
  % The horizontal isomorphisms then gives us that 
  % $(h_{R(\LIM X)}, h_{R(\pi_X)})$ is 
  % a limit of $h_{R(X)} = \DD(-,R(X))$ as desired. 

  Consider the following not necessarily commutative diagram : 
  \begin{cd}
    \CC \ar[r,"h_\star"] \ar[d,"R"{swap}] & 
    \SET^{\CC\op} \ar[d,"(-\circ L\op)"] \\
    \DD \ar[r,"h_\star"{swap}] & 
    \SET^{\DD\op}
  \end{cd}
  Starting on the top left,
  putting $u : \De(U) \to X$ through to the bottom right via 
  the two different paths yields the following square 
  (using \linkto{nat_const}{naturality of constant diagrams} along the way): 
  \begin{cd}
    \De(h_{R(U)}) \ar[d,"h_{R(u)}"] \ar[r,"\sim"] & 
    \De(h_{U}\circ L\op) \ar[d,"{h_u \circ L\op}"] \\
    h_{R(X)} \ar[r,"\sim"] & 
    h_X \circ L\op
  \end{cd}
  The horizontal isomorphisms come from the adjunction and 
  the square commutes by naturality of the adjunction isomorphism.
  For $(R(U),R(u))$ to be a limit of $R(X)$, 
  by continuity of the Yoneda embedding it suffices that 
  $(h_{R(U)},h_{R(u)})$ is a limit of $h_{R(X)}$.
  Then by the horizontal isomorphisms, 
  it suffice that $(h_U \circ L\op, h_u \circ L\op)$ is 
  a limit of $h_X \circ L\op$.

  By the \linkto{char_lim_funk}{characterisation of limits of functors},
  it suffices for every $V \in \obj{\DD\op}$ that the corresponding component
  $(h_U(L\op(V)),(h_u\circ L\op)_V)$ is a limit of $h_X(L\op(V))$.
  Well, since $(U,u)$ is a limit of $X$,
  by continuity of the Yoneda embedding, 
  we have $(h_U,h_u)$ is a limit of $h_X$ and for every $V \in \obj{\CC\op}$,
  the limit of $h_X(V)$ exists. 
  Then again by the 
  \linkto{char_lim_funk}{characterisation of limits of functors},
  we have in fact that the components of $(h_U,h_u)$ are 
  limits of the components of $h_X$.
  In particular, for any $V \in \obj{\DD\op}$,
  $(h_U(L\op(V)),(h_u)_{L\op(V)})$ is a limit of $h_X(L\op(V))$ as desired.
  
  The dual argument is similar. 
\end{proof}

\begin{rmk}
  The usual proof of the above given in texts is 
  the following line of ``equalities'' : 
  \begin{align*}
    \DD(-,R(\LIM X)) = \CC(L(-),\LIM X)
    = \LIM \CC(L(-),X)
    = \LIM \DD(-,R(X))
  \end{align*}
  from which one concludes $R(\LIM X) = \LIM R(X)$ by 
  \linkto{yoneda_cts}{continuity of the Yoneda embedding}.
  The second and third ``equality'' can be obtained if 
  one equates all limits of the same diagram. 

  However, the first equality is just a flat out lie.
  It corresponds to the top horizontal isomorphism in the second square 
  of the proof given here. 
  The usual proof can then be remedied by 
  attaching an extra ``equality'' at the end ``$= \DD(-,\LIM R(X))$'',
  for then by the Yoneda embedding being \linkto{yoneda}{fully faithful},
  one obtains an isomorphism $R(\LIM X) \iso \LIM R(X)$.
  This does indeed prove that ``$R(\LIM X)$'' is a limit, however, 
  we have lost the information about the morphism to the diagram $R(X)$.
  There is still work to be done to show that
  the morphism from $\De(R(\LIM X))$ to $R(X)$ is really 
  the image of the original morphism $\pi_X : \De(\LIM X) \to X$ under $R$.
  This work is precisely the commutativity of the 
  second square in the proof given here.
\end{rmk}

\begin{cor}[(Co)Limits commute with (Co)Limits]
  
  (TODO)
\end{cor}

\begin{dfn}[Filtered Categories, Diagrams, Colimits]\label{filtered}
  
  (TODO)
\end{dfn}

\begin{prop}[Characterisation of Filtered Diagrams in terms of $\SET$]
  \label{filtered_colimits_commute}

  (TODO)
\end{prop}