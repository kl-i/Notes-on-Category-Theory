\begin{dfn}[Functors]\link{functors}
  
  Let $\CC, \DD$ be categories. 
  Then a \emph{functor $F$ from $\CC$ to $\DD$} is defined by 
  the following data :
  \begin{enumerate}
    \item A map of objects $\obj{\CC} \to \obj{\DD}$,
    which we will denote by the same name $F$. 
    \item A map of morphisms for all $U, V \in \obj{\CC}$, 
    $\mor{U}{V}{\CC} \to \mor{F(U)}{F(V)}{\DD}$,
    which we will also denote by the same name $F$. 
    \item (Compositions are Preserved)
    For all $f : \map{U}{V}{\CC}{}$ and $g : \map{V}{W}{\CC}{}$, 
    $F(g \circ f) = F(g) \circ F(f)$. 
    \item (Identity Morphisms are Preserved)
    For all $U \in \obj{\CC}$, $F(\id{U}) = \id{F(U)}$.
  \end{enumerate}
\end{dfn}

\begin{dfn}[Category of Categories]\hypertarget{bigcat}{}
  
  We define the \emph{category of categories} $\CAT$, 
  \begin{enumerate}
    \item $\obj{\CAT}$ consists of categories.\footnote{
      This is technically not a set since it contains the category $\SET$,
      but we do not lose sleep over such issues. 
    }
    \item For $\CC, \DD \in \obj{\CAT}$, 
    $\mor{\CC}{\DD}{\CAT}$ consists of functors from $\CC$ to $\DD$.  
    \item For $\CC \in \obj{\CAT}$, $\id{\CC}$ is the obvious thing.
  \end{enumerate}
\end{dfn}

\begin{dfn}[Faithful, Full, Fully Faithful]\link{fully_faithful}
  
  Let $F : \map{\CC}{\DD}{\CAT}{}$.
  Then $F$ is called 
  \begin{enumerate}
    \item \emph{faithful} when for all $U, V \in \obj{\CC}$,
    $F : \mor{U}{V}{\CC} \to \mor{F(U)}{F(V)}{\DD}$ is injective. 
    \item \emph{full} when for all $U, V \in \obj{\CC}$,
    $F : \mor{U}{V}{\CC} \to \mor{F(U)}{F(V)}{\DD}$ is surjective.
    \item \emph{fully faithful} when for all $U, V \in \obj{\CC}$,
    $F : \mor{U}{V}{\CC} \to \mor{F(U)}{F(V)}{\DD}$ is bijective.
  \end{enumerate}
\end{dfn}

\begin{prop}[Fully Faithful Functors are Injective]
  \link{full_faith_inj}
  
  Let $F : \map{\CC}{\DD}{\CAT}{}$ be fully faithful,
  $U, V \in \obj{\CC}$ such that $F(U) \iso F(V)$.
  Then $U \iso V$. 
\end{prop}
\begin{proof}
  Let $f_1 \in \DD(F(U),F(V))$ and $f_2 \in \DD(F(V),F(U))$ such that 
  $\id{F(U)} = f_2 \circ f_1$ and $\id{F(V)} = f_1 \circ f_2$. 
  Then $f_1, f_2$ corresponds respectively to $g_1, g_2 \in \CC(U,V), \CC(V,U)$
  through $F$. 
  We thus have 
  \[ 
    F(g_2 \circ g_1) = F(g_2) \circ F(g_1) = f_2 \circ f_1 = \id{F(U)}
    = F(\id{U})
  \]
  which by $F$ fully faithful gives $g_2 \circ g_1 = \id{U}$. 
  Similarly, $g_1 \circ g_2 = \id{V}$.
\end{proof}

\begin{dfn}[Natural Transformations]\link{natural}
  
  Let $F, G : \map{\CC}{\DD}{\CAT}{}$. 
  Then a \emph{natural transformation $\eta$ from $F$ to $G$} is defined by 
  the following data : 
  \begin{enumerate}
    \item For all $U \in \obj{\CC}$, $\eta_U : \map{F(U)}{G(U)}{\DD}{}$. 
    \item (Naturality) 
    For all $U, V \in \obj{\CC}$ and $f : \map{U}{V}{\CC}{}$, 
    we have the following commutative diagram. \begin{figure}[H]
      \centering
      \begin{tikzcd}
        F(U) \arrow[r,"\eta_U"] \arrow[d,"F(f)"{swap}] & G(U) \arrow[d,"G(f)"] \\
        F(V) \arrow[r,"\eta_V"{swap}] & G(V) 
      \end{tikzcd}
    \end{figure}
  \end{enumerate}
\end{dfn}

\begin{dfn}[Category of Functors]\link{cat_functor}
  
  Let $\CC, \DD \in \obj{\CAT}$. 
  Then the \emph{category of functors from $\CC$ to $\DD$},
  denoted $\DD^\CC$, is defined by 
  \begin{enumerate}
    \item $\obj{\DD^\CC} := \mor{\CC}{\DD}{\CAT}$. 
    \item For all $F, G \in \obj{\DD^\CC}$, 
    $\mor{F}{G}{\DD^\CC} := $
    the set of natural transformations from $F$ to $G$. 
    \item For all $F \in \obj{\DD^\CC}$, $\id{F}$ is the obvious thing.
    \item The obvious way to define composition of natural tranformations is ``component-wise''.
  \end{enumerate}
\end{dfn}

\begin{dfn}[Equivalence of Categories]\link{equiv}
  
  Let $\CC, \DD$ be categories, $F \in \CAT(\CC,\DD)$.
  Then $F$ is called an \emph{equivalence of categories} when 
  there exists $G \in \CAT(\DD,\CC)$ such that  
  $G \circ F \iso \id{\CC}$ and $F\circ G \iso \id{\DD}$. 
\end{dfn}

\begin{dfn}[Essentially Surjective]\link{surj}
  
  Let $\CC,\DD$ be categories and $F : \map{\CC}{\DD}{\CAT}{}$. 
  The \emph{essential image of $F$} is defined as
  the set of $X \in \DD$ such that 
  there exists $U \in \CC$ where $F(U) \iso X$.
  $F$ is called \emph{essentially surjective} when
  its essential image is the whole of $\DD$.
\end{dfn}

\begin{prop}[Characterisation of Equivalence of Categories]\link{char_equiv}
  
  Let $F : \map{\CC}{\DD}{\CAT}{}$. 
  Then $F$ is an equivalence of categories if and only if 
  $F$ is fully faithful and essentially surjective. 
\end{prop}
\begin{proof}~
  $(\implies)$
  Let $G : \map{\DD}{\CC}{\CAT}{}$, $\ep : \map{\id{\CC}}{G \circ F}{}{\sim}$,
  $\eta : \map{F \circ G}{\id{\DD}}{}{\sim}$.
  It is clear that $F$ is essentially surjective. 
  For faithful, let $f, g \in \CC(U,V)$ such that 
  $F(f) = F(g)$.
  Then by naturality of $\ep$, we have the following commutative diagram : 
  \begin{cd}
    U \ar[r,"f"] \ar[d,"\ep_U"] & V \ar[d,"\ep_V"] \\
    GF(U) \ar[r,"GF(f)"] & GF(V)
  \end{cd}
  Then $f = g$ follows from $\ep_U, \ep_V$ being isomorphisms. 

  For fullness, let $f \in \DD(F(U), F(V))$.
  The guess is that by mapping $f$ back to $\CC$, 
  we get the morphism that maps to $f$.
  That is, we claim that $F(\ep_V\inv \circ G(f) \circ \ep_U) = f$.
  Since the arguments of the above paragraph also applies to $G$,
  we have $G$ is faithful and hence 
  it suffices to show that $GF(\ep_V\inv \circ G(f) \circ \ep_U) = G(f)$.
  We first show that perhaps unsurprisingly, 
  $GF(\ep_V\inv) = \ep_{GF(V)}\inv$. 
  By functoriality of $GF$, it suffices to show that
  $GF(\ep_V) = \ep_{GF(V)}$. 
  This follows from $\ep_V$ being an isomorphism and 
  the following commutative diagram due to
  the naturality of $\ep$ : 
  \begin{cd}
    V \ar[r,"\ep_V"] \ar[d,"\ep_V"]& GF(V) \ar[d,"GF(\ep_V)"] \\
    GF(V) \ar[r,"\ep_{GF(V)}"] & GFGF(V) 
  \end{cd}
  It now remains to show $GF(G(f) \circ \ep_U) = \ep_{GF(V)} \circ G(f)$.
  This follows from naturality of $\ep$ and $\ep_U$ being an isomorphism : 
  \begin{cd}
    U \ar[r,"G(f) \circ \ep_U"] \ar[d,"\ep_U"] & GF(V) \ar[d,"\ep_{GF(V)}"]
    \\
    GF(U) \ar[r,"GF(G(f) \circ \ep_U)"{swap,yshift = -1mm}] & GFGF(V)
  \end{cd}

  $(\limplies)$
  Using the axiom of choice, 
  for each $X \in \obj{\DD}$, let $G(X) \in \obj{\CC}$ and 
  $\eta_X \in \DD(FG(X), X)$ such that $\eta_X$ is an isomorphism. 
  For $f \in \DD(X, Y)$, 
  by full faithfulness of $F$ let $G(f) \in \CC(G(X),G(Y))$ be
  the unique morphism such that $FG(f) = \eta_Y\inv \circ f \eta_X$. 
  It then follows from uniqueness of the above morphisms and 
  functoriality of $F$ that 
  $G$ is a functor. 
  Note that by construction, 
  the collection of $\eta_X$ gives a natural isomorphism 
  $\eta : F \circ G \to \id{\DD}$.
  
  It remains to give a natural isomorphism $\ep : \id{\CC} \to G \circ F$.
  For $U \in \obj{\CC}$, we are looking for a morphism $U \to GF(U)$.
  Feeling optimistic, we use full and faithfulness of $F$ to define 
  $\ep_U \in \CC(U, GF(U))$ as the unique morphism such that 
  $F(\ep_U) = \eta_{F(U)} \inv$. 
  From $\eta_{F(U)}$ being an isomorphism and full faithfulness of $F$, 
  it follows that $\ep_U$ is also an isomorphism. 
  Finally, to check naturality of $\ep$, 
  let $f \in \CC(U,V)$.
  We need $\ep_V \circ f = GF(f) \circ \ep_U$.
  But since $F$ is faithful, 
  it suffices that $F$ applied to these morphisms are equal. 
  Well, indeed we have it \[
    F(\ep_V \circ f) = \eta_{F(V)}\inv \circ F(f)
    = \eta_{F(V)}\inv \circ F(f) \circ \eta_U \circ \eta_U\inv 
    = \eta_{F(V)}\inv \circ F(f) \circ \eta_U \circ F(\ep_U)
    = F(GF(f) \circ \ep_U)
  \]
\end{proof}

\begin{eg}[Functors and Natural Transformations]\link{eg:functors}
  
  The following is but a small sample of the vast sea of functors that
  appear in mathematics. 

  The first list concerns 
  ``freely creating structure from a subcategory to an ambient category''.
  It begins with the concept of a \emph{forgetful functor}.
  Details of this are explained \linkto{free}{in the section on adjunctions}. 
  \begin{itemize}
    \item (Forgetful functor)
    Given any subcategory $\CC$ of $\DD$, 
    there is an ``obvious'' functor from $\CC$ to $\DD$ that maps 
    $\obj{\CC} \to \obj{\DD}$ by doing nothing and 
    morphisms in $\CC$ to morphisms in $\DD$ by doing nothing. 
    Functors of this form are called \emph{forgetful functors}.

    Here is a graph showing subcategories of set and their ``inclusions''.
    \begin{cd}
      \SET & 
      \GRP \ar[l] & 
      \AB \ar[l] & 
      \RING \ar[l] &
      \CRING \ar[l]
       \\
      \TOP \ar[u] & 
      & 
      R\MOD \ar[u] & 
      \ALG(R) \ar[l] \ar[u] & 
      \CALG(R) \ar[l] \ar[u]
    \end{cd}
    For the category $R\MOD$ of left modules over $R$, $R$ need only be a ring. 
    For $\ALG(R)$, $R$ needs to be commutative with unity.
    The maps from $\RING, R\MOD$ into $\AB$ takes
    the underlying abelian groups. 

    In particular, we call a subcategory $\CC$ of $\DD$,
    we say $\CC$ is a \emph{full subcategory of $\DD$} when 
    the forgetful functor is full. 
    Faithfulness is given since morphism sets of $\CC$ are 
    by definition subsets of morphism sets of $\DD$.

    \item (Free Group)
    For each set $S$, 
    the \emph{free group over $S$} is an object $\<S\> \in \GRP$
    the comes with a morphism of sets $\lift{}{} : S \to \<S\>$ such that
    for any group $G$ and $\phi \in \SET(S,G)$,
    there is a unique morphism of groups $\<\phi\> \in \GRP(\<S\>,G)$ such that
    $\<\phi\> \circ \lift{}{} = \phi$.
    This makes $G \mapsto \<G\>$ into a functor from $\SET$ to $\GRP$.

    In particular, for $S \subs G$ and $f$ the usual inclusion, 
    $S$ is called \emph{generating} when $\<\io\>$ is surjective,
    where $\io \in \SET(S,G)$ is the usual inclusion. 

    \item (Free Module over a Ring)
    Let $R$ be a ring. 
    For each set $S$, 
    the \emph{free left $R$-module over $S$} is 
    an object $R^{\oplus S} \in R\MOD$ 
    that comes with a morphism of sets $\lift{}{} : S \to R^{\oplus S}$
    such that for any $R$-module $M$ and $f \in \SET(S,M)$,
    there is a unique $R$-linear map 
    $R^{\oplus f} : \oplus_{s \in S} R \to M$ such that 
    $R^{\oplus f} \circ \lift{}{} = f$. 
    This makes $S \mapsto R^{\oplus S}$ into a functor 
    from $\SET$ to $R\MOD$.

    In particular, for $S \subs M$ and $f$ the usual inclusion, 
    $S$ is called respectively linearly independent, spanning, a basis
    when $R^{\oplus f}$ is injective, surjective, an isomorphism.

    Note that the above covers $\AB$, since 
    $\AB$ is nothing more than $\Z\MOD$.

    \item (Free Commutative Algebra over a Ring)
    Let $K$ be a commutative ring. 
    For each set $S$, 
    the free commutative $K$-algebra over $S$ is an object $K[S] \in \CALG(K)$
    that comes with a morphism of sets $\lift{}{} : S \to K[S]$ such that
    for any $K$-algebra $A$ and $a \in \SET(S,A)$,
    there exists a unique $K$-algebra morphism $ev_a : K[S] \to A$
    such that $ev_a \circ \lift{}{} = a$.
    This makes $S \mapsto K[S]$ into a functor 
    from $\SET$ to $\CALG(K)$.

    These free commutative algebras are not unfamiliar.
    For instance, 
    the polynomial ring in $K[T]$ over $K$ is
    precisely $K[\set{*}]$ where $\set{*}$ is the singleton set.
    For any commutative $K$-algebra $A$,
    a set morphism $a : \set{*} \to A$ is nothing more than
    an element in $A$.
    So as suggested by the notation, $ev_a$ is precisely evaluation 
    of polynomials $f \mapsto f(a)$ where we have identified 
    the set morphism $a$ with the unique element in its image. 
    Generalizing, for an arbitrary set $S$,
    $K[S]$ is precisely the commutative $K$-algebra of polynomials 
    with variables indexed by $S$. 
    In particular, for a commutative $K$-algebra $A$ and $S \subs A$,
    $S$ is called respectively algebraically independent over $A$, generating
    when $ev_S : K[S] \to A$ is injective, surjective. 

    \item (Tensor Product, Extension and Contraction of Scalars)
    Let $B$ be a commutative $A$-algebra where $A$ is a commutative ring. 
    Every $B$-module $N$ has an obvious $A$-module structure. 
    This gives a forgetful functor from $B\MOD$ to $A\MOD$.
  
    For any $M \in \obj{A\MOD}$, 
    $M \otimes_A B$ the extension of scalars of $M$ has the property that
    it comes with an obvious morphism of $A$-modules 
    $\lift{}{} : M \to M \otimes_A B$ such that 
    for any $N \in \obj{B\MOD}$ and $f \in A\MOD(M,N)$, 
    there exists a unique $A$-module morphism 
    $f \otimes_A \id{B} : M \otimes_A B \to N$ such that 
    $(f\otimes_A \id{B}) \circ \lift{}{} = f$.

    In analogy with the prior examples,
    extension of scalars can be seen as 
    ``taking the free $B$-module over an $A$-module''.

    \item (Localization of Modules)
    Let $A$ be a commutative ring and $S \subs A$ multiplicative. 
    Define the category $S\inv A\MOD$ as the full subcategory of $S\inv A\MOD$ 
    with objects consisting of $M \in \obj{A\MOD}$ such that 
    for all $f \in S$, scalar multiplication by $f$ on $M$ is an isomorphism,
    i.e. $f$ is an ``invertible'' scalar for $M$.
    We then have a forgetful functor from $S\inv A\MOD$ to $A\MOD$.

    Now, for an $A$-module $M$, 
    the localization of $M$ with respect to $S$ is 
    is an object $S\inv M$ of $S\inv A\MOD$ that comes with 
    an $A$-linear map $\lift{}{} : M \to S\inv M$ such that 
    for all $N \in S\inv A\MOD$ and $f \in A\MOD(M,N)$,
    there is a unique $S\inv(f) \in S\inv A\MOD(M_S,N)$ where 
    $S\inv(f) \circ \lift{}{} = f$.
    This gives a functor $A\MOD \to S\inv A\MOD, M \mapsto S\inv M$
    and morphisms are mapped to induced morphisms. 
    In particular, the localization $S\inv A$ of $A$ itself 
    has an obvious ring structure.
    One can then realize $S\inv A\MOD$ as 
    the category of modules over $S\inv A$.

    In analogy with previous examples,
    $S\inv M$ can be seen as taking $M$ and 
    ``freely inverting the scalars in $S$''.

    \item (Symmetric Algebra)
    The following is similar to the free commutative algebra construction.
    Let $A$ be a commutative ring.
    Then for any $A$-module $M$,
    the symmetric algebra $Symm\,M$ is an object in $A\CALG$ 
    that comes with an $A$-linear map $\lift{}{} : M \to Symm\,M$
    such that for any commutative $A$-algebra $B$ and $\phi \in A\MOD(M,B)$,
    there exists a unique $A$-algebra morphism $Symm\,\phi : Symm\,M \to B$
    such that $Symm\,\phi \circ \lift{}{} = \phi$.
    In analogy with the prior examples,
    this may be seen as taking the ``free commutative $A$-algebra over $M$''.

    \item (Discrete Topology)
    For any set $X$, $(X,2^X)$ where $2^X$ is the powerset of $X$
    is a topological space. 
    Then for any topological space $Y$ and $f \in \SET(X,Y)$,
    $f$ is automatically continuous with respect to the discrete topology $2^X$.
    This gives rise to a functor $\SET \to \TOP$.
    In analogy with the prior examples, 
    this seen as taking the ``free topological space on $X$''.

    \item (Discrete Category)
    Given any set $S$, 
    you can give it the structure of a category by
    declaring $\obj{S} := S$ and the only morphisms to be identities. 
    Then for any category $\CC$ and function $f : S \to \obj(\CC)$,
    there is a unique functor $f_\CAT : S \to \CC$ that 
    agrees with $f$ on the objects. 
    This can be seen as the ``free category over $S$''.

  \end{itemize}

  The next list is themed ``moving structures on objects across morphisms''.
  \begin{itemize}
    \item (Image, Preimage of ``Subobjects'')
    Let $f : U \to V$ be a set morphism. 
    Then taking the image of subsets of $U$ under $f$
    gives a functor $f : 2^U \to 2^V$.
    In the other direction,
    taking preimages of subsets of $V$ under $f$ gives 
    a functor $f\inv : 2^V \to 2^U$.

    If $f : U \to V$ is a morphism of groups,
    the above functors restrict to \begin{align*}
      f_\GRP : \SUB\GRP(U) \to \SUB\GRP(V) \\
      f\inv_\GRP : \SUB\GRP(V) \to \SUB\GRP(U)
    \end{align*}

    Similar results for when $f$ is a morphism of $R\MOD, R\ALG$.

    \item (Image, Preimage of Filters)
    Let $f : X \to Y$ be a morphism of sets. 
    Then the functor $f : 2^X \to 2^Y$ gives rise to 
    the functor $f\inv : 2^{2^Y} \to 2^{2^X}$,
    which restricts to a functor $f\inv_{\fil} : \fil Y \to \fil X$.
    This is the \emph{inverse image functor for filters}.
    
    Unfortunately, given a $F \in \fil(X)$,
    the set of subsets $\set{f U \st U \in F}$ does not
    in general form a filter of $Y$. 
    However, 
    the set of subsets 
    $f_{\fil} F := \set{W \in 2^Y \st \exists\, V \in F, f V \subs W }$ 
    does form a filter on $Y$. 
    This is equivalently the largest filter on $Y$ such that 
    its inverse image filter is in $F$.
    It is not hard to check that 
    $f_{\fil} : \fil X \to \fil Y$ is functorial with respect to 
    the category structure on $\fil X, \fil Y$ given by 
    their $\POSET$ structure.

    \item (``Functions" into a Fixed Object)
    For any topological space $X$, 
    $\TOP(X,\R)$ with pointwise addition and multiplication is
    an object in $\R\CALG$.
    Call $\TOP(X,\R)$ the \emph{global functions on $X$}.
    Then for any topological morphism $\phi : X \to Y$ and 
    we naturally get a set morphism 
    $(- \circ \phi) : \TOP(X,\R) \to \TOP(Y,\R)$ which turns out to be 
    a morphism of $\R$-algebras. 
    This yields a functor, not from $\TOP$ to $\R\CALG$, but rather 
    ``$\TOP\op$'' to $\R\CALG$  where 
    ``$\TOP\op$'' looks like $\TOP$ but all morphisms are reversed. 
    See \linkto{dual}{later} for details. 

    Many situations analogous to the above can be found. 
    For instance, replacing $\TOP$ with $\SET$ and $\R$ with $\set{0,1}$
    yields the 
    
  \end{itemize}

  This list contains more exotic ``algebraic constructions''.
  \begin{itemize}
    \item Fundamental groups
    \item Singular Complex
    \item Classical Galois Correspondence
    \item Vanishing, Ideal
    \item Spec of a commutative ring
  \end{itemize}

  The final list consists of miscellaneous ``algebraic constructions'' :
  \begin{itemize}
    \item (Group Algebra)
    The following is similar to the free algebra construction. 
    Let $K$ be a commutative ring.
    Then for $A \in \ALG(K)$, $A^\times \in \AB$.
    Any $f \in \ALG(K)(A,B)$, 
    let $f^\times$ denote the restriction of $f$ onto $A^\times$.
    Then $f^\times$ is automatically a morphism of abelian groups
    (keeping in mind the group operation is multiplication).
    
    ``Conversely'',
    for any abelian group $G$,
    the group $K$-algebra over $G$ is a $K$-algebra $K[G]$ that comes with
    a morphism of abelian groups $\lift{}{} : G \to K[G]^\times$ such that
    for any other $K$-algebra $A$ and $\phi \in \AB(G,A^\times)$,
    there exists a unique $K[\phi] \in \ALG(K)(K[G],A)$ such that 
    $K[\phi] \circ \lift{}{} = \phi$. 
    This property makes $G \mapsto K[G]$ into a functor from 
    $\AB$ to $\ALG(K)$.
    In particular, $K[\Z]$ is precisely the localization $K[T,T\inv]$. 

    Note that we can actually remove the commutativity hypothesis
    on groups and algebras,
    obtaining functors between $\GRP$ and 
    the category of non-commutative algebras of $K$.  
    \item (Vector Spaces with a Basis)
    Let $K$ be a field and $\CC$ the category of $K$-vector spaces
    together with a basis. 
    Define the functor $F : \CC \to \CC$ that ``takes components'' as follows : 
    \begin{itemize}
      \item For $(V,B_V)$ in $\CC$, 
      let $F((V,B_V)) := (K^{\oplus B_V}, E)$ where 
      $K^{\oplus B_V}$ is the free $K$-vector space on $B_V$ and 
      $E$ is the standard basis.
      \item For a morphism $f \in \CC((V,B_V),(W,B_W))$,
      since $f B_V \subs B_W$, 
      this determines a map from the standard basis of $K^{\oplus B_V}$ to
      the standard basis of $K^{\oplus B_W}$, 
      thus extending to a unique 
      $K$-linear map $F(f) : K^{\oplus B_V} \to K^{\oplus B_W}$.
      \item Identity morphisms are clearly respected.
      \item Composition of morphisms are clearly respected. 
    \end{itemize}
    There is a natural isomorphism between 
    ``taking components'' and the identity functor : 
    For each $(V,B_V)$ in $\CC$, 
    consider the $K$-linear map $[-]_{B_V} : V \to K^{\oplus B_V}$ that 
    that takes vectors to their components with respect to $B_V$.
    This is well-defined and an isomorphism by $B_V$ being a basis. 
    (In fact, this can serve as a \emph{definition} of $B_V$ being a basis.)
    Then we have naturality : 
    \begin{cd}
      (V,B_V) \ar[r,"f"] \ar[d,"\sqbrkt{-}_{B_V}"{swap},"\sim"] &
      (W,B_W) \ar[d,"\sqbrkt{-}_{B_W}","\sim"{swap}] \\
      (K^{\oplus B_V},E) \ar[r,"F(f)"] &
      (K^{\oplus B_W},E) \\
    \end{cd}
    In particular for a fixed $K$-vector space $V$ and 
    two finite bases $B,B_1$,
    any total ordering on $B, B_1$ gives rise to a unique 
    $f \in \CC((V,B),(V,B_1))$.
    Then the (iso)morphism $F(f)$ is 
    what is usually known as \emph{change of basis}.

    \item  Dual Module
  
    \item (Power Set as an $\F_2$-Algebra)
    For any set $X$, we can see the power set $2^X$ as $\F_2^X$
    the set of set morphisms from $X$ to the field with two elements $\F_2$.
    Then $\F_2^X$ naturally has a structure of an $\F_2$-algebra.
    Explicitly, for two subsets $f, g \in \F_2^X$,
    $f g = f \cap g$ and $f + g = (f \cup g) \minus (f \cap g)$.
    The additive identity is $\nothing$ and the multiplicative identity is $X$.
    One can see that preimage functor $X \mapsto 2^X = \F_2^X$
    upgrades to a contravariant functor from $\SET$ to $\ALG(\F_2)$.

    \item  Tangent space of pointed differentiable manifold
  \end{itemize}
\end{eg}